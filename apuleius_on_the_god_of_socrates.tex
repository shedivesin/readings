% pdflatex
\documentclass[12pt]{article}
\usepackage{ebgaramond}
\usepackage{microtype}

\title{On the God\footnote{In many places, Plato calls the participants of the
divinities Gods. Thus in the Laws a divine soul is called a God; and in the
Ph{\ae}drus it is said, ``That all the horses and charioteers of the
Gods are good, and consist of things that are good.'' And when he says this, he
is speaking of divine souls. After this also, in the same dialogue, he still
more clearly says, ``And this is the life of the Gods.'' What however is still
more admirable is this, that he denominates those beings Gods, who are always
united to the Gods, and who, together with them, give completion to one series.
For in the Ph{\ae}drus, Tim{\ae}us, and other dialogues, he
extends the appellation of the Gods as far as to d{\ae}mons, though the latter
are essentially posterior to, and subsist about the Gods. But what is still
more paradoxical, he does not refuse to call certain men Gods: for in the
Sophista he thus denominates the Elean guest or stranger.

According to Plato, therefore, one thing is a God simply, another on account of
union, another through participation, another through contact, and another
through similitude. For of super-essential natures, each is primarily a God; of
intellectual natures, each is a God according to union; and of divine souls,
each is a God according to participation. But divine d{\ae}mons are Gods
according to contact with the Gods; and the souls of men are allotted this
appellation through similitude.

As the d{\ae}mon of Socrates, therefore, was doubtless one of the highest
order, as may be inferred from the intellectual superiority of Socrates to most
other men, Apuleius is justified in calling this d{\ae}mon a God. And that the
d{\ae}mon of Socrates indeed was divine, is evident from the testimony of
Socrates himself in the First Alcibiades: for in the course of that
dialogue he clearly says, ``I have long been of opinion that \textit{the God}
did not as yet direct me to hold any conversation with you.'' And in the
Apology he most unequivocally evinces that this d{\ae}mon is allotted a divine
transcendency, considered as ranking in the order of d{\ae}mons.

The ignorance of this distinction has been the source of infinite confusion and
absurd hypotheses, to the modern writers on the mythology and theology of the
Greeks.}~~of Socrates}
\author{Apuleius \and Thomas Taylor (tr.)\footnote{T.~Taylor. \textit{The
Metamorphosis, or Golden Ass, and Philosophical Works of Apuleius.} R.~Triphook
and T.~Rodd, 1822.}}
\date{}

\begin{document}
\maketitle

\noindent Plato gives a triple division to the whole nature of things, and
especially to that part of it which pertains to animals; and he likewise is of
opinion, that there are Gods in the highest, in the middle, and in the lowest
place of the universe. Understand, however, that this division is not only
derived from local separation, but also from dignity of nature, which is itself
distinguished not by one or two, but by many modes. Nevertheless, it will be
more manifest to begin from the distribution of place;\footnote{It is here
requisite to observe, that divine natures are not in bodies, but externally
rule over them. Hence they impart from themselves to bodies every good they are
able to receive, but they themselves receive nothing from bodies; so that
neither will they derive from them certain peculiarities. By no means,
therefore, must it be admitted (as Iamblichus well observes), that the cause of
the distinction of the divine genera is an arrangement with reference to
bodies; as of Gods to ethereal bodies, but of d{\ae}mons to a\"{e}rial bodies,
and of souls to such as are terrene. See sect.~\textsc{i} chap.~\textsc{viii}
of my translation of Iamblichus on the Mysteries.} for this order assigns the
heavens to the immortal Gods, conformably to what their majesty demands.  And
of these celestial Gods, some we apprehend by the sight, but others we
investigate by intellect; and by the sight, indeed, we perceive---

\begin{verse}
---Ye, the world's most refulgent lights,\\
Who through the heavens conduct the gliding year.\footnote{These lines are
taken from book \textsc{i} of the Georgics of Virgil.}
\end{verse}

\noindent We do not, however, only perceive by the eyes those principal Gods,
the Sun the artificer of the day, and the Moon the emulator of the Sun, and the
ornament of night; whether she is cornicular, or divided\footnote{In the
original, \textit{dividua}; and the moon is \textit{dividua} when she is a
quarter old.}, or gibbous, or full; exhibiting a various ignited torch; being
more largely illuminated the farther she departs from the Sun; and, by an equal
augment both of her path and her light, defining the month through her
increments, and after wards by her equal decrements; [for this must be
admitted] whether, as the Chaldeans think, she possesses a proper and permanent
light of her own, being in one part of herself endued with light, but in
another part deprived of splendour, and possessing a manifold convolution of
her various-coloured face, and thus changes her form; or whether, being wholly
deprived of a peculiar light, and requiring extraneous splendour, with a dense
body, or with a body polished like a mirror, she receives either the oblique or
direct rays of the Sun, and, that I may use the words of Lucretius, [in
lib.~\textsc{v}]

\begin{verse}
---throws from her orb a spurious light. 
\end{verse}

\noindent Whichever of these opinions is true, for this I shall afterwards
consider, there is not any Greek, or any barbarian, who will not easily
conjecture that the Sun and Moon are Gods; and not these only, as I have said,
but also the five stars, which are commonly called by the unlearned erratic,
though, by their undeviating, certain, and established motions, they produce by
their divine revolutions the most orderly and eternal transitions; by a various
form of convolution indeed, but with a celerity perpetually equable and the
same, representing, through an admirable vicissitude, at one time progressions,
and at another regressions, according to the position, curvature, and obliquity
of their circles, which he will know in the best manner, who is skilled in the
risings and settings of the stars.

You who accord with Plato must also rank in the same number of visible Gods
those other stars,

\begin{verse}
The rainy Hyades, Arcturus, both the Bears:\footnote{This verse is taken from
book \textsc{iii} of the \textit{{\AE}neid}.}
\end{verse}

\noindent and likewise other radiant Gods, by whom we perceive, in a serene
sky, the celestial choir adorned and crowned, when the nights are painted with
a severe grace and a stern beauty; beholding, as Ennius says, in this most
perfect shield of the world, engravings diversified with admirable splendours.
There is another species of Gods, which nature has denied us the power of
seeing, and yet we may with astonishment contemplate them through intellect,
acutely surveying them with the eye of the mind. In the number of these are
those twelve Gods\footnote{These Gods form, in the Platonic theology, the
super-celestial, or liberated order, being immediately proximate to the mundane
order of Gods. Concerning these divinities, see book \textsc{vi} of my
translation of Proclus on the Theology of Plato.} which are comprehended by
Ennius, with an appropriate arrangement of their names, in two verses:

\begin{verse}
Juno, Vesta, Minerva, Ceres, Diana, Venus, Mars,\\
Mercurius, Jovi, Neptunus, Vulcanus, Apollo;
\end{verse}

\noindent and others of the like kind, whose names indeed have been for a long
time known by our ears, but whose powers are conjectured by our minds, being
perceived through the various benefits which they impart to us in the affairs
of life, in those things over which they severally preside. The crowd, however,
of the ignorant, who are rejected by Philosophy as profane, whose sanctity is
vain, who are deprived of right reason, destitute of religion, and incapable of
obtaining truth, dishonour the Gods, either by a most scrupulous worship or a
most insolent disdain of them; one part being timid through superstition, but
another tumid through contempt. Many venerate all these Gods, who are
established on the lofty summit of ether, far removed from human contagion; but
they venerate them improperly. For all fear them, but ignorantly; and a few
deny their existence, but impiously. Plato thought these Gods to be
incorporeal\footnote{The Delphin editor of this treatise, who appears to have
been perfectly ignorant of the philosophy of Plato, says, that Plato is of an
opinion contrary to what is here asserted by Apuleius, in the Epinomis
and in the Tim{\ae}us, because, in the former dialogue, he gives to
the celestial Gods \textit{a most beautiful}, and in the latter an
\textit{igneous} body. But if rational souls are incorporeal, according to
Plato, though connected with bodies, much more must this be the case with the
Gods.} and animated natures, without any end or beginning, but eternal both
with reference to the time past and the time to come; spontaneously separated
from the contagion of body; through a perfect intellect possessing supreme
beatitude; good, not through the participation of any extraneous good, but from
themselves; and able to procure for themselves every thing which is requisite,
with prompt facility, with simple, unrestrained, and absolute power. But of the
father of these, who is the lord and author of all things, and who is liberated
from all necessity of acting or suffering, not being bound by any duty to the
performance of any offices, why should I now begin to speak? Since Plato, who
was endued with celestial eloquence, when employing language worthy of the
immortal Gods, frequently proclaims that this cause of all things, on account
of his incredible and ineffable transcendency, cannot be even moderately
comprehended by any definition, through the poverty of human speech; and that
the intellectual apprehension of this God can scarcely be obtained by wise men,
when they have separated themselves from body, as much as possible, through the
vigorous energies of the mind. \textit{He also adds, that this knowledge
sometimes shines forth with a most rapid coruscation, like a bright and clear
light in the most profound darkness.}\footnote{This is a very remarkable
passage, but is not to be found in any of the writings of Plato that are now
extant. Something similar to this is said by Plato, in his seventh epistle,
respecting the intuition of \textit{idea}, or \textit{intellectual form};
viz.~``that from long converse with the thing itself, accompanied by a life in
conformity to it, on a sudden, a light, as if from a leaping fire, will be
enkindled in the soul, and will there itself nourish itself.''} I will
therefore omit the discussion of this, in which all words adequate to the
amplitude of the thing are not only wanting to me, but could not even be found
by my master Plato. Hence, I shall now sound a retreat, in things which far
surpass my mediocrity, and at length bring down my discourse from heaven to
earth, in which we men are the principal animal, though most of us, through the
neglect of good discipline, are so depraved by all errors, so imbued with the
most atrocious crimes, and have become so excessively ferocious, through having
nearly destroyed the mildness of our nature, that it may seem there is not any
animal on the earth viler than man. Our discussion, however, at present is not
concerning errors, but concerning the natural distribution of things.

Men, therefore, dwell on the earth, being endued with reason, possessing the
power of speech, having immortal souls, but mortal members, light and anxious
minds, brutal and infirm bodies, dissimilar manners, but similar errors,
pervicacious audacity, pertinacious hope, vain labour, and decaying fortune,
severally mortal, yet all of them eternal in their whole species, and mutable
in this, that they alternately leave offspring to supply their place; [and
besides all this] are conversant with fleeting time, slow wisdom, a rapid
death, and a querulous life. In the meanwhile you will have two kinds of
animals, Gods very much differing from men, in sublimity of place, in
perpetuity of life, in perfection of nature, and having no proximate
communication with them;\footnote{A divine nature is \textit{immediately}
present with all things, but all things are not immediately present with it;
because aptitude in the participant is here requisite to an union with that
which is participable.} since those supreme are separated from the lowest
habitations by such an interval of altitude; and the life there is eternal and
never-failing, but is here decaying and interrupted; and the natures there are
elevated to beatitude, but those that are here are depressed to calamity. What
then? Does nature connect itself by no bond, but leave itself separated into
the divine and human part, and suffer itself to be interrupted, and as it were
debile? For, as the same Plato says, \textit{no God is mingled with men.} But
this is a principal indication of the sublimity of the Gods, that they are not
contaminated by any contact with us.\footnote{\textit{i.~e.}~By any habitude or
alliance to our nature.} One part of them is only to be seen by us with
debilitated vision; as the stars, about whose magnitude and colour men are
still ambiguous. But the rest are only known by intellect, and not by this with
a prompt perception. This, however, must not be considered as an admirable
circumstance in the immortal Gods, since even among men, who are elevated by
the opulent gifts of Fortune to the tottering throne and pendulous tribunal of
a kingdom, the access is rare, in consequence of their living remote from
witnesses, in certain penetralia of their dignity: for familiarity produces
contempt, but infrequency conciliates admiration.

What, therefore, shall I do (some orator may object) after this decision of
yours, which is indeed celestial, but inhuman [or foreign from human nature]?
If men are entirely removed far from the immortal Gods, and are so banished
into these Tartarean realms of earth that all communication with the celestial
Gods is denied to them, nor any one of the number of the celestials
occasionally visits them, in the same manner as a shepherd visits his flocks of
sheep, or an equerry his horses, or a herdsman his lowing cattle, in order that
he may repress the more ferocious, heal the morbid, and assist those that are
in want? You say that no God intervenes in human affairs. To whom, therefore,
shall I pray? To whom shall I make vows? To whom shall I immolate victims? Whom
shall I invoke through the whole of my life, as my helper in misery, as the
favourer of the good, and the adversary of the evil? And lastly (which is a
thing that most frequently occurs), whom shall I adduce as a witness to my
oath? Shall I say, as the Virgilian Ascanius,\footnote{See book \textsc{xi} of
the {\AE}neid.}

\begin{verse}
Now by this head I swear, by which before\\
My father used to swear.
\end{verse}

\noindent But, O Iulus, your father might employ this oath among the Trojans,
who were allied to him by their origin, and also perhaps among the Greeks, who
were known to him in battle; but among the Rutuli, who were recently known by
you, if no one believed in this head, what God would be a surety for you? Would
your right hand and your dart, as they were to the most ferocious Mezentius?
For these only, by which he defended himself, he adjured:

\begin{verse}
To me my right hand and the missile dart,\\
Which now well-poised I hurl, are each a God.\footnote{See book \textsc{x} of
the {\AE}neid.}
\end{verse}

\noindent Take away, I beseech you, such sanguinary Gods; a right hand weary
with slaughter, and a dart rusty with gore. It is not fit that you should
invoke either of these, nor that you should swear by them, since this is an
honour peculiar to the highest of the Gods. For a solemn oath, as Ennius says,
is also called \textit{Jovisjurandum}, as pertaining to Jupiter, by whom alone
it is proper to swear. What, therefore, do you think? Shall I swear by Jupiter,
holding a stone in my hand, after the most ancient manner of the Romans? But if
the opinion of Plato is true, that God never mingles himself with man, a stone
will hear me more easily than Jupiter. This, however, is not true: for Plato
will answer for his opinion by my voice. I do not, says he, assert that the
Gods are separated and alienated from us, so as to think that not even our
prayers reach them; for I do not remove them from an attention to, but only
from a contact with, human affairs.

Moreover, there are certain divine middle powers, situated in this interval of
the air, between the highest ether and earth, which is in the lowest place,
through whom our desires and our deserts pass to the Gods.  These are called by
a Greek name d{\ae}mons, who, being placed between the terrestrial and
celestial inhabitants, transmit prayers from the one, and gifts from the other.
They likewise carry supplications from the one, and auxiliaries from the other,
as certain interpreters and saluters of both. Through these same d{\ae}mons, as
Plato says in the Banquet, all denunciations, the various miracles of
enchanters, and all the species of presages, are directed.  Prefects, from
among the number of these, providentially attend to every thing, according to
the province assigned to each; either by the formation of dreams, or causing
the fissures in entrails, or governing the flights of some birds, and
instructing the songs of others, or by inspiring prophets, or hurling thunder,
or producing the coruscations of lightning in the clouds, or causing other
things to take place, by which we obtain a knowledge of future
events.\footnote{For a copious account of d{\ae}mons, their nature, and
different orders, see the notes on the First Alcibiades, in vol.~\textsc{i} of
my Plato, and also my translation of Iamblichus on the
Mysteries.} And it is requisite to think that all these particulars are
effected by the will, the power, and authority of the celestial Gods, but by
the compliance, operations, and ministrant offices of d{\ae}mons; for it was
through the employment, the operations, and the providential attention of
these, that dreams predicted to Hannibal the loss of one of his eyes; that the
inspection of the viscera previously announced to Flaminius the danger of a
great slaughter; and that auguries granted to Accius Navius the miracle of the
whetstone. It is also through these that forerunning indications of future
empire are imparted to certain persons; as that an eagle covered the head of
Tarquinius Priscus, and that a flame illuminated the head of Servius Tullius.
And lastly, to these are owing all the presages of diviners, the expiations of
the Hetruscans, the enclosure of places struck by lightning, and the verses of
the Sibyls; all which, as I have said, are effected by certain powers that are
media between men and Gods.  For it would not be conformable to the majesty of
the celestial Gods, that any one of them should either devise a dream for
Hannibal, or snatch the victim from Flaminius, or direct the flight of the bird
to Accius Navius, or versify the predictions of the Sibyl, or be willing to
snatch the hat from the head of Tarquin, and immediately restore it, or produce
a splendid flame from the head of Servius, but not such as would burn him. It
is not fit that the supernal Gods should descend to things of this kind. This
is the province of the intermediate Gods, who dwell in the regions of the air,
which border on the earth, and yet are no less conversant with the confines of
the heavens; just as in every part of the world there are animals adapted to
the several parts, the volant living in the air, and the gradient on the earth.
For since there are four most known elements, nature being as it were
quadrifariously separated into large parts, and there are animals appropriate
to earth and fire; since Aristotle asserts, that certain peculiar animals,
furnished with wings, fly in burning furnaces, and pass the whole of their life
in fire,\footnote{This is asserted by Aristotle, in book \textsc{v}
chap.~\textsc{xix} of his History of Animals.} rise into existence with it, and
together with it are extinguished; and, besides this, since, as we have before
said, so many various stars are beheld supernally in ether, \textit{i.~e.}~in
the most clear flagrancy of fire\footnote{It must be observed, however, that
the fire of which ether consists, and also the stars, for the most part, is,
according to Plato, vivific and unburning. See book \textsc{iii} of my
translation of Proclus on the Tim{\ae}us.}---since this is the case, why should
nature alone suffer this fourth element, the air, which is so widely extended,
to be void of every thing, and destitute of [proper] inhabitants?  Are not
animals, however, generated in the air, in the same manner as flame-coloured
animals are generated in fire, such as are unstable in water, and such as are
glebous in earth? For you may most justly say, that his opinion is false, who
attributes birds to the air; since no one of them is elevated above the summit
of mount Olympus, which, though it is said to be the highest of all mountains,
yet the perpendicular altitude of its summit is not equal, according to
geometricians, to ten stadia; but there is an immense mass of air, which
extends as far as to the nearest spiral gyrations of the moon, from which ether
supernally commences. What, therefore, shall we say of such a great abundance
of air, which is expanded from the lowest revolutions of the moon, as far as to
the highest summit of mount Olympus? Will it be destitute of its appropriate
animals, and will this part of nature be without life, and debile? But, if you
diligently observe, birds themselves may, with greater rectitude, be said to be
terrestrial than a\"{e}rial animals; for the whole of their life is always on
the earth; there they procure food, and there they rest; and they only pass
through that portion of the air in flying which is proximate to the earth. But,
when they are weary with the rowing of their wings, the earth is to them as a
port. If, therefore, reason evidently requires that proper animals must also be
admitted to exist in the air, it remains that we should consider what they are,
and what the species is to which they belong.

They are then by no means terrene animals; for these verge downwards by their
gravity. But neither are they of a fiery nature, lest they should be hastily
raised on high by their heat. A certain middle nature, therefore must be
fashioned for us, of a temperature adapted to the middle condition of the
place, so that the disposition of the inhabitants may be conformable to the
quality of the region.  Let us then form in our mind and generate bodies, so
constituted as neither to be so heavy as terrene, nor so light as ethereal
bodies, but after a manner separated from both, or mingled from both, whether
they are removed from, or are modified by, the participation of each. They
will, however, be more easily conceived, if they are admitted to be mingled
from both, than if they are said to be mingled with neither. These bodies of
d{\ae}mons, therefore, will have a little weight, in order that they may not
proceed to supernal natures; and they will also have something of levity, in
order that they may not be precipitated to the realms beneath. And, that I may
not seem to you to devise incredible things, after the manner of the poets, I
will give you, in the first place, an example of this equiponderant mediocrity.
For we see that the clouds coalesce, in a way not much different from this
tenuity of body; and if these were equally as light as those bodies which are
entirely without weight, they would never crown the summit of a lofty mountain
with, as it were, certain bent chains, being depressed beneath its vertex, as
we frequently perceive they do. Moreover, if they were naturally so dense and
ponderous that no admixture, of a more active levity, could elevate them, they
would certainly strike against the earth, by their own effort, no otherwise
than a rude mass of lead and a stone. Now, however, being pendulous and
moveable, they are governed in different directions by the winds in the sea of
air, in the same manner as ships, suffering some little variation by their
proximity and remoteness; for, if they are prolific with the moisture of water,
they are depressed downward, as if delivering a f{\oe}tus into light.  And on
this account clouds that are more moist descend lower, in a black troop, and
with a slower motion; but those that are serene ascend higher, like fleeces of
wool, in a white troop, and with a more rapid flight; or have you not heard
what Lucretius most eloquently sings concerning thunder [in his sixth book]:

\begin{verse}
The azure heavens by thunders dire are shook,\\
Because th' ethereal clouds, ascending high,\\
Dash on each other, driven by adverse winds.
\end{verse}
 
\noindent But if the clouds fly loftily, all of which originate from, and again
flow downward to, the earth, what should you at length think of the bodies of
d{\ae}mons, which are much less dense, and therefore so much more attenuated
than clouds? For they are not conglobed from a feculent nebula and a tumid
darkness, as the clouds are, but they consist of that most pure, liquid, and
serene element of air, and on this account are not easily visible to the human
eye, unless they exhibit an image of themselves by divine command. For no
terrene solidity occupies in them the place of light, so as to resist our
perception, since the energies of our sight, when opposed by opaque solidity,
are necessarily retarded; but the frame of their bodies is rare, splendid, and
attenuated, so that they pass through the rays of the whole of our sight by
their rarity, reverberate them by their splendour, and escape them by their
subtlety. From hence is that Homeric Minerva, who was present in the midst of
the assembly of the Greeks, for the purpose of repressing the anger of
Achilles. If you wait a little, I will enunciate to you, in Latin, the Greek
verse [in which this is mentioned by Homer], or rather let it be now given.
Minerva, therefore, as I have said, by the command of Juno, was present, in
order to restrain the rage of Achilles,

\begin{verse}
Seen by him only, by the rest unseen.\footnote{Iliad, \textsc{i} \textsc{v}
198.}
\end{verse}

\noindent From hence also is that Juturna in Virgil, who had intercourse
with many thousands of men, for the purpose of
giving assistance to her brother,

\begin{verse}
With soldiers mingled, but by none perceived.\footnote{{\AE}neid,
lib.~\textsc{xii}.}
\end{verse}

\noindent Entirely accomplishing that which the soldier of Plautus\footnote{In
the original, \textit{prorsus quod Plautinus miles,} \&c.; but Lipsius and the
Delphin editor, for \textit{prorsus quod}, read \textit{prorsus quam}. It does
not, however, appear to me that any emendation is requisite; or that
their alteration is an amendment.} boasted of having effected by his shield,

\begin{verse}
Which dazzled by its light the vision of his foes.
\end{verse}

\noindent And that I may not prolixly discuss what remains, poets, from this
multitude of d{\ae}mons, are accustomed, in a way by no means remote from
truth, to feign the Gods to be haters and lovers of certain men; and to give
prosperity and elevation to these; but on the contrary, to be averse from and
afflict those. Hence, they are influenced by pity, are indignant, solicitous,
and delighted, and suffer all the mutations of the human soul; and are agitated
by all the ebullitions of human thought, with a similar motion of the heart,
and tempest of the mind.\footnote{According to the ancient theology, the lowest
orders of those powers that are the perpetual attendants of the Gods, preserve
the characteristics of their leaders, though in a partial and multiplied
manner, and are called by their names. Hence, the passions of the subjects of
their government are, in fables, proximately referred to these. See the
Introduction to the second and third books of the Republic, in
vol.~\textsc{i} of my Plato.} All which storms and tempests are far exiled
from the tranquillity of the celestial Gods. For all the celestials always
enjoy the same state of mind, with an eternal equability: which in them is
never driven from its proper seat, either towards pleasure or pain. Nor are
they removed by any thing, from their own perpetual energy, to any sudden
habitude; neither by any foreign force, because nothing is more powerful than
deity; nor of their own accord, because nothing is more perfect than
themselves.

Moreover, how can he appear to have been perfect, who migrates from a former
condition of being to another which is better? Especially since no one
spontaneously embraces any thing new, except he despises what he possessed
before. For that altered mode of acting cannot take place, without the
debilitation of the preceding modes. Hence, it is requisite that God should
neither be employed in giving temporal assistance, or be impelled to love; and,
therefore, he is neither influenced by indignation nor by pity, nor is
disquieted by any anxiety, nor elated by any hilarity; but is liberated from
all the passions of the mind, so that he never either grieves or rejoices, nor
wills, nor is averse to any thing subitaneous.\footnote{\-``Divinity,'' says
Sallust (in chap.~\textsc{xiv} of his treatise On the Gods and the World)
``neither rejoices; for that which rejoices is also influenced by sorrow: nor
is angry; for anger is a passion: nor is appeased by gifts; for then he would
be influenced by delight. Nor is it lawful that a divine nature should be well
or ill affected from human concerns: for the divinities are perpetually good
and profitable, but are never noxious, and ever subsist in the same uniform
mode of being. But we, when we are virtuous, are conjoined to the Gods through
similitude: but when vicious, we are separated from them through dissimilitude.
And while we live according to virtue, we partake of the Gods, but when we
become evil, we cause them to become our enemies; not that they are angry, but
because guilt prevents us from receiving the illuminations of the Gods, and
subjects us to the power of avenging d{\ae}mons.''} But all these, and other
things of the like kind, properly accord with the middle nature of
d{\ae}mons.\footnote{This, however, applies only to the lowest order of
d{\ae}mons.} For as they are media between us and the Gods, in the place of
their habitation, so likewise is the nature of their mind; having immortality
in common with the Gods, and passion in common with the beings subordinate to
themselves.  For they are capable, in the same manner as we are, of suffering
all the mitigations or incitements of souls; so as to be stimulated by anger,
made to incline by pity, allured by gifts, appeased by prayers, exasperated by
contumely, soothed by honours, and changed by all other things, in the same way
that we are. Indeed, that I may comprehend the nature of them by a definition,
d{\ae}mons are in their genus animals, in their species rational, in mind
passive, in body a\"{e}rial, and in time perpetual. Of these five
characteristics which I have mentioned, the three first are the same as those
which we possess, the fourth is peculiar to them, and the last is common to
them with the immortal Gods, from whom they differ in being obnoxious to
passion. Hence, as I think, d{\ae}mons are not absurdly denominated passive,
because they are subject to the same perturbations that we are. On which
account, also, it is requisite to believe in the different observances of
religions, and the various supplications employed in sacred rites. There are,
likewise, some among this number of Gods who rejoice in victims, or ceremonies
or rites, which are nocturnal or diurnal, obvious or occult, more joyful or
more sad. Thus the Egyptian deities are almost all of them delighted with
lamentations, the Grecian for the most part with choirs, but the Barbarian with
the sound produced by cymbals, drums, and pipes. In like manner, other things
pertaining to sacred rites differ by a great variety, according to different
regions; as, for instance, the crowds of sacred processions, the arcana of
mysteries, the offices of priests, and the compliances of those that sacrifice;
and farther still, the effigies of the Gods, and the spoils dedicated to them,
the religions and situations of temples, and the variety of blood and colour in
victims. All which particulars are rightly accomplished, and after the
accustomed manner, if they are effected appropriately to the regions to which
they belong. Thus from dreams, predictions, and oracles, we have for the most
part found that the divinities have been indignant, if any thing in their
sacred rites has been neglected through indolence or pride; of which kind of
things I have an abundance of examples. They are, however, so celebrated, and
so generally known, that no one would attempt to relate them, without omitting
much more than he narrated.  On this account, I shall desist at present from
speaking about these particulars, which if they are not believed by all men,
yet certainly a promiscuous knowledge of them is universal. It will be better,
therefore, to discuss this in the Latin tongue, viz.~that various species of
d{\ae}mons are enumerated by philosophers, in order that you may more clearly
and fully understand the nature of the presage of Socrates, and of his familiar
d{\ae}mon.

The human soul, therefore, even when situated in the present body, is called,
according to a certain signification, a d{\ae}mon.

\begin{verse}
O say, Euryalus, do Gods inspire\\
In minds this ardour, or does fierce desire\\
Rule as a God in its possessor's breast?\footnote{These verses are taken from
book \textsc{ix} of the {\AE}neid.}
\end{verse}

\noindent For if this be the case, the upright desire of the soul is a good
d{\ae}mon. Hence, some persons think, as we have before observed, that the
blessed are called \textit{eud{\ae}mones}, the \textit{d{\ae}mon} of whom is
good, \textit{i.~e.}~whose mind is perfect in virtue. You may call this
d{\ae}mon in our tongue, according to my interpretation, a \textit{Genius}, I
know not whether rightly, but certainly at my peril; because this God [or
d{\ae}mon], who is the mind of every one,\footnote{\-``The soul,'' says Proclus
in his Commentary on the First Alcibiades, ``that, through its
similitude to the d{\ae}moniacal genus, produces energies more wonderful then
those which belong to human nature, and which suspends the whole of its life
from d{\ae}mons, is a d{\ae}mon according to habitude (\textit{i.~e.}~proximity
or alliance). But an \textit{essential} d{\ae}mon is neither called a d{\ae}mon
through habitude to secondary natures, nor through an assimilation to something
different from himself; but is allotted this peculiarity from himself, and is
defined by a certain summit, or flower of essence, by appropriate powers, and
by different modes of energies.''} though it is immortal, nevertheless, is
after a certain manner generated with man; so that those prayers by which we
implore the \textit{Genius}, and which we employ when we embrace the
\textit{knees} [genua] of those whom we supplicate, appear to me to testify our
connexion and union; since they comprehend in two words the body and mind;
through the communion and copulation of which we exist. There is also another
species of d{\ae}mons, according to a second signification, and this is a human
soul, which, after its departure from the present life, does not enter into
another body. I find that souls of this kind are called in the ancient Latin
tongue \textit{Lemures}. Of these \textit{Lemures}, therefore, he who, being
allotted the guardianship of his posterity, dwells in a house with an appeased
and tranquil power, is called a familiar [or domestic] \textit{Lar}. But those
are for the most part called \textit{Larv{\ae}}, who, having no proper
habitation, are punished with an uncertain wandering, as with a certain exile,
on account of the evil deeds of their life, and become a vain terror to good,
and are noxious to bad men. And when it is uncertain what the allotted
condition is of any one of these, they call the God by the name of
\textit{Manes}; the name of God being added for the sake of honour. For they
alone call those Gods, who being of the same number of \textit{Lemures}, and
having governed the course of their life justly and prudently, have afterwards
been celebrated by men as divinities, and are every where worshipped in
temples, and honoured by religious rites; such for instance as Amphiaraus in
B{\oe}otia, Mopsus in Africa, Osiris in Egypt, and some other in other nations,
but Esculapius every where. All this distribution, however, was of those
d{\ae}mons, who once existed in a human body.\footnote{Those human souls that
descend into the regions of mortality with impassivity and purity, were called
by the ancients \textit{heroes}, on account of their great proximity and
alliance to such as are \textit{essentially} heroes, and are the
\textit{perpetual} attendants of the Gods. These heroes called themselves by
the names of the divinities from whom they descended, and by whose
peculiarities their energies were characterised. When, however, through the
corruption of the heathen religion, these heroes were no longer reverenced in
an \textit{appropriate} manner, but the worship of the Gods was transferred to
them, the proper distinction between their essence and that of the divinities
was confounded; and from this that most dire opinion that the Gods of the
ancients were nothing more than men who once existed on the earth, derived its
origin. See more on this subject in the Introduction to my translation of
Proclus on the Theology of Plato.}

But there is another species of d{\ae}mons, more sublime and venerable, not
less numerous, but far superior in dignity, who, being always liberated from
the bonds and conjunction of the body, preside over certain powers. In the
number of these are Sleep and Love, who possess powers of a different nature;
Love, of exciting to wakefulness, but Sleep of lulling to rest. From this more
sublime order of d{\ae}mons, Plato asserts that a peculiar d{\ae}mon is
allotted to every man, who is a witness and a guardian\footnote{According to
Plato, our guardian d{\ae}mons belong to that order of d{\ae}mons, which is
arranged under the Gods that preside over the ascent and descent of souls.
Olympiodorus in his Commentary on the Ph{\ae}do of Plato observes,
``that there is one d{\ae}mon who leads the soul to its judges from the present
life; another who is ministrant to the judges, giving completion, as it were,
to the sentence which is passed; and a third, who is again allotted the
guardianship of life.''} of his conduct in life, who, without being visible to
any one, is always present, and who is an arbitrator not only of his deeds, but
also of his thoughts.  But when, life being finished, the soul returns [to the
judges of its conduct], then the d{\ae}mon who presided over it immediately
seizes, and leads it as his charge to judgement and is there present with it
while it pleads its cause. Hence, this d{\ae}mon reprehends it, if it has acted
on any false pretence; solemnly confirms what it says, if it asserts any thing
that is true; and conformably to its testimony passes sentence. All you,
therefore, who hear this divine opinion of Plato, as interpreted by me, so form
your minds to whatever you may do, or to whatever may be the subject of your
meditation, that you may know there is nothing concealed, from those guardians
either within the mind, or external to it; but that the d{\ae}mon who presides
over you inquisitively participates of all that concerns you, sees all things,
understands all things, and \textit{in the place of conscience dwells in the
most profound recesses of the mind}.\footnote{In the original, \textit{in ipsis
peritissimis mentibus vice conscienti{\ae} diversetur}. This is a most
remarkable passage, since it perfectly accords with what Olympiodorus says of
our allotted d{\ae}mon, in his Scholia on the First Alcibiades of
Plato, and contains a dogma concerning this d{\ae}mon, which is only to be
found explicitly maintained in these Scholia. But the words of
Olympiodorus are as follow: ``This is what is said by the interpreters [of
Plato] concerning d{\ae}mons, and those which are allotted to us. We, however,
shall endeavour to discuss these particulars in such a way as to reconcile them
with what is at present said by Plato; for Socrates was condemned to take
poison, in consequence of introducing to young men novel d{\ae}moniacal powers,
and for thinking those to be Gods which were not admitted to be so by the city.
It must be said, therefore, that the allotted d{\ae}mon is \textit{conscience},
which is \textit{the supreme flower of the soul}, is guiltless in us, is an
inflexible judge, and a witness to Minos and Rhadamanthus of the transactions
of the present life. This also becomes the cause to us of our salvation, as
always remaining in us without guilt, and not assenting to the errors of the
soul, but disdaining them, and converting the soul to what is proper. You will
not err, therefore, in calling the allotted d{\ae}mon conscience. But it is
requisite to know that, of conscience, one kind pertains to our gnostic powers,
and which is denominated \textit{conscience} (\textit{co-intelligence})
homonymously with the genus.'' In this passage, as Creuzer, the editor of these
Scholia, well observes, something is wanting at the end; and a part of
what is deficient, I conceive to be the words, \textit{but another kind to our
vital powers}; for the great division of the powers of the soul is into the
gnostic and vital.

The singularity in this dogma of Olympiodorus, respecting our \textit{allotted
d{\ae}mon}, is, that in making it to be the same with \textit{conscience}, if
conscience is admitted to be a part of the soul, the dogma of Plotinus must
also be admitted, ``that the whole of our soul does not enter into the body,
but that something belonging to it always abides in the intelligible world.''
But this dogma appears to have been opposed by all the Platonists posterior to
Plotinus; and Proclus has confuted it in the last proposition of his
Elements Of Theology; for he there demonstrates, ``that every partial
soul, in descending into generation [or the sublunary realms], descends wholly;
nor does one part of it remain on high, and another part descend.'' But his
demonstration of this is as follows: ``For if something pertaining to the soul
remained on high, in the intelligible world, it will always perceive
intellectually, without transition, or transitively. But if without transition,
it will be intellect, and not a part of the soul, and this partial soul will
proximately participate of intellect [\textit{i.~e.}~not through the medium of
d{\ae}moniacal and divine souls]. This, however, is impossible. But if it
perceives intellectually with transition, from that which always, and from that
which sometimes, energizes intellectually, one essence will be formed. This,
however, also is impossible; for these always differ, as has been demonstrated.
To which may be added, the absurdity resulting from supposing that the summit
of the soul is always perfect, and yet does not rule over the other powers, and
cause them to be perfect. Every partial soul, therefore, wholly descends.''
Hence, if Olympiodorus was likewise hostile to this dogma of Plotinus, it must
follow, according to him, that \textit{conscience} is not a part of the soul,
but something superior to it, and dwelling in its summit. Perhaps, therefore,
Olympiodorus on this account calls the allotted d{\ae}mon, \textit{the supreme
flower of the soul}. For the summit, or \textit{the one} of the soul, is
frequently called by Platonic writers, \textit{the flower}, but not \textit{the
supreme flower}; so that the addition of \textit{supreme} will distinguish the
presiding d{\ae}mon from the summit of the soul. The place in which this dogma
of Plotinus is to be found, is at the end of his treatise \textit{On the
Descent of the Soul}.

I only add, that the celebrated poet Menander appears to have been the source
of this dogma, that conscience is our allotted d{\ae}mon; for one of the
Excerpt{\ae} from his fragments is, ``To ev'ry mortal conscience is a
God.''} For he of whom I speak is a perfect guardian, a singular prefect, a
domestic speculator, a proper curator, an intimate inspector, an assiduous
observer, an inseparable arbiter, a reprobater of what is evil, an approver of
what is good; and if he is legitimately attended to, sedulously known, and
religiously worshipped, in the way in which he was reverenced by Socrates with
justice and innocence, will be a predictor in things uncertain, a premonitor in
things dubious, a defender in things dangerous, and an assistant in want.  He
will also be able, by dreams, by tokens, and perhaps also manifestly, when the
occasion demands it, to avert from you evil, increase your good, raise your
depressed, support your falling, illuminate your obscure, govern your
prosperous, and correct your adverse circumstances. It is not therefore
wonderful, if Socrates, who was a man exceedingly perfect, and also wise by the
testimony of Apollo, should know and worship this his God; and that hence, this
his keeper, and nearly, as I may say, his equal, his associate and domestic,
should repel from him every thing which ought to be repelled, foresee what
ought to be noticed, and pre-admonish him of what ought to be foreknown by him,
in those cases in which, human wisdom being no longer of any use, he was in
want, not of counsel, but of presage; in order that when he was vacillating
through doubt, he might be rendered firm through divination.  For there are
many things, concerning the development of which even wise men betake
themselves to diviners and oracles. Or do you not more clearly perceive in
Homer, as in a certain large mirror, these two offices of divination and wisdom
distributed apart from each other?  For when those two pillars of the whole
army were discordant, Agamemnon powerful in empire, and Achilles invincible in
battle, a man praised for his eloquence and renowned for his skill was wanting,
who might humble the pride of the son of Atreus, and repress the rage of
Pelides, and who might engage their attention by his authority, admonish them
by examples, and allure them by his words.  Who, therefore, at such a time
undertook to speak? The Pylian orator, who was courteous in his eloquence,
cautious through experience, and venerable by his age; who was known by all to
have a body debilitated by time, but a mind flourishing in wisdom, and words
abounding with sweetness.

In like manner, when in dubious and adverse circumstances, spies are to be
chosen, who may penetrate into the camps of the enemy at midnight, are not
Ulysses and Diomed selected for this purpose, as counsel and aid, mind and
hand, spirit and sword? But when the Greeks, ceasing from hostilities through
weariness, and being detained in Aulis, applied themselves to explore the
difficulty of the war, the facility of the journey, the tranquillity of the
sea, and the clemency of the winds, through the indications of fibres, the food
administered by birds, and the paths of serpents;\footnote{Apuleius here
alludes to the serpent which at Aulis, in the presence of the Greeks, ascended
into a plane tree, and devoured eight little sparrows together with their
mother. Whence Calchas prophesied that the Trojan war would last nine years,
but that the city would be captured in the tenth year. See the Iliad,
lib.~\textsc{ii} \textsc{v} 300.} then those two supreme summits of Grecian
wisdom, Ulysses and Nestor, were mutually silent; but Calchas, who was far more
skilful in divination, as soon as he had surveyed the birds, and the altars,
and the tree, immediately by his divination appeased the tempests, brought the
fleet into the sea, and predicted the ten years' war. No otherwise also in the
Trojan army, when the affairs require divination, that wise senate is silent,
nor either Hicetaon, or Lampus, or Clytius, dares to assert any thing; but all
of them listen in silence, either to the odious auguries of Helenus, or to the
never-to-be-believed predictions of Cassandra. After the same manner Socrates,
if at any time consultation foreign from the province of wisdom was requisite,
was then governed by the prophetic power of his d{\ae}mon. But he was
sedulously obedient to his admonitions, and on that account was far more
acceptable to his God.

The reason, however, has been after a manner already assigned, why the
d{\ae}mon of Socrates was nearly accustomed to prohibit him from what he was
going to undertake, but never exhorted him to the performance of any deed. For
Socrates, as being a man of himself exceedingly perfect, and prompt to the
performance of all the duties pertaining to him, never was in want of any
exhorter; but sometimes required a prohibiter, if danger happened to be latent
in any of his undertakings; in order that, being admonished, he might be
cautious, and omit for the present his attempt, which he might either more
safely resume afterwards, or enter upon in some other way. In things of this
kind, he said, ``That he heard a certain voice which originated from
divinity.'' For thus it is narrated by Plato; lest any one should think that
Socrates assumed omens from the conversation of men in common. For once also,
when he was with Ph{\ae}drus, beyond the precinct of the town, under the
covering of a certain umbrageous tree, and without any witnesses, he perceived
that sign which announced to him that he should not pass over the small current
of the river Ilissus, till he had appeased Love, who was indignant at his
reprehension of him, by a recantation.\footnote{See my translation of the
Ph{\ae}drus of Plato.} To which may be added, that, if he had observed
omens, he would sometimes also have received some exhortations from them, as we
see frequently happens to many of those, who, through a too superstitious
observance of omens, are not directed by their own mind, but by the words of
others; and in wandering through the streets, gather counsel from what is said
by passengers, and, as I may say, do not think with the understanding, but with
the ears.

Nevertheless, in whatever manner these things may take place, it is certain
that those who hear the words of diviners, frequently receive a voice through
their ears, concerning the meaning of which they are not at all dubious; and
which they know proceeds from a human mouth. But Socrates did not simply say
that he heard a voice, but a \textit{certain voice}, divinely transmitted to
him.  By which addition, you must understand, that neither a usual nor a human
voice is signified; for if it had been a thing of this kind he would not have
said \textit{a certain voice}, but rather either merely \textit{a voice}, or
\textit{the voice of some one}, as the harlot in Terence says,

\begin{verse}
I seemed just now to hear a soldier's voice.\footnote{This verse is from the
Eunuch of Terence.}
\end{verse}

\noindent But he who says that he hears \textit{a certain voice}, is either
ignorant from whence that voice originated, or is somewhat dubious concerning
it, or shows that it contained something unusual and arcane, as Socrates did in
that voice, which he said was transmitted to him opportunely and divinely. And,
indeed, I think that he perceived the indication of his d{\ae}mon, not only
with his ears, but also with his eyes; for he frequently asserted that not a
voice, but a divine sign, was exhibited to him.  That sign might also have been
the resemblance of his d{\ae}mon, which Socrates alone beheld, in the same
manner as the Homeric Achilles beheld Minerva. I am of opinion, that the
greatest part of you will with difficulty believe what I have now said, and
will wonder in the extreme at the form of the d{\ae}mon which was seen by
Socrates alone.  But Aristotle, whose authority is, I think, sufficient,
asserts, that it was usual with the Pythagoreans very much to admire, if any
one denied that he had ever seen a d{\ae}mon. If, therefore, the power of
beholding a divine resemblance may be possessed by any one, why might it not,
in an eminent degree, befall Socrates, whom the dignity of wisdom rendered
similar to the most excellent divinity? For nothing is more similar and more
acceptable to God, than a man intellectually good in perfection, who as much
excels other men as he himself is surpassed by the immortal Gods. Should not we
also rather elevate ourselves by the example and remembrance of Socrates?  And
should we not deliver ourselves to the felicitous study of a similar
philosophy, and pay attention to similar divinities? From which study we are
drawn away, though I know not for what reason. Nor is there any thing which
excites in me so much wonder, as that all men should desire to live most
happily, and should know that they cannot so live in any other way than by
cultivating the mind, and yet leave the mind uncultivated. If, however, any one
wishes to see acutely, it is requisite that he should pay attention to his eyes
through which he sees; if you desire to run with celerity, attention must be
paid to the feet, by which you run; and thus also, if you wish to be a powerful
pugilist, your arms must be strengthened, through which you engage in this
exercise. In a similar manner, in all the other members, attention to each must
be paid in the place of study. And, as all men may easily see that this is
true, I cannot sufficiently think with myself, and admire, in such a way as the
thing deserves to be admired, why they do not also cultivate their mind by
[right] reason: for this art of living [\textit{i.~e.}~according to right
reason] is equally necessary to all men; but this is not the case with the art
of painting, nor with the art of singing, which any worthy man may despise,
without any mental vituperation, without turpitude, and without a blemish [in
his reputation]. I know not how to play on the flute like Ismenias, yet I feel
no shame that I am not a piper: I know not how to paint in colours like
Apelles, nor to carve like Lysippus, but I am not ashamed that I am neither a
painter nor a statuary. But say, my friend, I know not how to live with
rectitude, as Socrates, as Plato, as Pythagoras lived, and yet I feel no shame
that I know not how to live rightly. You will never dare to say this.

It is, however, especially admirable in the multitude, that they should neglect
to learn those things of which they are by no means desirous of appearing to be
ignorant, and reject, at one and the same time, both the discipline and
ignorance of the same art. Hence, if you examine their daily conduct, you will
find that they are prodigally profuse in other things, but bestow nothing on
themselves, I mean, in a proper attention to their d{\ae}mon, which proper
attention is nothing else than the sacrament of philosophy. They build, indeed,
magnificent villas, most sumptuously adorn their houses, and procure numerous
servants; but in all these, and amidst such great affluence, there is nothing
to be ashamed of but the master of this abundance: and deservedly; for they
have an accumulation of things to which they are devoted, but they themselves
wander about them, unpolished, uncultivated, and ignorant. Hence you will find
the forms of those buildings, in which they idly waste their patrimony, to be
most pleasing to the view, most exquisitely built, and most elegantly adorned.
You will also see villas raised, which emulate cities, houses decorated like
temples, most numerous servants, and those with curled locks, costly furniture,
every thing exhibiting affluence, opulence, every where, and every thing
ornamented, except the master himself, who alone, like Tantalus, being needy
and poor in the midst of his riches, does not indeed pant after that fugitive
river, nor endeavour to quench his thirst with fallacious water, but hungers
and thirsts after true beatitude, \textit{i.~e.}~after a genuine,\footnote{In
the original, \textit{secund{\ae} vit{\ae}}; but I read, with the Roman
edition, \textit{sincer{\ae} vit{\ae}}.} prudent, and most fortunate life.For
he does not perceive that it is usual to consider rich men in the same way that
we do horses when we buy them; for in purchasing these we do not look to the
trappings, nor the decorations of the belt, nor do we contemplate the riches of
the most ornamented neck, and examine whether variegated chains, consisting of
silver, gold, or gems, depend from it; whether ornaments full of art surround
the head and neck; and whether the bridles are carved, the saddles are painted,
and the girths are gilt; but, all these spoils being removed, we survey the
naked horse itself, and alone direct our attention to his body and his soul, in
order that we may be able to ascertain whether his form is good, and whether he
is likely to be vigorous in the race, and strong for carriage. And in the first
place we consider whether there is in his body,

\begin{verse}
A head that's slender, and a belly small,\\
A back obese, and animated breast\\
In brawny flesh luxuriant.\footnote{These verses are taken from book
\textsc{iii} of the Georgics of Virgil.}
\end{verse}

\noindent And, besides this, whether a twofold spine passes through his loins;
for I wish that he may not only carry me swiftly, but also gently.

In a similar manner therefore, in surveying men, do not estimate those foreign
particulars, but intimately consider the man himself, and behold him poor, as
was my Socrates. But I call those things foreign which parents have procreated,
and which Fortune has bestowed, none of which do I mingle with the praises of
my Socrates; no nobility, no pedigree, no long series of ancestors, no envied
riches; for all these, as I say, are foreign. When you say, O son of
Prothanius, the glory of him who was this son is this, that he was not a
disgrace to his grandson, in like manner you may enumerate every thing of a
foreign nature. Is he of noble birth? You praise his parents. Is he rich? I do
not trust in Fortune; nor do I rank these, more [than their contraries], among
things really good. Is he strong? He will be debilitated by disease. Is he
swift in the race? He will arrive at old age. Is he beautiful? Wait a little,
and he will not be so. But is he instructed, and very learned in excellent
disciplines, and also wise, and skilled in the knowledge of good, as much as it
is possible for man to be? Now at length you praise the man himself; for this
is neither an hereditary possession from his father, nor depends on Fortune,
nor on the annual suffrages of the people, nor is it decaying through body, nor
mutable by age. All these my Socrates possessed, and therefore despised the
possession of other things. Why therefore do not you apply yourself to the
study of wisdom? Or at least you should earnestly endeavour that you may hear
nothing of a foreign nature in your praise; but that he who wishes to ennoble
you, may praise you in the same manner as Accius praises Ulysses, in his
\textit{Philoctetes}, in the beginning of that tragedy:

\begin{verse}
Fam'd hero, in a little island born,\\
Of celebrated name and powerful mind,\\
Once to the Grecian ships war's leading cause,\\
And to the Dardan race th' avenger dire,\\
Son of Laertes.
\end{verse}

\noindent He mentions his father in the last place. Moreover, you have heard
all the praises of that man; but Laertes, Anticlea, and Acrisius, vindicate
to themselves nothing from thence; for the whole of this praise, as you see, is
a possession peculiarly pertaining to Ulysses. Nor does Homer teach you any
thing else in the same Ulysses, by always giving him Wisdom as a companion,
whom he poetically calls Minerva. Hence, attended by this, he encounters all
horrible dangers, and vanquishes all adverse circumstances. For, assisted by
her, he entered the cavern of the Cyclops, but escaped from it; saw the oxen of
the Sun, but abstained from them; and descended to the realms beneath, but
emerged from them. With the same Wisdom also for his companion, he passed by
Scylla, and was not seized by her; was enclosed by Charybdis, yet was not
retained by it; drank the cup of Circe, and was not transformed; came to the
Lotophagi, yet did not remain with them; and heard the Sirens, yet did not
approach to them.\footnote{The concluding part of this treatise on the God of
Socrates has a great resemblance to the conclusion of the dissertation of
Maximus Tyrius, entitled, Whether there is a Sect in Philosophy,
according to Homer? and which is as follows: ``And with respect to Ulysses
himself, do you not see how virtue, and the confidence which he acquires
through her aid, preserve him, while he opposes art to all-various calamities?
This is the \textit{moly} in the island of Circe, this is the fillet in the
sea, this delivered him from the hands of Polyphemus, this led him up from
Hades, this constructed for him a raft, this persuaded Alcinous, this enabled
him to endure the blows of the suitors, the wrestling with Irus, and the
insolences of Melanthius. This liberated his palace, this avenged the injuries
of his wife, this made the man a descendant of Jupiter, like the Gods, and such
a one as the happy man is according to Plato.'' See my translation of Maximus
Tyrius, vol.~\textsc{i}.}

\end{document}
