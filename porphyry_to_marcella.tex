\documentclass[12pt]{article}
\usepackage{ebgaramond}
\usepackage{microtype}

\title{To Marcella}
\author{Porphyry \and Alice Zimmern (tr.)}
\date{}

\begin{document}
\maketitle

\paragraph{1} I chose thee as my wife, Marcella, though thou wert the mother of five daughters and two sons, some of whom were still little children, and the others approaching a marriageable age; nor was I deterred by the multitude of things which would be needful for their maintenance. And it was not for the sake of having children that I wedded thee, deeming that the lovers of true wisdom were my children, and that thy children too would be mine, if ever these should attain to right philosophy, when educated by us. Nor yet was it because a superfluity of riches had fallen either to thy lot or mine. For such necessaries as are ours must suffice us who are poor. Neither did I expect that thou wouldst afford me any ease through thy ministrations as I advanced in years, for thy frame is delicate, and more in need of care from others than fitted to succour or watch over them. Nor yet did I desire other housewifely care from thee, nor sought I after honour and praise from those who would not willingly have undertaken such a burden for the mere sake of doing good. Nay, it was far otherwise, for through the folly of thy fellow-citizens, and their envy towards thee and thine, I encountered much ill-speaking, and contrary to all expectation I fell into danger of death at their hands on your behalf.

\paragraph{2} For none of these causes did I choose another to be partner of my life, but there was a twofold and reasonable cause that determined me. One part was that I deemed I should thus propitiate the gods of generation; just as Sokrates in his prison chose to compose popular music, for the sake of safety in his departure from life, instead of pursuing his customary labours in philosophy, so did I strive to propitiate the divinities who preside over this tragi-comedy of ours, and shrank not from chanting in all willingness the marriage hymn, though I took as my lot thy numerous children, and thy straitened circumstances, and the malice of evil-speakers. Nor were there lacking any of those passions usually connected with a play---jealousy, hatred, laughter, quarrelling and anger; this alone excepted, that it was not with a view to ourselves but for the sake of others that we enacted this spectacle in honour of the gods.

\paragraph{3} Another worthier reason, in nowise resembling that commonplace one, was that I admired thee because thy disposition was suited to true philosophy; and when thou wast bereaved of thy husband, a man dear to me, I deemed it not fitting to leave thee without a helpmeet and wise protector suited to thy character. Wherefore I drove away all who were minded to use insult under false pretence, and I endured foolish contumely, and bore in patience with the plots laid against me, and strove, as far as in my power lay, to deliver thee from all who tried to lord it over thee. I recalled thee also to thy proper mode of life, and gave thee a share in philosophy, pointing out to thee a doctrine that should guide thy life. And who could be a more faithful witness to me than thyself, for I should deem it shame to equivocate to thee, or conceal aught of mine from thee, or to withhold from thee (who honourest truth above all things, and therefore didst deem our marriage a gift from Heaven) a truthful relation from beginning to end of all that I have done with respect to and during our union.

\paragraph{4} Had my business permitted me to remain longer in your country, it would have been possible for thee to still thy thirst with fresh and plentiful draughts from fountains close at hand, so that, not contenting thyself with as much of this gift as would be requisite for ends of utility, thou couldst rejoice in easily supplying thyself at thy leisure with plentiful refreshment. But now the affairs of the Greeks requiring me, and the gods too urging me on, it was impossible for thee, though willing, to answer the summons, with so large a number of daughters attending thee. And I held it to be both foolish and wicked to cast them thus without thee among ill-disposed men. Now that I am compelled to delay here, though I cherish the hope of a speedy return, I would exhort thee to keep firm hold of the gifts thou didst receive in those ten months during which thou didst live with me, and not to cast away that thou already hast from desire and longing for more. As for me, I am making what haste I can to rejoin thee.

\paragraph{5} But, as the future is uncertain for travellers, I must, while sending thee consolation, also lay commands upon thee. And I would send a message more suitable for thee than Odysseus' to Penelope to take care of thyself and thy house,

\begin{quote}
``And keep all things in safety,''
\end{quote}

\noindent left behind as thou art, not unlike Philoktetes in the tragedy, suffering from his sore, though his was caused by a baleful serpent, thine by the knowledge how and from what high estate the soul has fallen at birth. Albeit the gods have not forsaken us, as the sons of Atreus forsook him, but they have become our helpers and have been mindful of us. Therefore seeing thou art hard beset in a contest, attended with much wrestling and labour, I earnestly beg thee to keep firm hold upon philosophy, the only sure refuge, and not to yield more than is fitting to the perplexities caused by my absence. Do not from desire for my instruction cast away what thou hast already received, and do not faint before the multitude of other cares that encompass thee, abandoning thyself to the rushing stream of outward things. Rather bear in mind that it is not by ease that men attain the possession of the true good; and practise thyself for the life thou expectest to lead by help of those very troubles which are the only opponents to thy fortitude that are able to disturb and constrain thee. As for plots laid against us, it is easy for those to despise them who are accustomed to disregard all that does not lie in our own power, and who deem that injustice rather recoils upon the doer than injures those who believe that the worst injury inflicted on them can cause them but little loss.

\paragraph{6} Now thou mayest console thyself for the absence of him who sustains thy soul, and is to thee father, husband, teacher, and kindred, yea, if thou wilt, even fatherland, though this seems to offer a reasonable ground for unhappiness, by placing before thee as arbiter not feeling but reason. In the first place consider that, as I have said before, it is impossible that those who desire to be mindful of their return, should accomplish their journey home from this terrestrial exile pleasantly and easily, as through some smooth plain. For no two things can be more entirely opposed to one another than a life of pleasure and ease, and the ascent to the gods. As the summits of mountains cannot be reached without danger and toil, so it is not possible to emerge from the inmost depths of the body through pleasure and ease which drag men down to the body. For 'tis by anxious thought that we reach the road, and by recollection of our fall. Even if we encounter difficulties in our way, hardship is natural to the ascent, for it is given to the gods alone to lead an easy life. But ease is most dangerous for souls which have fallen to this earthly life, making us forgetful in the pursuit of alien things, and bringing on a state of deep slumber, it we fall asleep beguiled by alluring visions.

\paragraph{7} Now there are some chains that are of very heavy gold, but, because of their beauty, they persuaded women who in their folly do not perceive the weight, that they contribute to ornament, and thus got them to bear fetters easily. But other fetters which are of iron compelled them to a knowledge of their sins, and by pain forced them to repent and seek release from the weight; while escape from the golden imprisonment, through the delight felt in it, often causes grievous woe. Whence it has seemed to men of wisdom that labours conduce to virtue more than do pleasures. And to toil is better for man, aye and for woman too, than to let the soul be puffed up and enervated by pleasure. For labour must lead the way to every fair possession, and he must toil who is eager to attain virtue. Thou knowest that Herakles and the Dioskuri, and Asklepios and all other children of the gods, through toil and steadfastness accomplished the blessed journey to heaven. For it is not those who live a life of pleasure that make the ascent to the gods, but rather those who have nobly learnt to endure the greatest misfortunes.

\paragraph{8} I know full well that there could be no greater contest than that which now lies before thee, since thou thinkest that in me thou wilt lose the path of safety and the guide therein. Yet thy circumstances are not altogether unendurable, if thou cast from thee the unreasoning distress of mind which springs from the feelings, and deem it no trivial matter to remember those words by which thou wert with divine rites initiated into true philosophy, approving by thy deeds the fidelity with which they have been apprehended. For it is a man's actions that naturally afford demonstrations of his opinions, and whoever holds a belief must live in accordance with it, in order that he may himself be a faithful witness to his disciples of his words. What was it then that we learned from those men who possess the clearest knowledge to be found among mortals? Was it not this---that I am in reality not this person who can be touched or perceived by any of the senses, but that which is farthest removed from the body, the colourless and formless essence which can by no means be touched by the hands, but is grasped by the mind alone? And it is not from outward things that we receive those principles which are implanted in us. We receive only the keynote as in a chorus, which recalls to our remembrance those commands we received from the god who gave them us ere we set forth on our wanderings.

\paragraph{9} Moreover is not every emotion of the soul most hostile to its safety? And is not want of education the mother of all the passions? Now education does not consist in the absorption of a large amount of knowledge, but in casting off the affections of the soul. The passions are the beginning of diseases. And vice is the disease of the soul; and every vice is disgraceful. And the disgraceful is opposed to the good. Now since the divine nature is good, it is impossible for it to consort with vice, since Plato says it is unlawful for the impure to approach the pure. Wherefore even now we need to purge away all our passions, and the sins that spring therefrom. Was it not this thou didst so much approve, reading as it were divine characters within thee, disclosed by my words? Is it not then absurd, though thou art persuaded that thou hast in thee the saving and the saved, the losing and the lost, wealth and poverty, father and husband and a guide to all true good, to pant after the mere shadow of a leader, as though thou hadst not within thyself a true leader, and all riches in thine own power? And these must thou lose and forfeit, if thou descend to the flesh, instead of turning towards that which saves and is saved.

\paragraph{10} As for my shadow and visible image, as thou wast not profited by their presence, so now their absence is not hurtful, if thou train thyself to escape from the body. And thou wouldst meet with me in all purity, and I should be most truly present and associated with thee, night and day, in purity and with the fairest kind of converse which can never be broken up, if thou wouldst practise entering into thyself, to collect together all the powers which the body has scattered and broken up into a multitude of parts unlike their former unity, to which concentration lent strength. Thou shouldst collect and combine into one the thoughts implanted within thee, endeavouring to isolate those that are confused, and to drag to light those that are enveloped in darkness. The divine Plato too made this his starting-point, summoning us away from the sensible to the intelligible. Also if thou wouldst remember, thou wouldst combine what thou hast heard, and recall it by memory, desiring to turn thy mind to discourses of this kind as to excellent counsellors, and afterwards practising in action what thou hast learnt, guarding it carefully, even amid thy labours.

\paragraph{11} Reason tells us that the Divine is present everywhere and in all men, but that only the mind of the wise man is sanctified as its temple, and God is best honoured by him who knows Him best. And this must naturally be the wise man alone, who in wisdom must honour the Divine, and in wisdom adorn for it a temple in his thought, honouring it with a living statue, the mind moulded in His image. [...] Now God is not in need of any one, and the wise man is in need of God alone. For no one could become good and noble, unless he knew the goodness and beauty which proceed from the Deity. Nor is any man unhappy, unless he has fitted up his soul as a dwelling-place for evil spirits. To a wise man God gives the authority of a god. And a man is purified by the knowledge of God, and issuing from God, he follows after righteousness.

\paragraph{12} Let God be at hand to behold and examine every act and deed and word. And let us consider Him the author of all our good deeds. But of evil we ourselves are the authors, since it is we who made choice of it, but God is without blame. Wherefore we should pray to God for that which is worthy of Him, and we should pray for what we could attain from none other. And we must pray that we may attain after our labours those things that are preceded by toil and virtue; for the prayer of the slothful is but vain speech. Neither ask of God what thou wilt not hold fast when thou hast attained it, since God's gifts cannot be taken from thee, and He will not give what thou wilt not hold fast. What thou wilt not require when thou art rid of the body, that despise, but practise thyself in that thou wilt need when thou art set free, calling on God to be thy helper. Thou wilt need none of those things which chance often gives and again takes away. Do not make any request before the fitting season, but only when God makes plain the right desire implanted by nature within thee.

\paragraph{13} Thus can God best be reflected, who cannot be seen by the body, nor yet by an impure soul darkened with vice. For purity is God's beauty, and His light is the life-giving flame of truth. Every vice is deceived by ignorance, and turned astray by wickedness. Wherefore desire and ask of God what is in accordance with His own will and nature, well assured that, inasmuch as a man longs after the body and the things of the body, in so far does he fail to know God, and is blind to the sight of God, even though all men should hold him as a god. Now the wise man, if known by only few, or, if thou wilt, unknown to all, yet is known by God. Let then thy mind follow after God and by likening itself unto Him reflect His image; let the soul follow the mind, and the body be subservient to the soul as far as may be, the pure body serving the pure soul. For if it be defiled by the emotions of the soul, the defilement reacts upon the soul itself.

\paragraph{14} In a pure body where soul and mind are loved by God, words should conform with deeds; since it is better for thee to cast a stone at random than a word, and to be defeated speaking the truth than to conquer through deceit; for he who conquers by deceit is worsted in his character. And lies are witnesses unto evil deeds. It is impossible for a man who loves God also to love pleasure and the body, for he who loves these must needs be a lover of riches. And he who loves riches must be unrighteous. And the unrighteous man is impious towards God and his fathers, and transgresses against all men. Though he slay whole hecatombs in sacrifice, and adorn the temples with ten thousand gifts, yet is he impious and godless and at heart a plunderer of holy places. Wherefore we should shun all addicted to love of the body as godless and impure.

\paragraph{15} Do not associate with any one whose opinions cannot profit thee, nor join with him in converse about God. For it is not safe to speak of God with those who are corrupted by false opinion. Yea, and in their presence to speak truth or falsehood about God is fraught with equal danger. It is not fitting for a man who is not purified from unholy deeds to speak of God himself, nor must we suppose that he who speaks of Him with such is not guilty of a crime. We should hear and use speech concerning God as though in His presence. Godlike deeds should precede talk of God, and in the presence of the multitude we should keep silence concerning Him, for the knowledge of God is not suitable to the vain conceit of the soul. Esteem it better to keep silence than to let fall random words about God. Thou wilt become worthy of Him if thou deem it wrong either to speak or do or know aught unworthy of Him. Now a man who was worthy of God would be himself a god.

\paragraph{16} Thou wilt best honour God by making thy mind like unto Him, and this thou canst do by virtue alone. For only virtue can draw the soul upward to that which is akin to it. Next to God there is nothing great but virtue, yet God is greater than virtue. And God strengthens the man who does noble deeds. But an evil spirit is the instigator of evil deeds. The wicked soul flies from God, and would fain that His providence did not exist, and it shrinks from the divine law which punishes all the wicked. But the wise man's soul is in harmony with God, and ever beholds Him and dwells with Him. If the ruler takes pleasure in the ruled, then God too cares for the wise man and watches over him. Therefore is the wise man blest, because he is in God's keeping. 'Tis not his speech that is acceptable to God, but his deed; for the wise man honours God even in his silence, while the fool dishonours Him even while praying and offering sacrifice. Thus the wise man only is a priest; he only is beloved by God, and knows how to pray.

\paragraph{17} The man who practises wisdom practises the knowledge of God; and he shows his piety not by continued prayers and sacrifices but by his actions. No one could become well-pleasing to God by the opinions of men or the vain talk of the Sophists. But he makes himself well-pleasing and consecrate to God by assimilating his own disposition to the blessed and incorruptible nature. It is he too who makes himself impious and displeasing to God, for God does not injure him (since the divine nature can only work good), but he injures himself, chiefly through his wrong opinion concerning God. He who disregards the images of the gods is less impious than the man who holds the opinions of the multitude concerning God. But do thou entertain no thought unworthy of Him or of His blessedness and immortality.

\paragraph{18} The chief fruit of piety is to honour God according to the laws of our country, not deeming that God has need of anything, but that He calls us to honour Him by His truly reverend and blessed majesty. We are not harmed by reverencing God's altars, nor benefited by neglecting them. But whoever honours God under the impression that He is in need of him, unconsciously deems himself greater than God. 'Tis not the anger of the gods that injures us, but our own ignorance of their nature. Anger is foreign to the gods, for anger is involuntary, and there is nothing involuntary in God. Do not then dishonour the divine nature by false human opinions, since thou wilt not injure the eternally blessed One, whose immortal nature is incapable of injury, but thou wilt blind thyself to the conception of what is greatest and chiefest.

\paragraph{19} Again thou couldst not suppose that I say this to exhort thee to reverence God, since it would be absurd to command this, as though the matter admitted of question. We do not worship Him only by doing or thinking this or that, neither can tears or supplications turn God from His purpose, nor yet is He honoured by sacrifices nor glorified by plentiful offerings; but it is the godlike mind that remains stably fixed in its place that is united to God. For like must needs approach like. The sacrifices of fools are mere food for fire, and from the offerings they bring temple-robbers get the supplies for their evil life. But do thou, as I bade, let thy temple be the mind that is within thee. This must thou tend and adorn, that it may be a fitting dwelling for God. Let not the adornment and the reception of God be but for a day, to be followed by mockery and folly and the return of the evil spirit.

\paragraph{20} If then thou ever bear in mind that wheresoever thy soul walks and inspires thy body with activity, God is present and overlooks all thy counsels and actions, then wilt thou feel reverence for the unforgotten presence of the spectator, and thou wilt have God to dwell with thee. And even though thy mouth discourse the sound of some other thing, let thy thought and mind be turned towards God. Thus shall even thy speech be inspired, shining through the light of God's truth and flowing the more easily; for the knowledge of God makes discourse short.

\paragraph{21} But wheresoever forgetfulness of God shall enter in, there must the evil spirit dwell. For the soul is a dwelling-place, as thou hast learnt, either of gods or of evil spirits. If the gods are present, it will do what is good both in word and in deed; but if it has welcomed in the evil guest, it does all things in wickedness. Whensoever then thou beholdest a man doing or rejoicing in that which is evil, know that he has denied God in his heart and is the dwelling-place of an evil spirit. They who believe that God exists and governs all things have this reward of their knowledge and firm faith: they have learnt that God has forethought for all things, and that there exist angels, divine and good spirits, who behold all that is done, and from whose notice we cannot escape. Being persuaded that this is so, they are careful not to fall in their life, keeping before their eyes the constant presence of the gods whence they cannot escape. They have attained to a wise mode of life, and know the gods and are known by them.

\paragraph{22} But they who believe that the gods do not exist and that the universe is not governed by God's providence, have this punishment: they neither trust the evidence of their own minds, nor that of others who assert that the gods exist, and that the universe is not directed by whirling motion void of reason. Thus they have cast themselves into unspeakable peril, trusting to an unreasoning and uncertain impulse in the events of life, and they do all that is unlawful in the endeavour to remove the belief in God. Assuredly such men are forsaken by the gods for their ignorance and unbelief. Yet they cannot flee and escape the notice of the gods nor of justice their attendant, but having chosen an evil and erring life, though they know not the gods, yet are they known by them and by justice that dwells with the gods.

\paragraph{23} Even if they think they honour the gods, and are persuaded that they exist, yet neglect virtue and wisdom, they really have denied the divinities and dishonour them. Mere unreasoning faith without right living does not attain to God. Nor is it an act of piety to honour God without having first ascertained in what manner He delights to be honoured. For if He is gratified and won over by libations and sacrifices, it would not be just that, while all men make the same requests, they should not all obtain the same reward. But if He desires none of these things and delights only in the purification of the mind, which every man can attain of his own free choice, what injustice could there be? If however the divine nature delights in both kinds of service, it should receive honour by sacred rites according to each man's power, and by the thoughts of his mind even beyond that power. It is not wrong to pray to God, for ingratitude is a grievous wrong.

\paragraph{24} No god is responsible for a man's evils, for he has chosen his lot himself. The prayer which is accompanied by base actions is impure, and therefore not acceptable to God; but that which is accompanied by noble actions is pure, and at the same time acceptable.

There are four first principles that must be upheld concerning God---faith, truth, love, hope. We must have faith that our only salvation is in turning to God. And having faith, we must strive with all our might to know the truth about God. And when we know this, we must love Him we do know. And when we love Him we must nourish our souls on good hopes for our life, for it is by their good hopes good men are superior to bad ones. Let then these four principles be firmly held.

\paragraph{25} Next let these three laws be distinguished. First, the law of God; second, the law of human nature; third, that which is laid down for nations and states. The law of nature fixes the limits of bodily needs, and shows what is necessary to these, and condemns all striving after what is needless and superfluous. That which is established and laid down for states regulates by fixed agreements the common relations of men, by their mutual observance of the covenants laid down. But the divine law is implanted by the supreme mind, for their salvation, in the thoughts of reasoning souls, and it is found truthfully inscribed therein. The law of nature is transgressed by him who through folly disregards it, owing to his excessive love for the pleasures of the body. And it is broken and despised by those who, even for the body's sake, strive to master the body. The conventional law is subject to expediency, and is differently laid down at different times according to the arbitrary will of the prevailing government. It punishes him who transgresses it, but it cannot reach a man's secret thoughts and intentions.

\paragraph{26} The divine law is unknown to the soul that folly and intemperance have rendered impure, but it shines forth in self-control and wisdom. It is impossible to transgress this, for there is nothing in man that can transcend it. Nor can it be despised, for it cannot shine forth in a man who will despise it. Nor is it moved by chances of fortune, because it is always superior to chance and stronger than any form of violence. Mind alone knows it, and diligently pursues the search thereafter, and finds it imprinted in itself, and supplies from it food to the soul as to its own body. We must regard the rational soul as the body of the mind, which the mind nourishes by bringing into recognition, through the light that is in it, the thoughts within, which mind imprinted and engraved in the soul in accordance with the truth of the divine law. Thus mind is become teacher and saviour, nurse, guardian and leader, speaking the truth in silence, unfolding and giving forth the divine law; and looking on the impressions thereof in itself it beholds them implanted in the soul from all eternity.

\paragraph{27} Thou must therefore first understand the law of nature, and then proceed to the divine law, by which also the natural law hath been prescribed. And if thou make these thy starting-point thou shalt never fear the written law. For written laws are made for the benefit of good men, not that they may do no wrong, but that they may not suffer it. Natural wealth is limited, and it is easy to attain. But the wealth desired of vain opinions has no limits, and is hard to attain. The true philosopher therefore, following nature and not vain opinions, is self-sufficing in all things; for in the light of the requirements of nature every possession is some wealth, but in the light of unlimited desires even the greatest wealth is but poverty. It is no uncommon thing to find a man who is rich if tried by the standard at which nature aims, but poor by the standard of vain opinions. No fool is satisfied with what he possesses; he rather mourns for what he has not. Just as men in a fever are always thirsty through the grievous nature of their malady, and desire things quite opposed to one another, so men whose souls are ill-regulated are ever in want of all things, and experience ever-varying desires through their greed.

\paragraph{28} Wherefore the gods too have commanded us to purify ourselves by abstaining from food and from love, bringing those who follow after piety within the law of that nature which they themselves have formed, since everything which transgresses this law is impure and deadly. The multitude, however, fearing simplicity in their mode of life, because of this fear, turn to the pursuits that can best procure riches. And many have attained wealth, and yet not found release from their troubles, but have exchanged them for greater ones. Wherefore philosophers say that nothing is so necessary as to know thoroughly what is unnecessary, and moreover that to be self-sufficing is the greatest of all wealth, and that it is honourable not to ask anything of any man. Wherefore too they exhort us to strive, not to acquire some necessary thing, but rather to remain of good cheer if we have not acquired it. 

\paragraph{29} Neither let us accuse our flesh as the cause of great evils, nor attribute our troubles to outward things. Rather let us seek the cause of these things in our souls, and casting away every vain striving and hope for fleeting joys, become completely masters of ourselves. For a man is unhappy either through fear or through unlimited and empty desire. Yet if he bridle these, he can attain to a happy mind. In as far as thou art in want, it is through forgetfulness of thy nature that thou feelest the want. For hereby thou causest to thyself vague fears and desires. And it were better for thee to be content and lie on a bed of rushes than to be troubled though thou hadst a golden couch and a luxurious table acquired by labour and sorrow. Whilst the pile of wealth is growing bigger, life is growing wretched.

\paragraph{30} Do not think it unnatural that when the flesh cries out for anything, the soul should cry out too. The cry of the flesh is, ``Let me not hunger, or thirst, or shiver,'' and 'tis hard for the soul to restrain these desires. 'Tis hard, too, by help of its own natural self-sufficing to disregard day by day the exhortations of nature, and to teach her to esteem the concerns of life as of little account. And when we enjoy good fortune, to learn to bear ill fortune, and when we are unfortunate not to place too much value on good fortune. And to receive with a calm mind the good gifts of fortune, and to stand firm against her seeming ills. Yea, all that the many hold good is but a fleeting thing.

\paragraph{31} But wisdom and knowledge have no part in chance. It is not painful to lack the gifts of chance, but rather to endure the unprofitable toil caused by vain opinions. Every disturbance and unprofitable desire is removed by the love of true philosophy. Vain is the word of that philosopher who can ease no mortal trouble. As there is no profit in the physician's art unless it cure the diseases of the body, so there is none in philosophy, unless it expel the troubles of the soul. These and other like commands are laid on us by the law of our nature.

\paragraph{32} The divine law cries aloud in the pure region of the mind: ``Unless thou remember that thy body is joined to thee as the outer covering to the child in the womb and the stalk to the sprouting corn, thou canst not know thyself.'' Nor can any one know himself who does not hold this opinion. As the outer covering grows with the child, and the stalk with the corn, yet, when they come to maturity, these are cast away, thus too the body which is fastened to the soul at birth is not a part of the man. But as the outer covering was formed along with the child that it may come to being in the womb, so likewise the body was yoked to the man that he may come to being on earth. In as far as a man turns to the mortal part of himself, in so far he makes his mind incommensurate with immortality. And in as far as he refrains from sharing the feelings of the body, in such a measure does he approach the divine. The wise man who is beloved of God strives and toils as much for the good of his soul as others do for the good of their body. He does not deem it sufficient merely to remember what he has heard, but strives by practising it to hasten on towards his duty.

\paragraph{33} Naked was he sent into the world, and naked shall he call on Him that sent him. For God listens only to those who are not weighed down by alien things, and guards those who are purified from corruption. Consider it a great help towards the blessed life if the captive in the thraldom of nature takes his captor captive. For we are bound in the chains that nature has cast around us, by the belly, the throat and the other members and parts of the body, and by the use of these and the pleasant sensations that arise therefrom and the fears they occasion. But if we rise superior to their witchcraft, and avoid the snares laid by them, we lead our captor captive. Neither trouble thyself much whether thou be male or female in body, nor look on thyself as a woman, for I did not approach thee as such. Flee all that is womanish in the soul, as though thou hadst a man's body about thee. For what is born from a virgin soul and a pure mind is most blessed, since imperishable springs from imperishable. But what the body produces is held corrupt by all the gods.

\paragraph{34} Much discipline therefore is needful to win the rule over the body. Often men cast off certain parts of the body; be thou ready for the soul's safety to cast away the whole body. Hesitate not to die for that for whose sake thou art willing to live. Let reason then direct all our impulses, and banish from us tyrannous and godless masters. For the rule of the passions is harder than that of tyrants, since it is impossible for a man to be free who is governed by his passions. As many as are the passions of the soul, so many cruel masters have we. 

\paragraph{35} Try not to wrong thy slaves nor to correct them when thou art angry. And before correcting them, prove to them that thou dost this for their good, and give them an opportunity for excuse. When purchasing slaves, avoid the stubborn ones. Practise doing many things thyself, for our own labour is simple and easy. And men should use each limb for the purpose for which nature intended it to be used, for nature needs no more. They who do not use their own bodies, but make excessive use of others, commit a twofold wrong, and are ungrateful to nature that has given them these parts. Never use thy bodily parts merely for the sake of pleasure, for it is far better to die than to obscure thy soul by intemperance [...] correct the vice of thy nature. If thou give aught to thy slaves, distinguish the better ones by a share of honour [...] for it is impossible that he who does wrong to man should honour God. But look on the love of mankind as the foundation of thy piety. And [...]\footnote{Here the manuscript ends abruptly.}

\end{document}
