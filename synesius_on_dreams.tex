\documentclass[12pt]{article}
\usepackage{fontspec}
\usepackage{microtype}
\setmainfont[Ligatures=TeX,Numbers=Lowercase]{EB Garamond}
\renewcommand{\thesection}{\textsc{\roman{section}}} 
\renewcommand{\thesubsection}{\thesection.\textsc{\roman{subsection}}}
\renewcommand{\thesubsubsection}{\thesubsection.\textsc{\roman{subsubsection}}}

\title{On Dreams}
\author{Saint Synesius\footnote{Synesius of Cyrene was born of a noble family
\textit{circa} \textsc{a.~d.}~370 in Cyrene, a city of the Libyan Pentapolis,
Egypt. At first a heathen he subsequently became a Christian. About 393 he
became a pupil at Alexandria of the celebrated Hypatia. In 403 was made Bishop
of Ptolemais which he did not accept till 410. He died \textit{circa}
\textsc{a.~d.}~413. He is best known through his Hymns. The Roman Catholic
Church canonized him as St.~Synesius.} \and Isaac Myer (tr.)\footnote{I.~Myer,
``On Dreams,'' in \textit{The Platonist,} vol.~4, nos.~4--6, 1888.} \and @sdi
(ed.)\footnote{Myer has a tendency towards precision, preferring to leave Greek
terms in the text and explain them with footnotes; I have edited these to
simply translate the terms as an aid to readability. I have also normalized
the spelling of proper nouns to more common variants.}}
\date{}

\begin{document}
\maketitle

\section*{Preface}

A very ancient custom, and one which Plato especially uses, is to conceal under
the form of a familiar subject the most serious teachings of philosophy; so
that the truths the discovery of which has been most difficult will not again
escape from the memory of men nor, being disseminated, receive the
contaminations of the profane vulgar. Such is the design that I propose in this
book. Whether my work conforms in all its parts to the antique mode of writing
I leave to the judgement of industrious and enlightened readers.


\section{The prophetic faculty is the noblest subject of study for man.}

If dreams prophesy the future, if visions which present themselves to the mind
during sleep afford some \textit{indicia} whereby to divine future things,
dreams will be at the same time true and obscure, and even in their obscurity
the truth will reside. ``The Gods with a thick veil have covered human
life.''\footnote{Hesiod, \textit{Works and Days} 42.}

To obtain things of the greatest value without labor is a happiness which
appertains only to the gods; but by men not only virtue but all blessings ``can
alone be achieved by sweat.''\footnote{Hesiod, \textit{Works and Days} 287.}
Now the prophetic power is the greatest of goods: it is through knowledge and
the gnostic faculty that God differs from man, and man from the brute. God
knows all by virtue of His own nature, but man by the aid of the prophetic
power may add much to his natural knowledge. The vulgar man knows only the
present; that which is future he can only conjecture. Calchas alone in the
assembly of all the Greeks apprehended ``the present, the future, and the
past.''\footnote{\textit{Iliad} \textsc{i} 70.} In Homer, Zeus regulates the
affairs of the gods, because he was ``born first, and therefore knows
more;''\footnote{\textit{Iliad} \textsc{xiii} 355.} because knowledge is the
privilege of the aged. If the poet thus recalls Zeus' age, it is because the
years bring with them that wisdom to which nothing else can be compared. If we
think, from other passages, that the supremacy of Zeus was the result of the
vigor of his arms, because Homer said, ``He carried it by
force,''\footnote{\textit{Iliad} \textsc{xv} 165.}---we do not understand
poetry, and cannot grasp the philosophy which it encloses, \textit{viz.}, that
the gods are no other than pure Intelligences. After saying that Zeus is very
strong, the poet immediately adds that he is the oldest, which signifies that
Zeus is the oldest Intelligence. But the vigor of intelligence, is this any
other thing than wisdom? Whoever may be the god who rules the other gods, it is
because he is wise that he reigns. Because he is superior in wisdom, ``he
carries it by force'' signifies that he knows more than the others. The sage
has then a species of affinity with God, because he endeavors to approach Him
by the faculty of knowledge, and exerts himself about intuitive thought which
is the essence of God. These facts show plainly that one of the most noble
subjects of research for man is the prophetic faculty.


\section{The universe is an animated existence which is connected together in
all its parts. The different kinds of prophesy.}

All things, by their relation to each other, can give omens; because all
together they are only different parts of one animal, the
Universe.\footnote{This is the doctrine of the Macrocosm and Microcosm.} The
Universe may be compared to a book in which are inscribed letters of every
description, such as Ph{\oe}nician, Egyptian, and Assyrian: the wise man
deciphers these letters; and he is really wise who learns by and from nature,
and another is wise in other things---this one more, and the other less. Thus
for example one merely learns the syllables, another apprehends the general
style, a third comprehends the whole discourse. Wise men see beforehand that
which may happen; some by regarding the planets, others the fixed stars, others
the comets and fires which traverse space. Some also prophesy by inspecting the
entrails of victims or by listening to the songs of birds, or through watching
their flight and their haunts. There are other omens by the aid of which they
can predict the futures, as words, or chance meetings; all are able to draw
prognostications from everything. If a bird had our intelligence, man would
serve it, as the bird serves man, for knowledge and divination: because we are
to them that which they are as to us, a race which, always renewing itself and
and as old as the world, is entirely qualified to give signs.


\section{All things have a mutual affinity and act upon each other.}

It was necessary that all the parts of this grand whole, animated with a common
life, should be united by an intimate relation, as members of the same body.

It is thus perhaps that we may be able to explain the enchantments by
magicians: because as there are omens in nature, attractions also exist. The
wise man is he who knows how all things are bound together in this world; he
makes one thing come to him through the intermediary of another thing; with the
assistance of the present objects he extends his power over the most distant
objects; he works by means of words, figures, and material substances.

In ourselves the suffering of one organ communicates itself to other organs; a
pain in the finger or the foot sometimes brings a tumor in the groin, while all
the intermediary organs do not suffer anything: it is because the groin and the
foot both belong to the same body and have their especial relations. Among the
gods who are in this world, there are some who have affinities and sympathies
with certain stones and certain plants, and these affinities are such that by
and through these stones or plants they may be attracted. In the same way the
musician who produces a note adds to it not the nearest note to it, but the
fifth and the octave.

And this arises from the ancient harmony of things. At present there is between
the different parts, as in a family, dissimilarity; because the world is not a
simple unity, but a composite unity. The elements therefore sometimes are in
accord, at other times are in conflict; but their strife always results in the
harmony of the whole. So the sounds which the lyre gives are an assemblage of
dissonances and consonances: it is from contraries that the unity is born,
which makes of the lyre, as of the universe, a well ordered whole.


\section{This reciprocal action of things can only be exercised in the world.
Obscurity is essential to divination.}

Archimedes, the Sicilian, asked for a fulcrum situated outside of the earth to
move the earth, saying: ``Whilst I inhabit it I cannot act upon it.'' But it is
entirely different with those who penetrate the mysteries of the world, and
thus acquire a portion of the divinatory or prophetic science: if such a one
was placed outside of the world, he could not any longer use his knowledge;
because he exercises it upon the world, and by means of the world. Going out of
our universe, you will look in vain: from the observation of the phenomena
which are produced above the region in which the soul is diffused you will not
apprehend anything. Over that which is divine, outside of the world, magic has
no power. ``He abides by himself, free of all care he is utterly regardless of
it.''\footnote{\textit{Iliad} \textsc{xv} 106.}

Intellect or Intuitive Reason is essentially independent: it must become
passive before it can be influence by enchantments. The multiplicity of beings
included in the world, and the affinities existing between them, give birth to
all kinds of divination and mysteries; different because they are many, but by
virtue of their affinities are really one great whole. It is not expedient to
speak inconsiderately of the mysteries, by reason of our respect for the laws
of the State; but it is permissible to explain divination or the prophetic
power. We have eulogized that art in general; we will now consider especially
the most perfect of all divinations. They all present the common characteristic
of obscurity; the attentive contemplation of the things of this world will not
help in any way to dissipate this obscurity. Obscurity, as we shall see, is
essential to divination, as mystery is to the sacred initiations. It is thus
that the oracle of Delphi is not comprehended by all because it expresses
itself in ambiguous terms; and when the gods pointed out to the Athenians how
they would be able to save themselves, the assembled people could not have
grasped the meaning of its words, if Themistocles had not been there to explain
them. Thus we must not reject divination through dreams because, in common with
all divination and the oracles, it has obscurity.


\section{Of Intellect, soul, reason, and imagination.}

Of all disciplines we should specially study that which is of us, and within
us, and the peculiar property of each soul. Intellect or Intuitive Reason
contains in itself the images of the things which are real, as the ancient
philosophy teaches; we may add that the soul contains images of things which
are contingent. There is then, between Intellect and the soul, the same
affinity as between the absolute and contingent. We will invert the order of
the terms. We will join the first to the third, the second to the fourth: the
proportions will still remain true, as knowledge demonstrates to us. It will
thus be established that the soul, as we have advanced, contains in itself the
images of the things which become. It encloses them wholly, but it produces
them outwardly only at a convenient time; imagination is similar to a mirror in
which it reflects itself, so that the animal may perceive the images which have
their seats in the soul. We are not conscious of these actions of the
Intellect, so long as the governing faculty does not reveal them to us, all
that which it ignores escapes the knowledge of the animal; in the same manner
we are not able to grasp any idea of the things which are in the first
soul,\footnote{Synesius holds to the idea of several divisions of the
spiritual. By the ``governing faculty'' it is necessary to understand Reason or
Wisdom. The first division is the Rational soul, in opposition to the soul
evident to the Senses, the Vegetative Soul. The Kabbalists divided the
spiritual into three main divisions, \textit{Neshamah} higher soul,
\textit{Ruach} spirit, and \textit{Nephesh} vital soul. The first, the highest,
they divided into two others still more idealized.} so long as the imagination
does not receive the images. This imaginative life is an inferior life, a
particular condition of our nature. The senses are present to it: we see
colors, we hear sounds, we touch and grasp objects, although our corporeal
organs remain inactive; perhaps in this state our perceptions are more pure. It
is thus that we often enter into conversation with the gods: they warn us,
answer us, and give to us useful advice. So that I am not surprised that some
have owed to a sleep the discovery of a treasure; and that one may have gone to
sleep very ignorant, and after having had in a dream a conversation with the
Muses, awakened an able poet, which has happened in my time to some, and in
which there is nothing strange. I do not speak of those who have had, in their
sleep, the revelation of a danger which threatened them, or the knowledge of a
remedy that would cure them. But when sleep opens the way to the most perfect
inspections of true things to the soul which previous had not desired these
inspections, nor thought concerning the ascent to Intellect---and arouses it to
pass beyond nature and reunite itself to the intelligible sphere from which it
has wandered so far that it does not now know even from which it
came,---\textit{this}, I say, is most marvelous and obscure.


\section{The power of imagination, which is pre-eminently the understanding.}

If one thinks it extraordinary that the soul may thus ascend to the superior
region, and does not believe that the way to this felicitous union lies through
the imagination, let him hear the sacred oracles when they speak about the
different roads which lead to the higher sphere. After enumerating the various
\textit{subsidia} which help the ascent of the soul by arousing and developing
its powers, they say:

\begin{verse}
``By lessons some are enlightened,\\
By sleep others are inspired.''\footnote{\textit{Sibylline Oracles.}}
\end{verse}

\noindent You see the distinction which the oracle establishes: upon the one
side, inspiration; upon the other, study; the former it says is instruction
whilst one is awake, the latter when asleep. Whilst awake, it is always a man
who is the instructor; but when asleep, it is from God that the knowledge
comes. They know from the first all that which is taught to them; because in
giving knowledge thus God does not instruct in the usual manner. These facts
are set forth to show the excellence of the imaginative life to those who do
not esteem it. I am not astonished at their opinion: through their
superabundant wisdom they are attached to practices condemned by the sacred
oracles, for the oracles say:

\begin{verse}
``Sacrificers and victims\\
Are only vain amusements.''\footnote{\textit{Sibylline Oracles.}}
\end{verse}

\noindent The oracles advise us to renounce such things. But the men of whom I
speak, being superior to the multitude in their own opinion, practice almost
all kinds of divination. However they neglect divination by dreams, a method
within the reach of all, alike of the ignorant and the wise. But why! Does not
the wise man know better that which is common to all? Almost all good things,
and especially the most precious, are the common property of humanity. In the
universe nothing is more magnificent and divine that the sun, nor more common.
It is a great happiness to behold God; but to know God by means of the
imagination is a higher intuition. Imagination is the sense of the senses,
necessary to all others; it is the first body of the soul. It dwells within us:
established in the head, as in a citadel which nature has built for us, and it
governs the animal life. The hearing and the sight are not true senses, but
rather instruments of sense, which put the animal in relation with the exterior
world; in the service of imagination they transmit to their mistress
impressions received by them from without, sensations which are transmitted to
us from the objects by which we are surrounded. Imagination is the collective
sense in which are united our various senses: in reality it is that which hears
and which sees; it is through it that all the perceptions occur; and it assigns
to each organ its particular function. From it all the faculties proceed: they
are like the rays which go out from the center and which meet wholly in the
center: many in progression, and one and the same in origin. The sense to which
the organs are indispensable is a purely material sense; or, to speak more
correctly, it is only a sense when it enters into the service of the
imagination: imagination is the sense which has power of acting instantaneously
without intermediaries. It has a divine character through which it approaches
intuitive Intellect.


\section{Imagination is less fallible than our physical senses, which often
deceive us.}

We hold our physical senses in great estimation because they put us in harmony
with the world; and that which we think we know the best is that which strikes
our attention. But if we have only disdain for the imaginations, because it is
often at variance with our senses, we must not forget that the eye itself
frequently deceives us; at one time it does not perceive things, at others it
sees them not as they really are, because of the medium through which they are
seen. According to the distance things appear larger or smaller; when in water
they seem greater, and refraction frequently makes a straight ray appear
broken. Sometimes moreover the eye suffers, and everything to it appears dim
and confused. In the same way when the imagination is diseased do not expect
clear and distinct visions.

What is the nature of these maladies? From whence come the imperfections which
imagination contracts? How can it recover and regain health? A profound
philosopher alone in able to tell us, and prescribe the sacred remedies which
can cure the imagination and again render it divine. But in order that God may
come and visit it, it is necessary that it first expel all the foreign
impressions which it has received. When we live in conformity to nature,
imagination remains pure and undefiled; it watches all its energies; it is thus
that it approaches truly to the soul: it then enters into relation with the
latter; it is then no longer a stranger to the soul, as is our corporeal
envelope, upon which the beneficent influence of the spiritual principle does
not act. Imagination is the vehicle of the soul, and according as the latter
inclines more toward virtue or vice, is more subtle and {\ae}therial or heavier
and more terrestrial. It occupies the mean between the existence endowed with
and existence deprived of reason---between spirit and matter---it serves them
as a medium, and thus united the two extremes: that is why its nature cannot be
seized with exactness by the philosopher.


\section{Imagination has been bestowed upon a multitude of existences, and it
is through it that we form our thoughts.}

Closely allied to matter and spirit, imagination borrows from each as to it
seems expedient, and, while guarding its own nature, forms its conceptions from
most opposite elements. The imaginative essence has been shared with a
multitude of existences; it descends even to animals devoid of intelligence:
then it is no longer the chariot upon which the divine soul is seated; it is
only itself which is seated upon the inferior faculties. It holds in the breast
the place of reason; it feels and acts sufficiently through
itself.\footnote{Imagination in that condition is only instinct.} With certain
animals it is purified and perfected. There are a multitude of d{\ae}mons of
which the existence is wholly imaginative: these are only phantoms of which the
apparitions are connected with contingent things. In man imagination can do
much through its inherent virtue, and much more through its association with
Intellect. We are ale to form thoughts only with the aid of imagination, unless
perhaps in short instants during which some men at once seize the
truth.\footnote{The most subtle and highest form of thought is intuition, but
we cannot formulate this into ideas without the assistance of imagination.} To
wholly transcend the imagination is a thing not less beautiful than difficult.
Happy the man to whom the years bring intelligence and wishom, said Plato, when
speaking of pure wisdom.\footnote{See the \textit{Philebus.}} But ordinary life
is dominated by imagination, or by intelligence using imagination as an
instrument.


\section{In this world imagination has been associated with the soul; sometimes
it commands the latter and at other times obeys it.}

This breathing animal, which the wise have called a soul endowed with breath,
takes all species of forms and becomes a god, a d{\ae}mon, a phantom, in whom
the soul receives the punishment for its faults. The oracles agree in saying
that the soul will have in the other world an existence conforming to the
visions which sleep now brings to it,\footnote{As the ideas which still pursue
us in our sleep are those which have occupied us during our waking hours, they
will still continue in the other life.} and philosophy assures us that all our
present life is only a preparation for the life which is to follow. Virtuous,
the soul renders the imagination lighter: it crushes it under the weight of its
stains. Naturally imagination raises itself above, when it is endowed with
heat and dryness: these are its wings, such is the meaning which it is
essential to attach to the expressions of Heraclitus, when he says, that the
soul truly wise is brilliant and dry; on the contrary, when it is thick and
humid, imagination is drawn by its weight towards the lower regions, into the
subterranean depths, the abode of the bad spirits; there it endures, in
punishments, an unhappy existence: however, in time and with much effort it can
in another life purify itself, and rise again towards heaven. At its entrance
into life two roads open before it; it goes some times in the good road, and at
others in the bad;\footnote{The idea of the two roads is very ancient. See the
oldest Church Manual, called the Teaching of the Twelve Apostles, etc., by
Phillip Schaff, New York, 1885, p.~18 sq. The Teaching of the Twelve Apostles
with illustrations from the Talmud, etc., by Charles Taylor, D.~D.~London,
1886.} then comes the soul, which descending from the celestial
sphere\footnote{This idea is also very ancient. It is set forth in the Republic
of Plato (\textit{circa} 428--429 \textsc{b.~c.}) in the narration of Er the
Pamphylian, and is also in the Republic of Cicero (b.~106, d.~43
\textsc{b.~c.}) in the dream of Scipio; a commentary on which may be found in
the writings of Macrobius (b.~\textit{circa} 376, d.~420 \textsc{a.~d.}) The
Hebrew Kabbalah has the same idea. Compare the prayer of Eliyahu in the
\textit{Tikunei HaZohar}, also \textit{Disputatio Cabalistica,} etc., by Joseph
De Voisin, Paris, 1635, pp.~277--279. He quotes from Plotinus to the same
effect. The idea most probably came from the ancient Chald{\ae}ans, at least
Pausanias strongly implies it. The Mysteries of Mithra also had some connection
with it. Compare also ``The Face in the Moon's Orb'' by Plutarch, likely a
relic of the religious philosophy of the Druids. The Manich{\ae}ans also held
similar views.} takes hold of the imagination; uses it as a chariot, in order
that it may accomplish its journey in the physical world; and strives to lead
it back to the higher regions, or at least not to remain borne down with the
soul in matter. Without doubt it is difficult for them to separate; sometimes,
however, when the imagination will not obey, the soul liberates itself from its
society: for we should not disbelieve known and true mysteries. It is a
disgrace for the soul to return to the intelligible sphere if she neglects to
restore that which is alien to her true nature, and leaves about the earth
what she had received from on high.

Thanks to the initiations and the divine favor, there are some men who attain
to the redemption of their souls from the bonds of imagination; but usually if
once united they keep together, and the soul is either drawn by or it draws the
imagination; and their association continues until the time when the soul
returns to the place from which it has been separated. When the imagination
falls under the weight of its miseries it draws down in the fall the soul which
has not known how to preserve it. This is the danger which the oracles point
out to the intelligent principle which is within us:

\begin{verse}
``Do not drag it down into this muddy world,\\
Into its deep gulfs, its sad and black kingdoms,\\
Somber hideous hells, entirely peopled with
phantoms.''\footnote{\textit{Sibylline Oracles.}}
\end{verse}

\noindent Indeed an unreasoning and stupid existence is not worthy to have
intelligence; but the phantom, because of the elements which compose it, enjoys
itself in the lower regions because the similar seeks only the similar.


\section{If the soul permits itself to be subdued by the attraction of matter
it becomes unhappy.}

If in this union intelligence becomes entirely confounded with imagination, it
plunges into an intoxication of the greatest voluptuousness. Now the height of
evil is to cease feeling it is evil, because then one will not seek for a cure:
and thus it is that we do not dream of banishing the callosities from which we
no longer suffer. Repentance is an aid to return to a better life. When we are
tormented with out condition we strive to leave it. To intend is to have
already accomplished a half of the expiation; because then all the actions and
words tend to the end. But when the will is absent, the expiatory ceremonies
have no longer any meaning; in order that they keep their efficacy, it is
essential that the soul be a consenting element. So the troubles which strike
us on different sides are marvelously proper to establish moral order; taking
the place of false joys the chagrins of life purify the soul. Even misfortunes
which seem to us unmerited are useful in this, that they deliver us from a too
exclusive attachment to the things of this world. It is thus that Providence
reveals itself to the wise, whilst the fools will not admit that it is
impossible for the soul to free itself from matter, when it has not been tried
by the sufferings in this world. The pleasures of this earth are then only a
trap that demons lay for the soul.  Others say, that on leaving this life it
drinks of a beverage which causes it to forget the past: my view is that it is
rather at its entrance into life that it drinks from the cup of deceitful
voluptuousness, the forgetfulness of its destiny. Entering into its first
life\footnote{According to the doctrine of Metempsychosis the soul passes
through a series of successive lives.} to be a servant, its service changes
into slavery; without doubt it ought to a certain extent, by virtue of the laws
of necessity, obey nature, but alas! seduced by the attractions of matter it
resembles those unfortunates who, born free, sell themselves for a time, inflamed
with the beauty of a slave; and in order to remain near that which they love
accept the same master. This is our condition when we allow ourselves to be
charmed with false benefits, by pleasures wholly external, which affect the
body alone; we appear then to admit that matter is beautiful. Matter takes hold
of our admission as if by a secret engagement which we had entered into with
it; and, later, if we wish to free ourselves from it and resume our liberty it
treats us as deserters, it attempts to regain possession of us, and invokes, so
as to make us return under its domination, the faith due to an engagement. It
is especially at this time that the soul needs energy and the divine
assistance: it is not a small affair to have to break, sometimes even
violently, contracted habits; because then (so destiny wills it) all the forces
of matter swoop down upon the rebels so as to crush and punish them. Without
doubt it is this that is meant by the labors of Heracles, which we read of in
the sacred legends, and those combats which other heroes so valiantly sustain,
until the day upon which they could elevate themselves to those heights where
nature no more had any hold upon them. If the soul makes vain efforts so as to
free itself from the walls of its prison, it falls back again on itself; we
have then to sustain rude contests, because matter then treats us as enemies:
it revenges itself upon us for our ineffectual attempts by rigorous
punishments. Then it is no longer that mixed life, of which Homer tells us, the
good and the evil, which go out of two vessels, and which Jupiter (it is still
the poet who is speaking) sovereign dispenser of things in this world,
distributes to men.\footnote{\textit{Iliad} \textsc{xxiv} 526 \textit{sq.}}
Never has he given us a taste of the entirely pure good, but he sometimes has
given us only the evil.


\section{The soul assimilates itself with particles of air and fire which it
ought to carry away with it when it returns to the higher spheres.}

In these different existences the soul ceases not to err if it does not
promptly return to the abode from which it came. See how vast is the course
that imagination can survey. When the soul descends, as we have just said,
imagination, which is heavy, falls and plunges into obscure and dark abysses;
but if the soul rises, it accompanies it and follows it as far as it is
permitted to rise, that is as far as the superior limits of the sublunary
world. Hear what is said upon this subject by the sacred oracles:

\begin{verse}
--- Do not throw\\
``The flower of matter into the terrestrial abysses;\\
The phantom has its place upon the brilliant
summits.''\footnote{The \textit{Sibylline Verses.}}
\end{verse}

\noindent That summit is the opposite of the dark region. But these verses
contain also another meaning which must be searched for: the soul ought not
only to return to the celestial sphere from whence it came, with all which
constitutes its own essence: it ought also to bear away those particles of fire
and air which constitute its second essence, that of phantom, and to which it
assimilated when it was descending towards the earth, before having received
that earthly covering; it takes back above that air and fire with its better
part: for we must not understand by ``the flower of matter'' the divine
body.\footnote{``The flower of matter'' our author deems to be particles of air
and fire. The divine body with him is imagination; this he also calls ``the
first body of the soul.'' He considers the imagination as something very subtle
yet material, corresponding perhaps to Kabbalistic \textit{Ruach:} ``the flower
of matter'' being the \textit{Nepesh,} and the soul the \textit{Neshamah} of
the Kabbalah.} Reason says to us, that the things which have at one time
participated in a common nature and been united to it, cannot any more be
entirely separated, especially when they are neighbors: thus it is that fire
touches the element which is diffused around the world (\textit{i.~e.}~the
{\ae}ther), and it is not like the earth which is in the lowest degree of the
scale of existences.  Admit that the better consents to be allied with that
which is worse, and thus produce an immortal body mingled with mud: if the more
noble of the two associates puts this body under subjection, the part less pure
cannot resist the action of the soul; docile and submissive it follows it
faithfully. Thus it is that the imagination, this intermediary essence, in
yielding to the direction of the soul, the superior essence, far from changing
itself, purifies itself and rises with the soul towards heaven. If there are
limits which it cannot pass, at least it elevates itself above the elements,
and reaches to the luminous spaces; for, as the oracles say, it has its place
in the brilliant region, \textit{i.~e.}~the circular vault which surrounds us.
But we have spoken sufficiently of the loads which the imagination makes to the
elements: you can grant or refuse your belief as to this dogma; but that which
is certain is, that the corporeal essence which comes from on high ought
necessarily, when the soul returns to its principle, raise itself and also take
its flight and join itself to the celestial spheres; that is to say, return to
its own nature.


\section{The two different destinies of the soul and the imagination.}

There are then two destines opposed to each other, one obscure, the other
brilliant; one the height of happiness, the other the excess of misery. But
between these two extreme limits, in this sublunary world, there are---do you
not think it?---a great number of intermediary stations which are neither the
light nor the darkness. The soul with the imagination can go over all this
space, changing its state, habits, and life, according to its location. When it
returns to its original nobleness, it is the receptacle of truth, pure,
brilliant, incorruptible, it is divine, and in order to be able to see the
future has only to wish it. But when it falls into the lower regions, it
contains only darkness, uncertainty and deceit, for the imagination in
obscuring itself becomes incapable of discerning things clearly. When it is
between the two extreme points, the soul has one part truth, the other part,
error. Thus it is that we can determine to which degree of the ladder the
different d{\ae}mons are placed. For to remain always or nearly always in the
truth, is the property of the divine or quasi-divine being; but to deceive
themselves without cessation, when they endeavor to foresee the future, is the
lot of those who themselves wallow in matter, blinded by their haughty
passions. The d{\ae}mons who retain celestial bonds, become gods or spirits of
a superior order; the raise themselves, and go to occupy the region prepared
for the most noble essences.


\section{How we are able to purify the soul and imagination. The excellence of
contemplation.}

In that way we can predict what place the human soul occupies. A man in whom
the imagination, pure and well regulated, perceives whilst awake or asleep only
faithful images of things, can be tranquil as to the state or condition of his
soul; it is the best condition. Now it is especially after visions which
imagination itself forms and to which it clings, whilst it is not under the
influence of exterior objects, that we are able to recognize the tendencies in
which it finds itself. It is for philosophy to teach us what care it is
necessary to give our imagination, and how we can preserve it from all error.
The best of all preparations is, to practice especially speculative virtue of
that kind which will make life a continual Intellectual progression. It is
necessary to as much as possible prevent the blind and disordered movements of
our imagination; in other words, lean towards the good and forsake the evil,
not mixing ourselves more with terrestrial things than the necessity requires.
There is nothing so efficacious as contemplation to disperse the enemies who
besiege the spirit. The spirit is refined by this more than we would think, and
turns towards God; then, suitably prepared, it attracts by a species of
affinity the divine spirit; it makes it enter into intercourse with the soul.
But when it is thickened, contracted, and dwarfed to the point of not being
able any more to fill the place destined for it by Providence when It formed
man, (I intend by that the habitation of the brain,) as nature abhors a vacuum,
it introduces into us an evil spirit. And what sufferings does this detestable
guest bring to us! Because, since these habitations have been made to receive
the spirit, nature desires that they should always be occupied by a spirit,
good or wicked. This last condition is the punishment of the impious who have
soiled that which they had in them which was divine; the other is even the end
or nearly the end of a pious life.


\section{In order to obtain possession of the knowledge of divination by
dreams, it is from the beginning necessary to be chaste and temperate.}

We have wished, in studying divination by dreams, to prove that this science is
not to be despised, but on the contrary merits study, so that we may obtain all
the advantages that can be drawn from it, and it is necessary to examine what
is the nature of imagination. But of what use this divination can be in
ordinary life we have not yet shown. The best profit that we are able to obtain
is to render the spirit healthy, and raise the soul; also it is religious
exercise which renders us apt at divination. Many in their desire to foresee
the future have renounced the excesses of the table so as to live sober and
temperate lives; they have kept their bed pure and chaste: for the man, who
desires to make his bed like the tripod of Delphi will watch himself well from
rendering it a witness of nocturnal debauches; he prostrates himself before God
to pray. Thus little by little he makes provision for admirable virtues; he
attains an aim more elevated than the purpose he desired, and without having
at first dreamed of it he comes to attach and unite himself to God.


\section{Divination through dreams is precious and easy.}

It is then necessary that we do not neglect divination; for it conducts us
towards the divine summits and puts in play the most precious of our faculties.

The intercourse of the soul with God does not render it less fit for the
affairs of this life; its noble aspirations do not make us forget the animal
existence. From an elevated position it sees more clearly all that which is
below it, than if it lived confined in that inferior region; without losing any
of its serenity, it gives to the animal part exact representations of all that
which is produced in this contingent world.

The proverb, ``descend without descending,'' is especially true of him who,
lowering his thoughts towards objects less dignified than himself, does not
keep them fixed there. This science of divination I desire to possess and
bequeath to my children. In order to acquire it there is no need to undertake
at great cost a painful journey, nor a long voyage, to go to Delphi or into the
desert of Ammon; it is sufficient to sleep after having made ablutions and a
prayer. Observe the Penelope of Homer:

\begin{verse}
``Going out of the pure water,\\
Covering her body with a veil of dazzling whiteness\\
She invoked Minerva.''\footnote{\textit{Odyssey} \textsc{xvii} 48.}
\end{verse}

\noindent We should do as she did so as to taste sleep. Are you in the right
condition?  God, who holds himself afar, comes to you. You have no need to give
yourself trouble: He presents himself always during your sleep.\footnote{The
\textit{Zohar} holds that whenever man is asleep his \textit{Neshamah},
\textit{i.~e.}~Intellectual Soul, returns to the higher place, the Garden of
Eden, from which it originally came down.} In sleep, the whole business of
initiation is performed. Never has a poor man been able to complain that his
poverty hindered him from being initiated as well as the rich. Some Hierophants
cannot be taken, as are the trierarchs of Athens, from amongst those who
possess great fortunes; because it is necessary to spend much in order to
obtain the herb of Crete, a bird of Egypt, a bone of Iberia, and and other
rarities of that kind which are only found in the depths of the earth and the
sea, on the shores ``Where the sun begins and finishes its
course.''\footnote{\textit{Odyssey} p.~24.}

External divination then demands costly preparations; and who is the individual
sufficiently wealthy to incur all these expenses? But if it be a question of
dreams, it matters little what income he possesses; he may be in a modest
condition or even till the ground to gain a living: boatmen, hirelings,
citizens, strangers, are all equal in this. God has not made any difference
between the first of the priests and the last of the slaves. Thanks to its
character divination by dreams is placed within the reach of all: plain and
without artifice, it is pre-eminently rational; holy, because it does not make
use of violent methods, it can be exercised anywhere; it dispenses with
fountain, rock, and gulf, and thus it is that which is truly divine. To
practice it there is no need of neglecting any of our occupations, or to rob
our business for a single moment, and that is the advantage I should have
described at first. No one is advised to quit his work and go to sleep,
especially to have dreams. But as the body cannot resist prolonged
night-watches, the time that nature has ordained for us to consecrate to repose
brings us, with sleep, an accessory more precious than sleep itself: that
natural necessity becomes a source of enjoyment and we do not sleep merely to
live, but to learn to live well. On the contrary, divination which is exercised
by the aid of material means takes the greatest part of our time, and it is a
happiness if it leaves us some hours of liberty for our necessities and
business. It is very rare that it is of any usefulness to us in the ordinary
affairs of life; because the circumstances, the places, do not lend themselves
to the accomplishment of the necessary ceremonies; and besides it is not easy
to carry with us everywhere an equipage of instruments. Indeed, without
speaking of the inconveniences, all this baggage, which lately the narrow walls
of a prison could not contain,\footnote{The Emperors, after they became
Christians, interdicted superstitious practices. Synesius is here speaking of
the seizure of the instruments which were used in these practices.} would be a
load for a wagon or a ship. Add again that these ceremonies have witnesses, who
are able to reveal them, as it has happened in our time: so also, obeying legal
prescriptions, many of the people have divulged these mysteries, and have
delivered them up to the gaze and ears of the profane multitude. Beyond that it
is humiliating to see the knowledge debased, that species of divination should
be held in abhorrence by God. Really not to await that of which we desire the
presence to come freely, but to press it, to harass it so as to draw it to us,
is violence, and is to commit a fault of the nature of those that even our
human laws do not leave unpunished. All this is grave; but it is not all: when
we employ, in order to perceive the future, artificial means, we run the risk
of being interrupted in our operations; and if we travel, leave our knowledge
in our house; for it is no little matter to pack up this thing and carry it
away. But in divination by dreams, each of us is in himself his proper
instrument; whatever we may do, we cannot separate ourselves from our oracle:
it dwells with us; it follows us everywhere, in our journeys, in war, in public
office, in agricultural pursuits, in commercial enterprises. The laws of a
jealous Republic do not interdict that divination: if they did they could do
nothing: because how can the offense be proven? What harm is there in sleeping?
No tyrant is able to carry out an edict against dreams, still less proscribe
sleep in his dominions; that would be at once fully to command the impossible,
and an impiety to put himself in opposition to the desires of nature and God.


\section{It brings to all the joys of hope.}

The let all of us deliver ourselves to the interpretation of dreams, men and
women, young and old, rich and poor, private citizens and magistrates,
inhabitants of the town and of the country, artisans and orators. There is not
any privileged, neither by sex, neither by age, nor fortune or profession.
Sleep offers itself to all: it is an oracle always ready, and an infallible and
silent counsellor; in these mysteries of a new species each is at the same time
priest and initiate. It, as well as divination, announces to us the joys to
come, and, through the anticipated happiness which it procures for us, it gives
to our pleasures a longer duration; and it warns us of the misfortunes that
threaten us, so that we may be put on our guard. The charming promises of hope
so dear to man, the foreseeing calculations of fear, all come to us through
dreams. Nothing is more qualified in its effect to nourish hope in us; this
good, so great and so precious that without it we could not be able, as said
the most illustrious Sophists, to support life; for who would desire to remain
always in the same condition? Surrounded by so much evil, man would soon allow
himself to be discouraged, if Prometheus had not put in man's heart the hope
which charms his pains, and gives him with forgetfulness of the present the
certainty of a better future. Such is the strength of illusion that the
prisoner, whose feet are held captive in the shackles, as soon as he lets his
thoughts wander, sees freedom; he is a soldier, he commands half a cohort; he
becomes centurion, general; he is victorious; offers sacrifices, and crowns
himself so as to celebrate his triumph; he gives feasts in which shine if he
chooses all the luxury of Sicily and Persia; he dreams no more of his irons,
all the time that it pleases him to be a general. These reveries come even in
our waking hours as in our sleep; but it is always the imagination which
precedes them. Imagination, when set in play by our will, renders us the unique
service of charming our existence, of offering to our soul the flattering
illusions of hope, and thus consoling us for our pains.


\section{Dreams are veracious but it is necessary to know how to comprehend
them.}

But when dreaming brings to us from itself hope, as it comes during sleep, then
we are able to consider God as the surety of the promises that dreams make to
us. In preparing one's self to receive the benefits announced in dreams, we
have a double happiness: at first because we enjoy in advance these benefits in
idea; afterwards when we possess them in reality, we know how to use them as we
ought; because we have seen the right employment that we should make of them.
Pindar, speaking of the happy man, celebrates hope. ``It is sweet,'' he says,
``it nourishes the heart; it accompanies and animates youth, it is it
especially which governs the variable spirits of
mortals.''\footnote{Fragments.} Without doubt there cannot be a question of
that deceiving hope which we fabricate in ourselves when entirely awake. But
all that Pindar says, is only a feeble part of the praise that we can render to
dreams. Divination by dreams is a science which pursues the exact truth, and
which inspires such confidence that we should not relegate it to an inferior
rank. If the Penelope of Homer tells us that two different gates allow the
passage of dreams, and that one permits the escape of deceiving
dreams,\footnote{``There are two portals of unsubstantial dreams: one is made
of horn, one of ivory; whichever come through the sawn ivory deceive and bring
promises which will never be fulfilled; but those which come out of the doors
of the polished horn bring a true issue when any one of mortals sees them.''
\textit{Odyssey} \textsc{xix} 562.} it is because she lacks a correct knowledge
of the nature of dreams: better instructed she would have made them all go out
of the door of horn. She is convicted of error and ignorance, when she refuses
to believe a vision which ought nevertheless to inspire her with confidence.
``The geese are the wooers, and the eagle that was is
Ulysses.''\footnote{\textit{Odyssey} \textsc{xix} 548.}

Ulysses was near her, and it is to him that she was speaking of the falseness
of his dream. Homer evidently desired to show by this, that we must not
challenge dreams, and that, if we do not deceive ourselves in our dreams, the
dream itself is not deceptive. Agamemnon also was wrong in believing that a
dream was false: he did not understand the prophesy that foretold victory for
him:

\begin{verse}
``Order all the Greeks to put on their arms,\\
And the walls of Ilion will fall before thee.''\footnote{\textit{Iliad}
\textsc{ii} 11}.
\end{verse}

\noindent He then marched, supposing that the city would fall at the first
assault; but the prophesy said that it was necessary that \textit{all} the
Greeks should arm themselves. Now Achilles and the troop of Myrmidons, the very
flower of the army, refused to take part in the combat.


\section{The obligations of Synesius to dreams.}

But dreams have been sufficiently eulogized; let us stop. But I must not be
ungrateful. I have already shown that, traveling the seas or resting at
firesides, be you merchants or soldiers, always and everywhere we carry with us
the faculty of foreseeing the future. But I have not yet stated my own
indebtedness to dreams. And yet it is to the minds given to philosophy that
dreams especially come, to enlighten them in their difficulties and researches,
so as to bring them during sleep the solutions which escape them when awake. We
seem in sleeping at one time to apprehend, at another to find, through our own
reflection. As for me, how often dreams have come to my assistance in the
composition of my writings! Often have they aided me to put my ideas in order,
and my style in harmony with my ideas; they have made me expunge certain
expressions, and choose others. When I allowed myself to use images and pompous
expressions, in imitation of the new Attic style, so far removed from the old,
a god warned me in my sleep, censured my writings, and making the affected
phrases disappear, brought me back to a natural style. At other times, in the
hunting season, I invented, after a dream, traps to catch the swiftest animals
and the most skillful in hiding. If, discouraged from too long waiting, I was
preparing to return to my home, dreams would give me courage, by announcing to
me, for such or such a day, a better result: I think patiently watched some
nights more; many animals would fall in my nets or under my arrows. All my life
has been spent among books or in hunting, except the time of my embassy: and
would to the gods I had never lived those three cursed years! But then again
divination has been singularly useful to me: it preserved me from ambushes that
certain magicians laid for me, revealed their sorceries and saved me from all
danger; it sustained me during the whole duration of the missing which has
caused to prosper the greatest good in the cities of Libya; it conducted me
even before the Emperor, in the midst of his court, in which I have spoken with
an independence, of which no Greek ever before had given an example.


\section{Wherefore dreams are seldom lucid, and wherefore art is needed to
explain them.}

Each kind of divination has its particular adepts; but divination by dreams
addresses itself to all. It offers itself to each of us as a propitious
divinity; it adds new conceptions to those which we have found in our waking
meditations. Nothing is wiser than a soul disengaged from the tumult of the
senses, which only bring to it from without troubles without end. The ideas
that it possesses, and, when it is wrapped in itself, those that it receives
from intelligence, it communicates to those who are turned towards the interior
life; it makes all that which is from God enter into them; because between that
soul and the divinity which animates the world there exist intimate affinities,
because both come from the same source. Dreams then have nothing earthly; they
are clear and give a perfect or nearly perfect evidence; there is no need of an
interpreter. But this happiness is reserved only to those who live in the
practice of virtue, acquired by an effort of reason or by habit. It is very
rarely that other men have such lucid dreams; sometimes this happens, but only
in very grave conjunctures. At other times their dreams are vulgar and confused
and full of obscurity; it is necessary to have the aid of art in order to
explain them. As their origin is, so to say, strange and fantastic, by virtue
of that origin they only offer uncertainty.


\section{All things past, present, and future convey to us images which are
reflected in our imagination.}

All things which exist in nature, which have existed, which will exist (because
the future is yet a mode of existence), send out images which escape from their
substance. Perceptible objects are composed of form and matter:\footnote{This
idea is fully set forth in the works of Solomon ibn Gabirol or Avicebron,
especially in his \textit{Fons Vit{\ae}} (``Source of Life''). It is a doctrine
also set forth with great positiveness in the \textit{Zohar} and the Hebrew
Kabbalah.} now, as we see that matter is a perpetual condition of motion, and
that the images which it produces are borne away by it, we are forced to admit
this; thus images and matter, all that which falls into generation, does not
approach in dignity permanent existence (\textit{real} being). All these
fugitive images reflect themselves in the imagination as in a brilliant mirror.
Wandering at random and detached from the objects in which they have taken
existence, as they have only an undecided existence, and as none of the beings
who exist by themselves will receive them, when they meet animal spirits who
themselves are also images\footnote{\textit{Eidolois.} This word has the double
meaning of images and phantoms.} but from images residing in us, they penetrate
into these spirits, they establish themselves there as it a dwelling. Things
passed, since they have been realities, give clear images, which finish at
length by effacing themselves and disappearing; present things, as they
continue to exist, form images still more clear and living; but the future
gives us nothing except vagueness and indistinctness: so from the buds, which
have just made their appearance, we surmise the flowers and leaves, as yet
badly formed, which they contain and which will open and burst out in a short
time. Thus art is indispensable in order to know the future; we can only have
an uncertain sketch of that which is to come; we have only in exact
representation that which is.


\section{It is necessary by means of philosophy to keep our imagination free
from the passions.}

But is it not astonishing that it can itself produce images of that which will
be only later? It is here that I ought to speak of how we can acquire that art
of divination. That which is first necessary is that the divine spirit which is
in us be sufficiently prepared, so as to be visited by intelligence and by God,
and not be the receptacle of vain images. Now, when this (the latter) happens,
we should take refuge in philosophy, whose beneficial action appeases the
passions which besiege the spirit and invade it so as to make it their
dwelling. Foster in your life habits of temperance and frugality, so as not to
agitate the animal part of your existence: the troubles of the senses extend
even to the imagination, which must be kept quiet and tranquil. That calmness
is very easy to be desired but very difficult to obtain. For myself, as I wish
that sleep be not useless to any one, I will try and discover a fixed rule
which is applicable to the infinite variety of dreams; in other words, it is my
object to establish a science of the nocturnal appearances. Here is how we can
undertake it.


\section{How we can undertake the interpretation of dreams.}

The navigator who, after having passed a rock, perceives a city, knows
afterwards, when he sees the same rock that the same city will soon be in
sight. We have no need to see a general in order to know that he is coming, his
approach becomes known to us by the escort which precedes him: because each
time it has appeared, it was because the general was coming. So images which
present themselves to our spirit are \textit{indicia} of the future; the
return of the same signs predict the return of the same events. He is a stupid
pilot who repasses near the same rock without recognizing it, and who cannot
tell what shore he is near; he navigates at random. So the man who has dreamed
the same dream several times, and who has not observed what the dream
predicts---accident, happiness, undertaking,---he directs his life as that pilot
directs his vessel, without reflection. We prognosticate storms even when
everything in the atmosphere is tranquil, if we observe circles around the
moon; because we have often noticed that this phenomenon is frequently followed
by a storm:

\begin{verse}
A single circle, fading, denotes fine weather;\\
If it is broken, it certainly announces wind;\\
If it is double, believe me, a tempest is near;\\
But if it is triple, and dark, and broken, I expect\\
Then more than ever, the fury of storms.\footnote{Aratus,
\textit{Ph{\ae}nomena} 811.}
\end{verse}

\noindent Thus always, as Aristotle has said\footnote{\textit{Metaphysics}
\textsc{i} 1.} with reason, perception precedes memory, from memory comes
experience and from experience, knowledge. It is by this means that we come to
the interpretation of dreams.


\section{On account of the diversity of minds there is no general rule for the
explanation of dreams.}

There are men who use many books in which are set forth the rules of this art.
Personally I ignore these books, and regard them as useless. For though the
last or lowest body, which is a composition of different elements, cannot by
reason of its nature be an object of a knowledge which is one and positive,
since the affections which it experiences are produced nearly always alike, and
through the same causes, (because the elements constituting it differ very
little from each other, and the difficulties which trouble the organism cannot
remain concealed,) it is not thus with imagination. Here it is entirely a
different thing: between different spirits there exist great differences,
according as they are connected with the spheres or reign over matter.

\begin{verse}
Happy in this world, among all the souls,\\
Is the soul which has descended from the ethereal heights,\\
The soul also, which knew Jupiter's Court,\\
And which living here below contains its destiny,\\
Even in this exile remains nevertheless happy.
\end{verse}

\noindent It was this which Tim{\ae}us signified, when he assigned a star to
each soul. But souls have lapsed: inflamed through material desires they have
fallen more or less below, and in their fall the imagination has been defiled.
So sunk they inhabit bodies: life is now one long discord; the spirit is sick:
an unnatural condition if we consider its noble origin, but natural as to the
animal existence with which it is connected, and which it animates. Perhaps
however the nature of the spirit depends entirely on the rank which it
occupies, according to its practice of vice or virtue. For there is nothing so
variable as the spirit. How, with dissimilar natures obeying different laws and
passions, are there the same apparitions? That is not so; it cannot be. Water,
muddy or clear, tranquil or agitated, can it equally reflect objects? Vary its
tints, move it in different ways, the figures will change in appearance; they
will only have one characteristic in common, that of deviating from the true.
If this is contested,---if some Phemon{\oe}, some Melampus, or other diviner
pretends to establish, for the explanation of dreams, a general rule, we would
ask him if plane or convex mirrors, or those made of different materials,
reflect similar images. But never, I think, have these people considered the
nature of spirit. As the imagination is akin to the spirit in a certain
respect, they apprehend that there is one rule and canon for the interpretation
of all things. I do not claim that between things most dissimilar there is
absolutely no relation; but this relation is obscure, and becomes more obscure
if it is unduly extended. Add, as I have said, that it is difficult to have a
clear image of future things before they come into actual existence. Finally,
as we all have our idiosyncrasies, it is not possible that the same visions
should have the same signification for all.


\section{Each ought to make his divinatory knowledge for himself, by noting his
dreams.}

We must not hope then to establish general rules: each one must search for his
knowledge within himself. We should inscribe in our memory all that has come to
us in our dreams. It is easy to do that which is entirely profitable; the
profit which it brings is a stimulant, especially when we have that which we
exercise. What is more usual than dreams? What exercises a stronger influence
on the mind? Such an influence, indeed, that even the dullest give attention to
their dreams. It is a disgrace, at twenty-five years of age, to still need an
interpreter for one's dreams, and not to possess the principles of this art.
For the memories which should have carefully kept the visions of our sleep as
well as the events which happen when we are awake certainly have their value.
It is a novelty which will perhaps shock received ideas: but nevertheless,
wherefore should we not complete the history of our days with that of our
nights, and so retain a remembrance of our dual lives? There is a life of the
imagination, as we have demonstrated, sometimes better, sometimes worse, than
ordinary life, according as the spirit is healthy or sick. If then we are
careful in noting our dreams, while thus acquiring the knowledge of divination,
we will not let anything escape our memory, and we will have pleasure in
composing this biography, which will give our history both waking and sleeping.
Moreover, if we desire to become rhetoricians, we can find no better method for
the development of the oral faculties. When we commit our daily impressions to
writing, as we neglect no details, and note little things as well as great, we
habituate ourselves, says the sophist of Lemnos,\footnote{Philostratus.} to
successfully treat all subjects. But what an admirable theme does the history
of our nightly visions furnish to the orator!


\section{Dreams bring to the mind all kinds of images and impressions.}

It is not an easy thing to set forth exactly all the circumstances of a dream,
to separate that which is found united by nature and to unite that which is
separated, and give to others, by our descriptions, dreams which they have not
had. It is no easy work to make our own impressions pass into the soul of
another. Imagination relegates into nothingness beings which exist; it causes
to proceed from nothing beings which do not exist, which cannot exist. How, at
a time when we have no idea of anything similar, can we represent objects which
it is impossible for us even to name? Imagination assembles many images at the
same time, and presents them at the same instant, but confused, such as the
dream gives them; for it is according to the dream that our visions are
produced. In order to faithfully render these various impressions all the
resources of language are necessary. Imagination acts upon our affections more
than one would think: dreams excite different emotions in us; we at one time
experience sentiments of sympathy and attachment, at another aversion. It is
also during sleep that the enchantments of magic exercise themselves upon us,
and that we are especially subject to voluptuousness; love and hate penetrate
into our souls, and persist in remaining even after our awaking.


\section{Various marvels are presented to us by dreams.}

If we would communicate to our hearers our impressions and ideas, a lively and
forcible language is essential. In dreams, one is a conqueror, we walk, we fly.
Imagination lends itself to all; have words the same facilities? Sometimes we
dream that we sleep, that we are dreaming, that we arise, that we shake off
sleep, and yet we are asleep; we reflect on the dream we have just had; even
that is still a dream, a double dream; we think no more of recent chim{\ae}ras;
we imagine ourselves now awake, and we regard the present visions as if
realities. Thus is produces in our mind a veritable combat; we think that we
make an effort for ourselves, that we have driven away the dream, that we are
no longer asleep, that we have taken the full possession of our being, and that
we have ceased to be the dupe of an illusion. The Aloid{\ae}, for attempting to
climb to heaven, by heaping one upon another the mountains of Thessaly, were
punished; but what law forbids a sleeper from rising above the earth upon wings
surer than those of Icarus, from excelling the flight of eagles, from soaring
above the celestial spheres? We perceive the earth from afar, we discover a
world which even the moon does not see. We can talk with the stars, mingle with
the invisible company of the gods who rule the universe. These marvels which
cannot be readily described, are nevertheless accomplished without the least
effort. We enjoy the presence of the gods without being exposed to jealousy.
Without having the trouble of redescending, we find ourselves upon the earth;
for one of the privileges of our dreams is the suppression of time and space.
Then we talk with sheep: their bleating becomes a clear and distinct language.
Is there not therefore a vast field opened to an eloquence of a new kind? From
whence came the apologue which makes the peacock, the fox, and even the sea
speak? The audacities of imagination are insignificant when compared with the
temerity of dreams; but, although the apologue is only a very feeble
reproduction of some of our dreams, it furnishes nevertheless ample material
for oratorical talent. Why should we not exercise ourselves in interpreting
dreams? By this one not only trains himself in the art of oratory, but also
gains wisdom.


\section{It is much more useful to take our dreams for the texts of our
literary exercises, that the ridiculous subjects chosen by many of the
rhetoricians.}

Let us then employ our leisure in telling the events which happen to us whilst
awake or during sleep; consecrate to this work a portion of your time and from
it you will derive, as I have shown, inestimable advantages. You will acquire
the science of divination which we have eulogized, and above which we cannot
place anything: elegance of diction, something not to be despised, will
likewise come to you. In this kind of work the philosopher unbends his mind as
the Scythian unbends his bow. Dreams also furnish to the rhetoricians admirable
texts for their showy discourses. I can scarcely comprehend what interest they
find in celebrating the virtues of Miltiades, of Cimon, or even some anonymous
person: in making the rich speak and the poor struggle with each other about
public affairs. I have nevertheless seen old men wrangling on these subjects at
the theater; and such old men! They made a show of philosophical gravity, and
wore a beard which might well, I imagine, weigh several pounds. But their
gravity did not hinder them from insulting each other, from getting the better
of each other, from supporting, with extravagant gestures, their long
discourses. It seemed to me that they were pleading the cause of some parent:
but what a surprise when I afterwards learned that the persons whom they were
defending, far from being of their family, did not even exist, had never
existed, and could not exist! Is there a republic which, to recompense the
services of a citizen, permits him to kill his enemy?\footnote{A wealthy man
and a poor man are enemies: the rich man promises to furnish food for the
people, if they will authorize him to kill the poor man; this permission is
granted him. But the rich man does not feed the son of the poor man, who dies
of hunger.---This is the subject to which Synesius alludes.} When at the age of
ninety years they are still disputing on such pitiable inventions, at what time
of life will they take up serious work and discourses? Do these people then not
know the meaning of words? They are ignorant that declamation involves
preparatory exercises; they take the means for the end, the road for the goal
which it is necessary to attain. They make even the preparation the sole object
of all their efforts. To bend the arms in wrestling exercises, is that
sufficient in order to be proclaimed conqueror in the Olympic games? Scarcity
of thoughts but abundance of words is what characterizes these people---always
ready to speak, even when they have nothing to say. Why not profit by the
example of Alc{\ae}us and Archilochus, who narrated their own lives? In this
way the memory of those things which happened to them---whether pleasant or
painful---was preserved for posterity. Neither did they record vain and
unprofitable things, as the new race of wits who practice themselves on
imaginary subjects. Neither have these wits consecrated their genius to the
glory of others, like Homer and Stesichorus, who have added by their poems to
the celebrity of heroes, and who excite our souls to virtue, entirely in
forgetfulness of themselves. All we know of them is that they were excellent
poets. So then if you wish to make a name for yourselves with posterity, if you
feel yourself capable of bringing forth a work which may live forever, do not
hesitate to enter into the entirely new path which I recommend to you. Count on
the future: the future faithfully guards that which, with the aid of God, we
confide to it.


\end{document}
