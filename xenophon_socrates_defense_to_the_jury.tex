% pdflatex
\documentclass[12pt]{article}
\usepackage{ebgaramond}
\usepackage{microtype}

\title{Socrates' Defense to the Jury}
\author{Xenophon \and Otis Johnson Todd (tr.)\footnote{E.~C.~Marchant and
O.~J.~Todd. \textit{Xenophon,} vol.~4. Harvard University Press, 1923.}}
\date{}

\begin{document}
\maketitle

\noindent It seems to me fitting to hand down to memory, furthermore, how
Socrates, on being indicted, deliberated on his defence and on his end. It is
true that others have written about this, and that all of them have reproduced
the loftiness of his words,---a fact which proves that his utterance really was
of the character intimated;---but they have not shown clearly that he had now
come to the conclusion that for him death was more to be desired than life; and
hence his lofty utterance appears rather ill-considered. Hermogenes, the son of
Hipponicus, however, was a companion of his and has given us reports of such a
nature as to show that the sublimity of his speech was appropriate to the
resolve he had made. For he stated that on seeing Socrates discussing any and
every subject rather than the trial, he had said: ``Socrates, ought you not to
be giving some thought to what defence you are going to make?'' That Socrates
had at first replied, ``Why, do I not seem to you to have spent my whole life
in preparing to defend myself?'' Then when he asked, ``How so?'' he had said,
``Because all my life I have been guiltless of wrong-doing; and that I consider
the finest preparation for a defence.'' Then when Hermogenes again asked, ``Do
you not observe that the Athenian courts have often been carried away by an
eloquent speech and have condemned innocent men to death, and often on the
other hand the guilty have been acquitted either because their plea aroused
compassion or because their speech was witty?'' ``Yes, indeed!'' he had
answered; ``and I have tried twice already to meditate on my defence, but my
divine sign interposes.'' And when Hermogenes observed, ``That is a surprising
statement,'' he had replied, ``Do you think it surprising that even God holds
it better for me to die now? Do you not know that I would refuse to concede
that any man has lived a better life than I have up to now? For I have realized
that my whole life has been spent in righteousness toward God and man,---a fact
that affords the greatest satisfaction; and so I have felt a deep self-respect
and have discovered that my associates hold corresponding sentiments toward me.
But now, if my years are prolonged, I know that the frailties of old age will
inevitably be realized,---that my vision must be less perfect and my hearing
less keen, that I shall be slower to learn and more forgetful of what I have
learned. If I perceive my decay and take to complaining, how,'' he had
continued, ``could I any longer take pleasure in life? Perhaps,'' he added,
``God in his kindness is taking my part and securing me the opportunity of
ending my life not only in season but also in the way that is easiest. For if I
am condemned now, it will clearly be my privilege to suffer a death that is
adjudged by those who have superintended this matter to be not only the easiest
but also the least irksome to one's friends and one that implants in them the
deepest feeling of loss for the dead. For when a person leaves behind in the
hearts of his companions no remembrance to cause a blush or a pang, but
dissolution comes while he still possesses a sound body and a spirit capable of
showing kindliness, how could such a one fail to be sorely missed? It was with
good reason,'' Socrates had continued, ``that the gods opposed my studying up
my speech at the time when we held that by fair means or foul we must find some
plea that would effect my acquittal. For if I had achieved this end, it is
clear that instead of now passing out of life, I should merely have provided
for dying in the throes of illness or vexed by old age, the sink into which all
distresses flow, unrelieved by any joy. As Heaven is my witness, Hermogenes,''
he had gone on, ``I shall never court that fate; but if I am going to offend
the jury by declaring all the blessings that I feel gods and men have bestowed
on me, as well as my personal opinion of myself, I shall prefer death to
begging meanly for longer life and thus gaining a life far less worthy in
exchange for death.''

Hermogenes stated that with this resolve Socrates came before the jury after
his adversaries had charged him with not believing in the gods worshipped by
the state and with the introduction of new deities in their stead and with
corruption of the young, and replied: ``One thing that I marvel at in Meletus,
gentlemen, is what may be the basis of his assertion that I do not believe in
the gods worshipped by the state; for all who have happened to be near at the
time, as well as Meletus himself,---if he so desired,---have seen me
sacrificing at the communal festivals and on the public altars. As for
introducing `new divinities,' how could I be guilty of that merely in
asserting that a voice of God is made manifest to me indicating my duty? Surely
those who take their omens from the cries of birds and the utterances of men
form their judgments on `voices.' Will any one dispute either that thunder
utters its `voice,' or that it is an omen of the greatest moment? Does not the
very priestess who sits on the tripod at Delphi divulge the god's will through
a `voice?' But more than that, in regard to God's foreknowledge of the future
and his forewarning thereof to whomsoever he will, these are the same terms,
assert, that all men use, and this is their belief. The only difference between
them and me is that whereas they call the sources of their forewarning
`birds,' `utterances,' `chance meetings,' `prophets,' I call mine a `divine'
thing;\footnote{Or ``divine sign.'' Here, as earlier, the mere adjective is
used; but in Plato's \textit{Theages} (128\textsc{d}~ff.) and \textit{Apology}
(31\textsc{d}) this admonitory something is described as a voice sent by
heavenly dispensation, and is called variously ``the sign'' (\textit{Apology}
41\textsc{d}), ``the usual sign'' (\textit{Apology} 40\textsc{c}), ``the divine
sign'' (\textit{Rep.}~496\textsc{c}), ``the usual divine sign''
(\textit{Euthyd.}~272\textsc{e}, \textit{Ph{\ae}drus} 242\textsc{b},
\textit{Theages} 129 \textsc{b}), ``the sign from God'' (\textit{Apology}
40\textsc{b}), ``something God-sent and divine'' (\textit{Apology}
31\textsc{d}). Plato reports Socrates' description of this as a voice not
directing his actions but serving only as a deterrent when he or his friends
were contemplating doing something inadvisable.} and I think that in using such
a term I am speaking with more truth and deeper religious feeling than do those
who ascribe the gods' power to birds. Now that I do not lie against God I have
the following proof: I have revealed to many of my friends the counsels which
God has given me, and in no instance has the event shown that I was mistaken.''

Hermogenes further reported that when the jurors raised a clamor at hearing
these words, some of them disbelieving his statements, others showing jealousy
at his receiving greater favors even from the gods than they, Socrates resumed:
``Hark ye; let me tell you something more, so that those of you who feel so
inclined may have still greater disbelief in my being honored of Heaven. Once
on a time when Ch{\ae}rephon\footnote{A very enthusiastic follower of
Socrates.} made inquiry at the Delphic oracle concerning me, in the presence of
many people Apollo answered that no man was more free than I, or more just, or
more prudent.''

When the jurors, naturally enough, made a still greater tumult on hearing this
statement, he said that Socrates again went on: ``And yet, gentlemen, the god
uttered in oracles greater things of Lycurgus, the Laced{\ae}monian law-giver,
than he did of me. For there is a legend that, as Lycurgus entered the temple,
the god thus addressed him: `I am pondering whether to call you god or man.'
Now Apollo did not compare me to a god; he did, however, judge that I far
excelled the rest of mankind. However, do not believe the god even in this
without due grounds, but examine the god's utterance in detail. First, who is
there in your knowledge that is less a slave to his bodily appetites than I am?
Who in the world more free,---for I accept neither gifts nor pay from any one?
Whom would you with reason regard as more just than the one so reconciled to
his present possessions as to want nothing beside that belongs to another? And
would not a person with good reason call me a wise man, who from the time when
I began to understand spoken words have never left off seeking after and
learning every good thing that I could? And that my labor has not been in vain
do you not think is attested by this fact, that many of my fellow-citizens who
strive for virtue and many from abroad choose to associate with me above all
other men? And what shall we say is accountable for this fact, that although
everybody knows that it is quite impossible for me to repay with money, many
people are eager to make me some gift? Or for this, that no demands are made on
me by a single person for the repayment of benefits, while many confess that
they owe me a debt of gratitude? Or for this, that during the
siege,\footnote{The blockade of Athens by the Spartans in the last year of the
Peloponnesian War.} while others were commiserating their lot, I got along
without feeling the pinch of poverty any worse than when the city's prosperity
was at its height? Or for this, that while other men get their delicacies in
the markets and pay a high price for them, I devise more pleasurable ones from
the resources of my soul, with no expenditure of money? And now, if no one can
convict me of misstatement in all that I have said of myself, do I not
unquestionably merit praise from both gods and men? But in spite of all,
Meletus, do you maintain that I corrupt the young by such practices? And yet
surely we know what kinds of corruption affect the young; so you tell us
whether you know of any one who under my influence has fallen from piety into
impiety, or from sober into wanton conduct, or from moderation in living into
extravagance, or from temperate drinking into sottishness, or from
strenuousness into effeminacy, or has been overcome of any other base
pleasure.'' ``But, by Heaven!'' said Meletus: ``there is one set of men I
know,---those whom you have persuaded to obey you rather than their parents.''
``I admit it,'' he reports Socrates as replying, ``at least so far as education
is concerned; for people know that I have taken an interest in that. But in a
question of health, men take the advice of physicians rather than that of their
parents; and moreover, in the meetings of the legislative assembly all the
people of Athens, without question, follow the advice of those whose words are
wisest rather than that of their own relatives. Do you not also elect for your
generals, in preference to fathers and brothers,---yes, by Heaven! in
preference to your very selves,---those whom you regard as having the greatest
wisdom in military affairs?'' ``Yes,'' Meletus had said; ``for that is both
expedient and conventional.'' ``Well, then,'' Socrates had rejoined, ``does it
not seem to you an amazing thing that while in other activities those who excel
receive honors not merely on a parity with their fellows but even more marked
ones, yet I, because I am adjudged by some people supreme in what is man's
greatest blessing,---education,---am being prosecuted by you on a capital
charge?''

More than this of course was said both by Socrates himself and by the friends
who joined in his defence. But I have not made it a point to report the whole
trial; rather I am satisfied to make it clear that while Socrates' whole
concern was to keep free from any act of impiety toward the gods or any
appearance of wrong-doing toward man, he did not think it meet to beseech the
jury to let him escape death; instead, he believed that the time had now come
for him to die. This conviction of his became more evident than ever after the
adverse issue of the trial. For, first of all, when he was bidden to name his
penalty, he refused personally and forbade his friends to name one, but said
that naming the penalty in itself implied an acknowledgment of guilt. Then,
when his companions wished to remove him clandestinely from prison, he would
not accompany them, but seemed actually to banter them, asking them whether
they knew of any spot outside of Attica that was inaccessible to death.

When the trial was over, Socrates (according to Hermogenes) remarked: ``Well,
gentlemen, those who instructed the witnesses that they must bear false witness
against me, perjuring themselves to do so, and those who were won over to do
this must feel in their hearts a guilty consciousness of great impiety and
iniquity; but as for me, why should my spirit be any less exalted now than
before my condemnation, since I have not been proved guilty of having done any
of the acts mentioned in the indictment? For it has not been shown that I have
sacrificed to new deities in the stead of Zeus and Hera and the gods of their
company, or that I have invoked in oaths or mentioned other gods. And how could
I be corrupting the young by habituating them to fortitude and frugality? Now
of all the acts for which the laws have prescribed the death-penalty---temple
robbery, burglary, enslavement, treason to the state---not even my adversaries
themselves charge me with having committed any of these. And so it seems
astonishing to me how you could ever have been convinced that I had committed
an act meriting death. But further, my spirit need not be less exalted because
I am to be executed unjustly; for the ignominy of that attaches not to me but
to those who condemned me. And I get comfort from the case of
Palamedes,\footnote{One of the Greek warriors at Troy; put to death on a charge
of treason trumped up by Odysseus; or by Odysseus, Diomedes, and Agamemnon.}
also, who died in circumstances similar to mine; for even yet he affords us far
more noble themes for song than does Odysseus, the man who unjustly put him to
death. And I know that time to come as well as time past will attest that I,
too, far from ever doing any man a wrong or rendering him more wicked, have
rather profited those who conversed with me by teaching them, without reward,
every good thing that lay in my power.''

With these words he departed, blithe in glance, in mien, in gait, as comported
well indeed with the words he had just uttered. When he noticed that those who
accompanied him were in tears, ``What is this?'' Hermogenes reports him as
asking. ``Are you just now beginning to weep? Have you not known all along that
from the moment of my birth nature had condemned me to death? Verily, if I am
being destroyed before my time while blessings are still pouring in upon me,
clearly that should bring grief to me and to my well-wishers; but if I am
ending my life when only troubles are in view, my own opinion is that you ought
all to feel cheered, in the assurance that my state is happy.''

A man named Apollodorus, who was there with him, a very ardent disciple of
Socrates, but otherwise simple, exclaimed, ``But, Socrates, what I find it
hardest to bear is that I see you being put to death unjustly!'' The other,
stroking Apollodorus' head, is said to have replied, ``My beloved Apollodorus,
was it your preference to see me put to death justly?'' and smiled as he asked
the question.

It is said also that he remarked as he saw Anytus\footnote{One of the three
plaintiffs in Socrates' trial.} passing by: ``There goes a man who is filled
with pride at the thought that he has accomplished some great and noble end in
putting me to death, because, seeing him honored by the state with the highest
offices, I said that he ought not to confine his son's education to
hides.\footnote{The tanning trade had been in the family from at least the time
of the boy's grandfather.} What a vicious fellow,'' he continued, ``not to
know, apparently, that whichever one of us has wrought the more beneficial and
noble deeds for all time, he is the real victor. But,'' he is reported to have
added, ``Homer has attributed to some of his heroes at the moment of
dissolution the power to foresee the future; and so I too wish to utter a
prophecy. At one time I had a brief association with the son of Anytus, and I
thought him not lacking in firmness of spirit; and so I predict that he will
not continue in the servile occupation that his father has provided for him;
but through want of a worthy adviser he will fall into some disgraceful
propensity and will surely go far in the career of vice.'' In saying this he
was not mistaken; the young man, delighting in wine, never left off drinking
night or day, and at last turned out worth nothing to his city, his friends, or
himself. So Anytus, even though dead, still enjoys an evil repute for his son's
mischievous education and for his own hard-heartedness. And as for Socrates, by
exalting himself before the court, he brought ill-will upon himself and made
his conviction by the jury all the more certain. Now to me he seems to have met
a fate that the gods love; for he escaped the hardest part of life and met the
easiest sort of death. And he displayed the stalwart nature of his heart; for
having once decided that to die was better for him than to live longer, he did
not weaken in the presence of death (just as he had never set his face against
any other thing, either, that was for his good), but was cheerful not only in
the expectation of death but in meeting it.

And so, in contemplating the man's wisdom and nobility of character, I find it
beyond my power to forget him or, in remembering him, to refrain from praising
him. And if among those who make virtue their aim any one has ever been brought
into contact with a person more helpful than Socrates, I count that man worthy
to be called most blessed.

\end{document}
