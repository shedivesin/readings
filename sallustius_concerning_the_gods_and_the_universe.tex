\documentclass[12pt]{article}
\usepackage{ebgaramond}
\usepackage{ebgaramond-maths}
\usepackage{microtype}

\title{Concerning the Gods and the Universe}
\author{Sallustius \and Arthur Darby Nock (tr.)\footnote{Nock, A.~D. \textit{Sallustius Concerning the Gods and the Universe.} Cambridge University Press, 1926.}}
\date{}

\begin{document}
\maketitle

\paragraph{I} Those who would learn about the gods need to have been well
educated from childhood and must not be bred up among foolish ideas; they must
also be good and intelligent by nature, in order that they may have something
in common with the subject. Further, they must be acquainted with universal
opinions, by which I mean those in which all men, if rightly questioned, would
concur; such opinions are that every god is good and impassive and unchangeable
(since whatever changes, changes for better or for worse; if for worse, it
becomes bad, if for the better, it proves to have been bad in the first place).

\paragraph{II} Such must be the learner, and his instruction should be as
follows. The essences of the gods never came into being, for whatever always
exists never comes into being, and all things that have first power and are by
nature impassive do exist always; they are not formed of bodies, for even of
bodies the powers are bodiless; they are not limited by space, for that
certainly is an attribute of bodies; and they are never separated from the
First Cause or from one another, any more than are thoughts from the mind,
sciences from the soul, or the senses from a living creature.

\paragraph{III} It is worth our while to enquire why the ancients left the
statement of these truths and employed myths, and so to obtain this first
benefit from the myths, that we enquire and do not keep our intellects in
idleness. Consideration of those who have employed myths justifies us in saying
that myths are divine; for indeed the inspired among poets, and the best of
philosophers, and the founders of solemn rites, and the gods themselves in
oracles, have employed myths. Why myths are divine is a question belonging to
philosophy. Since all things in existence rejoice in likeness and turn from
unlikeness, it follows that our statements about the gods ought to be like the
gods, in order that being worthy of their true nature they may find favor for
their narrators (and such favor can by myths alone be won). So the myths
represent the gods in respect of that which is speakable and that which is
unspeakable, of that which is obscure and that which is manifest, of that which
is clear and that which is hidden, and represent the goodness of the gods; just
as the gods have given to all alike the benefits to be drawn from objects
perceptible to the senses while restricting to the wise the enjoyment of those
received from objects perceptible to the intellect, so the myths proclaim to
all that the gods exist, telling who they are and of what sort to those able to
know it. Again, myths represent the active operations of the gods. The universe
itself can be called a myth, since bodies and material objects are apparent in
it, while souls and intellects are concealed. Furthermore, to wish to teach all
men the truth about the gods causes the foolish to despise, because they cannot
learn, and the good to be slothful, whereas to conceal the truth by myths
prevents the former from despising philosophy and compels the latter to study
it. Why, however, have the ancients told in their myths of adulteries and
thefts and binding of fathers and other strange things? Is this also admirable,
meant to teach the soul by the seeming strangeness at once to think the words a
veil and the truth a mystery?

\paragraph{IV} Of myths some are theological, some physical; there are also
psychical myths and material myths and myths blended from these elements.
Theological myths are those which do not attach themselves to any material
objects but regard the actual natures of the gods. Such is the tale that Kronos
swallowed his children; since the god is intellectual, and all intellect is
directed towards itself, the myth hints at the god's essential nature. Again,
it is possible to regard myths in a physical way when one describes the
activities of the gods in the universe; so some before now have thought Kronos
to be Chronos or Time, and calling the parts of Time children of the whole say
that the father swallows his children. The psychical interpretation lies in
considering the activities of the soul itself: the thoughts of our souls, even
if they go forth to others, still remain in their creators. The worst
explanation, the material, is that which the Egyptians because of their
ignorance used most; they regarded and described material things as goods,
earth as Isis, moisture as Osiris, heat as Typhon, or water as Kronos, the
fruits of the soil as Adonis, wine as Dionysos.\footnote{This sentence, in
which Greek gods are named after Egyptian deities, apparently as in the same
category, is clumsy, but the clumsiness may well be due to the author.} To say
that these things, as also plants and stones and animals, are sacred to the
gods, is the part of reasonable men, to call them gods is the part of madmen,
unless by a common figure of speech, as we call the sphere of the sun and the
ray coming from that sphere the sun. The blended kind of myths can be seen in
numerous examples; one is the tale they tell that at the banquet of the gods
Strife threw a golden apple and the goddesses, vying with one another for its
possession, were sent by Zeus to Paris to be judged; Paris thought Aphrodite
beautiful, and gave her the apple. Here the banquet signifies the supramundane
powers of the gods, and that is why they are together, the golden apple
signifies the universe, which, as it is made of opposites, is rightly said to
be thrown by Strife, and as the various gods give various gifts to the universe
they are thought to vie with one another for the possession of the apple;
further, the soul that lives in accordance with sense-perception (for that is
Paris), seeing beauty alone and not the other powers in the universe, says that
the apple is Aphrodite's.

Theological myths suit philosophers, physical and psychical myths poets;
blended myths suit solemn rites, since every rite seeks to give us union with
the universe and with the gods. If I must relate another myth, it is said that
the Mother of the gods saw Attis lying by the river Gallos and became enamoured
of him, and took and set on his head the starry cap, and kept him thereafter
with her, and he, becoming enamoured of a nymph, left the Mother of the gods
and consorted with the nymph. Wherefore the Mother of the gods caused Attis to
go mad and to cut off his genitals and leave them with the nymph and to return
and dwell with her again. Well, the Mother of the gods is a life-giving
goddess, and therefore she is called mother, while Attis is creator of things
that come into being and perish, and therefore is he said to have been found by
the river Gallos: for Gallos suggests the Galaxias Kyklos or Milky Way, which
is the upper boundary of matter liable to change. So, as the first gods
perfect the second, the Mother loves Attis and gives him heavenly powers
(signified by the cap). Attis, however, loves the nymph, and the nymphs
preside over coming into being, since whatever comes into being is in flux. But
since it was necessary that the process of coming into being should stop and
that what was worse should not sink to the worst, the creator who was making
these things cast away generative powers into the world of becoming and was
again united with the gods. All this did not happen at any one time but always
is so: the mind sees the whole process at once, words tell of part first, part
second.\footnote{What is ever present to the \textit{nous} is projected into
the succession of historical events.} Since the myth is so intimately related
to the universe we imitate the latter in its order (for in what way could we
better order ourselves?) and keep a festival therefore. First, as having like
Attis fallen from heaven and consorting with the nymph, we are dejected and
abstain from bread and all other rich\footnote{As for instance pomegranates,
dates, fish, and pork.} and coarse food (for both are unsuited to the soul).
Then come the cutting of the tree and the fast, as though we also were cutting
off the further progress of generation; after this we are fed on milk as though
being reborn; that is followed by rejoicings and garlands and as it were a new
ascent to the gods. This interpretation is supported also by the season at
which the ceremonies are performed, for it as about the time of spring and the
equinox, when things coming into being cease so to do, and day becomes longer
than night, which suits souls rising to life. Certainly the rape of Kore is
said in the myth to have happened near the other equinox, and this signifies
the descent of souls. To us who have spoken thus concerning myths may the gods
themselves and the spirits of those who wrote the myths be kind.

\paragraph{V} Next, the learner should know the First Cause and the classes of
the gods subordinated to it and the nature of the universe, the essential
characters of mind and soul, Providence too and Fate and Chance, virtue and
vice, and should see the good and evil constitutions arising from them, and
whence it was that evils came into the universe. Each of these topics requires
many long discussions, but there is perhaps no reason why we should not treat
them here in a summary way, to prevent readers from being completely ignorant
of them.

The First Cause must be one, since the unit is superior to all other numbers,
and surpasses all things in power and goodness, for which reason all things
must partake of it; because of its power nothing else will bar it, and by
reason of its goodness it will not keep itself aloof. Now if the First Cause
was soul, everything would be animated by soul, if intelligence, everything
would be intellectual, if being, everything would share in being. Some in fact,
seeing that all things possess being, have thought that the First Cause was
being. This would be correct if things that were in being were in being only
and were not good. If, however, things that are are by reason of their goodness
and share in the good, then what is first must be higher than being and in fact
good. A very clear indication of this is that fine souls for the sake of the
good despise being, when they are willing to face danger for country or friends
or virtue. After this unspeakable power come the orders of the gods.

\paragraph{VI} Of the gods some are mundane, some supramundane. By mundane I
mean the gods who make the universe. Of the supramundane some make the essences
of the gods, some the intelligence, some the souls:\footnote{So rather than as
Murray, ``Of the Hypercosmic gods some create Essences.''} they are therefore
divided into three orders, all of which may be found in treatises on these
matters. Of the mundane some cause the universe to exist, others animate it,
others harmonise it out of its varied components, others guard it when so
harmonised.  These are four operations, and each has a beginning, a middle, and
an end; their superintendents, therefore, must be twelve in number. The
creators of the universe are Zeus, Poseidon, and Heph{\ae}stos, the animators
Demeter, Hera, and Artemis, the harmonisers Apollo, Aphrodite, and Hermes, and
the guardians Hestia, Athena, and Ares. Hints of these functions may be seen in
their images: Apollo strings a lyre, Athena is armed, and Aphrodite is naked
because harmony causes beauty, [...], and beauty in things seen is not
concealed. While these gods possess the universe in a primary way, the other
gods must be supposed to be contained in them, as for instance Dionysos in
Zeus, Asklepios in Apollo, and the Graces in Aphrodite. Further, we can see
their spheres, earth as Hestia's, water as Poseidon's, air as Hera's, fire as
that of Heph{\ae}stos, and six spheres, those higher, belonging to the gods to
whom they are usually assigned; for we must regard Apollo and Artemis as Sun
and Moon. We must give the sphere of Kronos to Demeter, the ether again to
Athena, while the firmament is common to them all. So in this manner have the
orders and powers and spheres of the twelve gods been set forth and hymned.

\paragraph{VII} The universe itself must be imperishable and uncreated,
imperishable because if it perishes God must necessarily make either a better
or a worse or the same or disorder: (if He made a worse, then He is bad in that
He makes what is worse from what is better; if He made a better, He must have
been deficient in power not to have made the better thing in the first place;
if the same, that will be a purposeless creation; if disorder, why, that will
not bear hearing). That it is uncreated even what I have said suffices to show,
because if it does not perish, neither did it come into being, since whatever
comes into being perishes, coupled with the fact that, since the universe
exists because of God's goodness, it follows that God is ever good and the
universe ever exists, as light accompanies the existence of sun and fire, and
shadow that of body.

Of the bodies in the universe some imitate mind and have a circular motion,
while others imitate soul and have a rectilinear motion. Of the latter, fire
and air move upwards, earth and water downwards: of the former the sphere of
the fixed stars moves from East to West, and the seven planetary spheres move
from West to East: among the many reasons for this is the need of preventing
the process of creation from being imperfect if the rotation of the spheres is
rapid. This difference of motion implies a difference in the nature of the
bodies; the heavenly body cannot scorch or chill or perform any other function
of the four elements. Since the universe is a sphere (as is shown by the
zodiac), and the lowest part of a sphere, being furthest distant from all
points on its circumference, is its center,\footnote{That is, as Murray
renders, ``in every sphere `down' means `towards the center.'{''}} and heavy
bodies move downwards and move towards the earth, it follows that the earth is
the center of the universe. All these things are made by the gods, ordered by
mind, and set in motion by soul. Concerning the gods I have spoken earlier.

\paragraph{VIII} Mind is a power inferior to being and superior to soul,
deriving existence from being and perfecting soul (as the sun perfects sight).
Of souls some are rational and immortal, others irrational and mortal: the
former are derived from the primary gods, the latter from the secondary. We
must first investigate the nature of soul. It is that whereby animate differs
from inanimate, and the differences lies in motion, perception, imagination,
and intelligence. Irrational soul is life with perception and imagination,
rational is life controlling perception and imagination and employing reason.
Irrational soul is subject to the feelings of the body, it desires and is
angered unreasonably. Rational soul despises the body reasonably and fights
against the irrational; if it is successful, it produces virtue, if it is
worsted, vice. Immortal it must be, because it knows the gods (and nothing
mortal knows what is immortal), and despises human affairs as not affecting
itself, and, not being of the nature of body, has an experience which is the
opposite of the body's; when the body is beautiful and young, the soul errs,
when the body is aging, the soul is at its prime. Again, every good soul has
employed mind, and mind is created by no body; how indeed could things lacking
in mind create mind? The soul uses the body as an instrument, but is not within
it, just as the engineer is not within the engine, and in fact many engines
move without any one touching them. If the soul is often caused by the body to
err, we must not be surprised: even so the arts cannot do their work if their
instruments are spoiled.

\paragraph{IX} The providence of the gods can be seen even from these facts
which have been stated.\footnote{The first question which follows looks back to
ch.~\textsc{vii}, the second to ch.~\textsc{viii}.} Whence comes the order of
the universe if there is nothing that sets it in order? Why is it that
everything comes into being for a purpose, as, for instance, irrational soul
that there may be perception, rational soul that the earth may be adorned?
Providence can be seen again from its application to our bodies. The eyes were
made transparent that we might see, the nose put over the mouth that we might
distinguish evil-smelling food; of the teeth those in front are sharp, to cut
the food, those within flat, to grind it. In this way we see that every detail
in every part is in accordance with reason. But it is impossible that there
should be providence to such an extent in mean details, and not at all in first
things. The oracles and healings which happen in the universe also belong to
the good providence of the gods. We must consider that the gods bestow all this
attention on the universe without any deliberation or toil: just as bodies with
a function do what they do merely by existing, as the sun lights and warms
merely by existing, in this way and much more so does the providence of the
gods benefit its objects without involving toil for itself. Hence the questions
of the Epicureans are answered: their contention is that what is divine neither
is itself troubled nor troubles others. Such is the incorporeal providence of
the gods for bodies and souls. Their providence exercised from bodies upon
bodies is different from this and is called Heimarmene, because the Heirmos or
chain appears more clearly in bodies. It is with reference to this Heimarmene
that the art of astrology has been invented. It is reasonable and correct to
believe that not only the gods but also the divine heavenly bodies govern human
affairs, and in particular our bodily nature. Hence reason discovers that
health and disease and good and evil fortune come as deserved from this cause.
On the other hand, to suppose that acts of injustice and wantonness come thence
is to make us good and the gods bad, unless what is meant thereby is that
everything happens for the good of the universe as a whole and of all things in
a natural condition, but that evil education or weakness of nature changes the
blessings of Heimarmene to evil, as the sun, good as it is for all, is found to
be harmful to those suffering from inflammation of the eyes or from fever.
Otherwise, why do the Massaget{\ae} eat their fathers and the Jews circumcise
themselves and the Persians preserve their nobility by begetting children on
their mothers? How, when astrologers call Saturn and Mars maleficent, do they
again make them beneficent, ascribing philosophy and kingship, commands in war
and finding of treasures to them? If they talk of trines and squares, it is
strange that human virtue should remain the same everywhere, but the gods
change their natures with their positions. The mentioning in horoscopes of good
birth or evil birth of ancestors shows that the stars do not cause all things,
but do no more than indicate some. How indeed could events before the moment of
birth be produced by the conjunction of heavenly bodies at that moment?

So then, as Providence and Heimarmene exist for tribes and cities and exist
also for each individual, in like manner does Fortune, about which I must next
speak. The power of the gods that orders the good diverse and unexpected
happenings is considered to be Fortune: and for this reason in particular
cities ought to pay corporate worship to this goddess, since every city is
composed of diverse components. Fortune's power rests in the moon,\footnote{Or
``Fortune's power extends to the moon.''} since above the moon nothing
whatsoever could happen because of her. If the bad prosper and the good suffer
poverty, we must not be surprised. The former do anything to obtain wealth, the
latter nothing: from the bad prosperity cannot take their badness, while the
good will be content with virtue alone.

\paragraph{X} This discussion of virtue and vice requires again a discussion of
the soul. When the irrational soul enters bodies and at once produces spirit
and desire, the rational soul, presiding over these, causes the entire soul to
consist of three parts, reason, spirit, and desire. The excellence of reason is
wisdom, of spirit courage, of desire temperance, of the whole soul justice.
Reason must make a right judgement, spirit must, in obedience to reason,
despise seeming dangers, and desire must pursue not seeming pleasure but
reasonable pleasure. When these conditions are fulfilled life becomes just
(justice in money matters is but a small part of virtue). For this reason in
the educated all virtues may be seen, while among the uneducated one is brave
and unjust, one temperate and imprudent, one prudent and intemperate, and
indeed it is not right to call these qualities virtues when shorn of reason and
imperfect and occurring in certain unreasoning creatures. Vice must be
considered by examining the opposites; the vice of reason is folly, of spirit
cowardice, of desire intemperance, and of the whole soul injustice. Virtues
are the products of a rightly constituted state and of good upbringing and
education, vices of their opposites.

\paragraph{XI} Constitutions also correspond to the triple division of the
soul: the rulers resemble reason, the soldiers spirit, and the commoners
desire. Where everything is done in accordance with reason, and the best man of
all rules, monarchy results; where everything is done in accordance with reason
and spirit, and more than one rule, the product is aristocracy; where men
regulate their political life by desire, and honors go by wealth, the
constitution is called timocracy. The opposite of monarchy is tyranny, since
monarchy acts always in accordance with reason, tyranny never; of aristocracy
oligarchy, since not the best but a few and the basest rule; of timocracy
democracy, since not men of property but the commons control the state.

\paragraph{XII} But how is it, if the gods are good and make everything, that
there are evils in the universe? Perhaps we must first say that, since the gods
are good and make everything, evil has no objective existence, and comes into
being through the absence of good, just as darkness has no absolute existence,
and comes into being through the absence of light. If evils exist, they must
be in gods or in minds or in souls or in bodies. But in gods they cannot be,
since every god is good, and if anyone says that mind is evil, he represents it
as the negation of itself, if soul, he will make it worse than the body, since
every body in itself is free from evil; if he asserts that evil arises from the
soul and the body, it is unreasonable that they should not be evil when
separate but should, when combined, create evil. If again spirits are called
evil, they, if they owe their existence to the gods, cannot be evil; if they
owe it to some other source, it follows that the gods do not make everything,
and if they do not make everything, either they wish to do so but cannot, or
they can but will not; neither supposition is suitable to a god. From these
considerations it can be perceived that there is nothing naturally evil in the
universe; evils appear in connection with the activities of men, and not of all
men or at all times. Now, if men caused these evils for the evil's sake, Nature
itself would be evil; but if the adulterer thinks adultery evil, but pleasure
good, or the murderer murder bad, but money good, or he who harms an enemy hard
bad, but vengeance good, and all the soul's sins happen in this way, evil
arises because of goodness. In fact, the soul sins because, though desiring
good, it errs in respect of what is good through not being of First Being.
That it may not err, and that if it errs it may be cured, is the object of many
things which the gods have created and we can see; arts and sciences and
virtuous deeds, prayers and sacrifices and solemn rites, laws and
constitutions, trials and punishments came into being to prevent souls from
sinning, and when souls have left the body they are purged of their sins by
gods and spirits of purification.

\paragraph{XIII} Of the gods and of the universe and of human affairs this
account will suffice for those who neither can be steeped in philosophy nor are
incurably diseased in soul. It remains that we should discuss the fact that all
these things never came into existence nor are separated from one another,
since I have spoken earlier of second things proceeding from first things.

Everything that comes into being is created by technical skill or by natural
process or in virtue of a function. Creators by skill or by a natural process
must be prior to their creations: creators in virtue of a function bring their
products into existence with themselves, since their function, like the sun's
light, fire's heat, snow's cold, cannot be separated from them. If then the
gods make the universe by skill, they make its character but not its existence,
since form is what technical skill always makes. Whence in that case does the
universe derive its existence? If the gods create by nature, we know that what
creates by nature must give of itself to its creation. So, as the gods are
incorporeal, the universe ought to be incorporeal; and if it is maintained that
the gods are corporeal, whence comes the power of things incorporeal? If we
accepted this view, the destruction of the universe involves also the
destruction of its creator, if he created by natural process. If, however, the
gods make the universe neither by technical skill nor by nature, the remaining
view is that they make it by a function. Everything made in virtue of a
function comes into being with the possessor of the function, and things so
made cannot ever perish, unless their maker is deprived of the functional
power. Accordingly, those who suppose that the universe perishes deny the
existence of gods, or, if they assert that existence, make the Creator
powerless. Therefore, as He makes everything in virtue of a functional power,
He makes all things coexistent with Himself. So, as He had the greatest power,
it was necessary that He should make not only men and animals, but also gods
and angels (?) and spirits, and the wider the gap is between our nature and the
first god, the more powers must there be between us and Him, since all things
furthest removed have many intermediate points.

\paragraph{XIV} If any man thinks it a reasonable and correct view that the
gods are not subject to change, and then is unable to see how they take
pleasure in the good and turn their faces away from the bad, are angry with
sinners and propitiated by service, it must be replied that a god does not take
pleasure (for that which does is also subject to pain) or feel anger (for anger
also is an emotion), nor is he appeased by gifts (that would put him under the
dominion of pleasure), nor is it right that the divine nature should be
affected for good or for evil by human affairs. Rather, the gods are always
good and do nothing but benefit us, nor do they ever harm us: they are always
in the same state. We, when we are good, have union with the gods because we
are like them; if we become bad, we are separated from them because we are
unlike them. If we live in the exercise of virtue, we cling to them; if we
become bad, we make them our enemies, not because they are angry but because
our sins do not allow the gods to shed their light upon us and instead subject
us to spirits of punishment. If by prayers and sacrifices we obtain release
from our sins, we do not serve the gods nor change them, but by the acts we
perform and by our turning to the divine we heal our vice and again enjoy the
goodness of the gods. Accordingly, to say that the gods turn their faces away
from the bad is like saying that the sun hides himself from those bereft of
sight.

\paragraph{XV} These considerations settle also the question concerning
sacrifices and the other honors which are paid to the gods. The divine nature
itself is free from needs; the honors done to it are for our good. The
providence of the gods stretches everywhere and needs only fitness for its
enjoyment. Now all fitness is produced by imitation and likeness. That is why
temples are a copy of heaven, altars of earth, images of life (and that is why
they are made in the likeness of living creatures), prayers of the intellectual
element, letters of the unspeakable powers on high, plants and stones of
matter, and the animals that are sacrificed of the unreasonable life in us.
From all these things the gods gain nothing (what is there for a god to gain?),
but we gain union with them.

\paragraph{XVI} I think it worth while to add a few words about sacrifices.  In
the first place, since everything we have comes from the gods, and it is just
to offer to the givers first fruits of what is given, we offer first fruits of
our possessions in the form of votive offerings, of our bodies in the form of
hair, of our life in the form of sacrifices. Secondly, prayers divorced from
sacrifices are only words, prayers with sacrifices are animated words, the word
giving power to the life and the life animation to the word. Furthermore, the
happiness of anything lies in its appropriate perfection, and the appropriate
perfection of each object is union with its cause. For this reason also we pray
that we may have union with the gods. So, since though the highest life is that
of the gods, yet man's life also is life of some sort, and this life wishes to
have union with that, it needs an intermediary (for objects most widely
separated are never united without a middle term), and the intermediary ought
to be like the objects being united. Accordingly, the intermediary between life
and life should be life, and for this reason living animals are sacrificed by
the blessed among men today and were sacrificed by the men of old, not in a
uniform manner, but to every god the fitting victims, with much other
reverence. Concerning this subject I have said enough.

\paragraph{XVII} That the gods will not destroy the universe has been stated;
that its nature is immortal must now be set forth. Whatever is destroyed is
destroyed either by itself or by something else. If the universe is destroyed
by itself, fire ought to burn itself and water dry itself. If the universe is
destroyed by something else, that something must be either corporeal or
incorporeal. Incorporeal it cannot be, since things incorporeal, as nature and
soul, preserve things corporeal, and nothing is destroyed by what naturally
preserves it. If corporeal, it must be one of existents or of non-existents; if
the first then bodies moving in circles must destroy bodies moving in straight
lines or bodies moving in straight lines must destroy bodies moving in circles.
But bodies moving in circles do not possess a destructive nature; otherwise,
why do we see nothing perishing thence? Nor can bodies moving in straight lines
touch bodies moving in circles; otherwise, why have they hitherto been unable
to do so? Nor, again, can bodies moving in straight lines be destroyed by one
another, since the destruction of one is the creation of another, and this is
not destruction but change.

If the universe is destroyed by other bodies, whence they come or where they
now are cannot be said. Further, whatever perishes, perishes either in form or
in matter, form being the shape, matter the body. If the form perishes and the
matter remains we see other things being produced; if matter perishes, why has
it not failed in all these years? If matter perishes, and other matter takes
its place, the latter must come either from existents or from non-existents; if
from existents, so long as they remain for ever, matter is for ever, and if
existents perish, this means the destruction not merely of the universe but of
everything; if from non-existents, firstly, it is impossible that anything
should come from non-existents, and secondly, if this should happen and it
should be possible for matter to come from non-existents, so long as
non-existents are, matter also will be: for surely non-existents do not also
perish. But if they say that matter remains without form, firstly, why does
this happen to the whole universe and not to parts? Secondly, they deprive
bodies of beauty alone, not of being.

Further, whatever perishes either is resolved into its components or disappears
into nothingness. If it is resolved into its elements, other things are again
produced; if this were not so, why were the components made in the first place?
If, however, existents will disappear into nothingness, what prevents this from
happening to God too? But if His functional power prevents it, such power does
not belong to one able only to preserve himself. It is equally impossible for
existents to be produced out of non-existents and for existents to vanish into
nothingness.

Then too, the universe, if it perishes, must perish either in accordance with
nature or contrary to nature. $\langle$If it perishes in accordance with
nature, then the making and continuance till now of the universe prove to be
unnatural, and yet nothing is made contrary to nature$\rangle$, nor does what
is contrary to nature take precedence over nature. If it perishes contrary to
nature, there must be another nature changing the nature of the universe, and
this we do not see. Further, whatever perishes naturally we too can destroy:
but the circular body of the universe no one has ever destroyed or changed,
while the elements can be changed, but not destroyed. Moreover, whatever
perishes is changed by time and grows old, but the universe remains unchanged
by all the lapse of time. Having said so much in answer to those who require
stronger proofs, I pray that the universe may itself be propitious to me.

\paragraph{XVIII} Again, the fact that unbelief has arisen in certain parts of
the earth and will often occur hereafter should not disturb men of sense. Such
neglect does not affect the gods, just as we saw that honors do not benefit
them: further, the soul, being of a middle nature, cannot always judge aright,
and the entire universe cannot equally enjoy the providence of the gods: some
sections can always participate therein, some at times, some in the first
degree, some in the second degree, just as the head possesses all the senses,
the body as a whole, one only. For this reason, it seems, the founders of
festivals established also banned days, on which some temples were idle, some
shut, some even stripped of their ornaments: the perfunctory service was done
in view of the weakness of human nature. It is, moreover, not unlikely that
unbelief is a kind of punishment: it is reasonable that those who have known
the gods and despised them should in another life be deprived of this
knowledge, and that Justice should cause those who honored kings of their own
as gods to be banished from the true gods.

\paragraph{XIX} But if neither for these sins nor for others the punishment
follows directly on the offence, we must not be surprised, because not only are
there spirits that punish souls but also the soul brings itself to judgment,
and because, since souls survive through eternity, they ought not in a short
time to bear all their chastisement, and because there must be human virtue;
for if punishments followed directly on offences, men would do right from fear
and would not have virtue. Souls are punished after leaving the body, some
wandering here, others to hot or cold places in the earth, others being
tormented by spirits; all these things they endure together with the
unreasonable soul, in whose company they sinned: because of this the shadowy
form seen about tombs, especially of evil livers, comes into being.

\paragraph{XX} If transmigration of a soul happens into a rational creature,
the soul becomes precisely that body's soul, if into an unreasoning creature,
the soul accompanies it from outside as our guardian spirits accompany us; for
a rational soul could never become the soul of an irrational creature. The
reality of transmigration can be seen from the existence of congenital
complaints (else why are some born blind, some born paralysed, some born
diseased in soul?) and from the fact that souls which are naturally qualified
to act in the body must not, once they have left it, remain inactive throughout
time. Indeed, if souls do not return into bodies, they must either be unlimited
in number or God must continually be making others. But there is nothing
unlimited in the universe, since in what is ordered by limit there cannot be
anything unlimited. Nor is it possible that other souls should come into being,
for everything in which something new is produced must be imperfect, and the
universe, as proceeding from what is perfect, should be perfect.

\paragraph{XXI} Souls that have lived in accordance with virtue have as the
crown of their happiness that freed from the unreasonable element and purified
from all body they are in union with the gods and share with them the
government of the whole universe. Yet, even if they attained none of these
things, virtue itself and the pleasure and honors of virtue, and the life free
from pain and from all other tyrants, would suffice to make happy those who had
chosen to live in accordance with virtue and have proved able.

\end{document}
