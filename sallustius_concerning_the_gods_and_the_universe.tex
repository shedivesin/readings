\documentclass[12pt]{article}
\usepackage{ebgaramond}
\usepackage{microtype}

\title{Concerning the Gods and the Universe}
\author{Sallustius \and Arthur Darby Nock (tr.)}
\date{}

\begin{document}
\maketitle

\paragraph{I} Those who would learn about the gods need to have been
well educated from childhood and must not be bred up
among foolish ideas; they must also be good and intelligent
by nature, in order that they may have something in common
with the subject. Further, they must be acquainted with
universal opinions, by which I mean those in which all men,
if rightly questioned, would concur; such opinions are that
every god is good and impassive and unchangeable (since
whatever changes, changes for better or for worse; if for
worse, it becomes bad, if for the better, it proves to have been
bad in the first place).

\paragraph{II} Such must be the learner, and his
instruction should be as follows. The essences of the gods
never came into being, for whatever always exists never
comes into being, and all things that have first power and
are by nature impassive do exist always; they are not formed
of bodies, for even of bodies the powers are bodiless; they
are not limited by space, for that certainly is an attribute
of bodies; and they are never separated from the First Cause
or from one another, any more than are thoughts from the
mind, sciences from the soul, or the senses from a living
creature.

\paragraph{III} It is worth our while to enquire why the ancients left the
statement of these truths and employed myths, and so to
obtain this first benefit from the myths, that we enquire and
do not keep our intellects in idleness. Consideration of those
who have employed myths justifies us in saying that myths
are divine; for indeed the inspired among poets, and the
best of philosophers, and the founders of solemn rites, and
the gods themselves in oracles, have employed myths. Why
myths are divine is a question belonging to philosophy.
Since all things in existence rejoice in likeness and turn
from unlikeness, it follows that our statements about the
gods ought to be like the gods, in order that being worthy
of their true nature they may find favour for their narrators
(and such favour can by myths alone be won). So the myths
represent the gods in respect of that which is speakable and
that which is unspeakable, of that which is obscure and that
which is manifest, of that which is clear and that which is
hidden, and represent the goodness of the gods; just as the
gods have given to all alike the benefits to be drawn from
objects perceptible to the senses while restricting to the wise
the enjoyment of those received from objects perceptible to
the intellect, so the myths proclaim to all that the gods
exist, telling who they are and of what sort to those able to
know it. Again, myths represent the active operations of
the gods. The universe itself can be called a myth, since
bodies and material objects are apparent in it, while souls
and intellects are concealed. Furthermore, to wish to teach
all men the truth about the gods causes the foolish to
despise, because they cannot learn, and the good to be
slothful, whereas to conceal the truth by myths prevents
the former from despising philosophy and compels the latter
to study it. Why, however, have the ancients told in their
myths of adulteries and thefts and binding of fathers and
other strange things? Is this also admirable, meant to teach
the soul by the seeming strangeness at once to think the
words a veil and the truth a mystery?

\paragraph{IV} Of myths some are theological, some physical; there are
also psychical myths and material myths and myths blended
from these elements. Theological myths are those which
do not attach themselves to any material objects but regard
the actual natures of the gods. Such is the tale that Kronos
swallowed his children; since the god is intellectual, and all
intellect is directed towards itself, the myth hints at the
god's essential nature. Again, it is possible to regard myths
in a physical way when one describes the activities of the
gods in the universe; so some before now have thought
Kronos to be Chronos or Time, and calling the parts of Time
children of the whole say that the father swallows his
children. The psychical interpretation lies in considering
the activities of the soul itself: the thoughts of our souls,
even if they go forth to others, still remain in their creators.
The worst explanation, the material, is that which the
Egyptians because of their ignorance used most; they
regarded and described material things as goods, earth as
Isis, moisture as Osiris, heat as Typhon, or water as Kronos,
the fruits of the soil as Adonis, wine as Dionysos.\footnote{As Wendland
remarks, \textit{Berl.~phil.~Woch.}~1899, 1411, this sentence,
in which Greek gods are named after Egyptian deities, apparently as
in the same category, is clumsy, but the clumsiness may well be due
to the author.} To say
that these things, as also plants and stones and animals, are
sacred to the gods, is the part of reasonable men, to call
them gods is the part of madmen, unless by a common
figure of speech, as we call the sphere of the sun and the ray
coming from that sphere the sun. The blended kind of
myths can be seen in numerous examples; one is the tale
they tell that at the banquet of the gods Strife threw a
golden apple and the goddesses, vying with one another for
its possession, were sent by Zeus to Paris to be judged;
Paris thought Aphrodite beautiful, and gave her the apple.
Here the banquet signifies the supramundane powers of the
gods, and that is why they are together, the golden apple
signifies the universe, which, as it is made of opposites, is
rightly said to be thrown by Strife, and as the various gods
give various gifts to the universe they are thought to vie
with one another for the possession of the apple; further, the
soul that lives in accordance with sense-perception (for that
is Paris), seeing beauty alone and not the other powers in
the universe, says that the apple is Aphrodite's.

Theological myths suit philosophers, physical and psychical
myths poets; blended myths suit solemn rites, since
every rite seeks to give us union with the universe and with
the gods. If I must relate another myth, it is said that the
Mother of the gods saw Attis lying by the river Gallos and
became enamoured of him, and took and set on his head
the starry cap, and kept him thereafter with her, and he,
becoming enamoured of a nymph, left the Mother of the
gods and consorted with the nymph. Wherefore the Mother
of the gods caused Attis to go mad and to cut off his genitals
and leave them with the nymph and to return and dwell
with her again. Well, the Mother of the gods is a life-giving
goddess, and therefore she is called mother, while Attis is
creator of things that come into being and perish, and
therefore is he said to have eben found by the river Gallos:
for Gallos suggests the Galaxias Kyklos or Milky Way,
which is the upper boundary of matter liable to change.
So, as the first gods perfect the second, the Mother loves
Attis and gives him heavenly powers (signifies by the cap).
Attis, however, loves the nymph, and the nymphs preseide
over coming into being, since whatever comes into being is
in flux. But since it was necessary that the process of coming
into being should stop and that what was worse should
not sink to the worst, the creator who was making these
things cast away generative powers into the world of
becoming and was again united with the gods. All this did
not happen at any one time but always is so: the mind sees
the whole process at once, words tell of part first, part second.\footnote{As
Praechter explains, \textit{W.~kl.~Ph.}~1900, 184, what is ever present
to the \textit{nous} is projected into the succession of historical events.}
Since the myth is so inteimately related to the universe we
imitate the latter in its order (for in what way could we
better order ourselves?) and keep a festival therefore. First,
as having like Attis fallen from heaven and consorting with
the nymph, we are dejected and abstain from bread and all
other rich\footnote{As for instance pomegranates, dates, fish, pork
(H.~Hepding, \textit{Attis}, 156 f.).} and coarse food (for both are unsuited
to the soul). Then come the cutting of the tree and the fast, as
though we also were cutting off the further progress of
generation; after this we are fed on milk as though being
reborn; that is followed by rejoicings and garlands and as
it were a new ascent to the gods. This interpretation is
supported also by the season at which the ceremonies are
performed, for it as about the time of spring and the equinox,
when things coming into being cease so to do, and day
becomes longer than night, which suits souls rising to life.
Certainly the rape of Kore is said in the myth to have
happened near the other equinox, and this signifies the
descent of souls. To us who have spoken thus concerning
myths may the gods themselves and the spirits of those who
wrote the myths be kind.

\paragraph{V} Next, the learner should know the First Cause and the
classes of the gods subordinated to it and the nature of the
universe, the essential characters of mind and soul, Providence
too and Fate and Chance, virtue and vice, and should
see the good and evil constitutions arising from them, and
whence it was that evils came into the universe. Each of
these topics requires many long discussions, but there is
perhaps no reason why we should not treat them here in a
summary way, to prevent readers from being completely
ignorant of them.

The First Cause must be one, since the unit is superior
to all other numbers, and surpasses all things in power and
goodness, for which reason all things must partake of it;
because of its power nothing else will bar it, and by reason
of its goodness it will not keep itself aloof. Now if the
First Cause was soul, everything would be animated by soul,
if intelligence, everything would be intellectual, if being,
everything would share in being. Some in fact, seeing that
all things possess being, have thought that the First Cause
was being. This would be correct if things that were in being
were in being only and were not good. If, however, things
that are are by reason of their goodness and share in the
good, then what is first must be higher than being and in
fact good. A very clear indication of this is that fine souls
for the sake of the good despise being, when they are willing
to face danger for country or friends or virtue. After this
unspeakable power come the orders of the gods.

\end{document}
