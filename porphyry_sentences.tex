\documentclass{article}
\usepackage{ebgaramond}
\usepackage{graphicx}
\usepackage{microtype}

\title{Sentences}
\author{Porphyry \and Thomas Davidson (tr.)}
\date{}

\begin{document}
\maketitle

\paragraph{1} All body is in space: no one of the things which in themselves are incorporeal, or anything of such nature, is in space.

\paragraph{2} The things which in themselves are incorporeal, from the fact that they are superior to all body and space, are everywhere; not in a sundered, but in an undivided condition.

\paragraph{3} The things which in themselves are incorporeal are not locally present in bodies, but are present in them when they wish, inclining to them in the manner in which it is their nature to incline. But though not locally present to them, they are present by relation.

\paragraph{4} The things which in themselves are incorporeal are not present in reality and essence; for they are not commingled with bodies; but by the existence consequent upon their inclination, they impart a certain power which is immediate to the bodies. For the inclination gives existence to a second power, which is immediate to bodies.

\paragraph{5} The soul is a somewhat mediate between the essence which is undivided, and, as regards bodies, divided. But the intellect is undivided essence only; bodies are divided only; qualities and material species are concerned with bodies as divided.

\paragraph{6} That which acts upon something else, does not do what it does by approach or contact: but even those things which do perform an action by approach and contact, employ approach by accident.

\paragraph{7} [The] soul is bound down to the body by adverting to the passions arising from it, and it is loosed again by impassivity to it.

\paragraph{8} What Nature has bound, Nature also looses; and what the soul has bound, that it also looses. Now Nature bound the body in the soul; but the soul bound itself in the body. Nature, accordingly, looses the body from the soul; but the soul looses itself from the body.

\paragraph{9} Death, therefore, is twofold, that which is generally recognized, when the body separates from the soul; and that of the philosophers, when the soul separates from the body. And the one does not at all follow the other.

\paragraph{10} We do not think in the same manner in all things, but in a manner consonant with the essence of each. In intellect, for example, we think intellectually; in soul, logically; in plants, seminally; in bodies, phantasmically; and in what transcends these, inconceivably and superessentially.

\paragraph{11} The incorporeal existences in descending are divided and multiplied into atomic things by a remission of power, whilst in ascending they are unified, and revert to inseparateness by superabundance of power.

\paragraph{12} Not only in bodies is there ambiguity [variety of things included under a common name], but life also is of those things that are in many ways. For the life of a plant is one, that of an animal another; that of the intellectual is one, that of (the nature of) the transcendent another. One belongs to the soul, another is intellectual. For these things also live, even although no one of the things that exist from them possesses similar life.

\paragraph{13} Every thing that generates from its own essence generates something inferior to itself, and everything that has been generated adverts by nature to that which generated it. But of the things which generate, some do not advert at all to the things generated, while some partly do and partly do not advert, and some advert only to the products from not adverting to themselves.

\paragraph{14} Everything generated from another contains the cause of the generation, since indeed nothing is generated without a cause. But of things generated, those that possess being through composition (synthesis) would for this very reason be destructible; whereas those things which, being simple and incomposite, possess being in the simple [fact] of existence, being indissoluble, are also indestructible, and are said to be generated, not because they are composite, but because they depend on some cause. Bodies, therefore, are generated in two senses, i.~e.~they either depend upon a cause which produces them, or they are composite. Soul and intellect are generated things, only as depending upon a cause, not as being composite. Some bodies, therefore, are generated and dissoluble and destructible; others are ungenerated in so far as they are incomposite, and hence indissoluble and indestructible, but generated as depending upon a cause.

\paragraph{15} Intellect is not the beginning of all things; for intellect is many. But before the many there must be the one. And that intellect is many is evident; for it always thinks thoughts which are not one, but many, and are not other than it. If, then, it is [one and] the same with them, and they are many, the intellect also must be many: and that it is the same with its intelligibles [objects of thought] is shown thus. For if there is anything which contemplates, it must contemplate what is contained in itself as such, or as placed in another. And what does contemplate is plain, for along with thinking there must be intellect. Being deprived of thinking, it is deprived of essence. Wherefore, directing our attention to those states which are incident upon cognitions, we must trace out its intuition. And the cognitive powers in us are, in general terms, perception, imagination, intellect. That which extends its activity to external things by means of perception, contemplates by contact, not by union with the things which it contemplates, but merely receiving a representation of them from its application to them. When, therefore, the eye sees the seen, it is impossible that it should have come into identity with the seen; for it would not see if it were not at a distance. In a similar manner, that which is touched would be destroyed by coming into identity. From which it is evident that both perception and that which uses perception are always directed outward, if they are to seize the perceptible. Similarly also the imagination is always directed outward, and by its tension it brings into dependent existence an image, or, in other words, it prepares outside, by its very tension outwards, an exhibition of the image as being without. And the act of seizing by these powers is such, that none of them, by converging or contracting into itself, would meet with either perceptible or imperceptible form; whereas in the case of the intellect, the act of seizing is not in this manner, but [takes place] by its converging into itself, and contemplating itself. For by going beyond the viewing of its own energies, and beyond being the eye of its own energies, and the spectacle of essences, it would think nothing. Thus then, in the same manner, as there are [as we have seen] perception and perceptible [object of perception], so also there are intellect and intelligible. The former contemplates, by extending outwards, finding the perceptible contained in matter. But the intellect does so by drawing together into itself, and not at all by extending outwards; although some have held the contrary, thinking that there was merely a difference of name between the existence of the intellect and that of the imagination; for the imagination in the logical animal had appeared to them action of intellect. But for those who make all things depend upon matter and corporeal nature, the logical conclusion is that intellect also depends on these. Whereas intellect, in our sense, is the spectator of corporeal and other essences. Where, then, shall it find and seize them? Since they are things outside of matter, they cannot be anywhere. It is manifest, therefore, that intellectual things must be connected with action of intellect. And hence, if intellectual things are for the intellect, it will contemplate both the intelligible and itself in thinking intelligible things. And withdrawing into itself, it thinks through withdrawing into these. And since intelligible things are many (for the intellect thinks many and not one) the intellect itself must be many. But before the many lies the one, so that the one must be prior to intellect.

\paragraph{16} The memory is not a conserver of imaginations, but of those things which have been meditated to be put forward anew as problems.

\paragraph{17} The soul contains the reasons of all things, and energizes according to them, either when provoked to outward effort by something else, or when directing itself inward upon them. And when called out by something else, it extends its perceptions as to things without; but sinking into itself, towards intellect, it becomes engaged as in the acts of intellect. Therefore, neither are the perceptions nor the acts of intellect outside of the imagination, one might say; [nor is perception or action of intellect anything else. Further,] as in the animal, perceptions do not take place without an affection of the perceptive organs, so the acts of intellect do not take place without imagination; or, to keep up the analogy, as impression is an accompaniment of the perceptive animal (animality) so the image of the animal accompanies the intellectual action of the soul.

\paragraph{18} The soul is an essence without magnitude, immaterial, indestructible, with life which has living from itself, possessing being.

\paragraph{19} The affection of bodies is one, that of incorporeals another. For the affection of bodies is accompanied with change, whereas the intimacies and affections of the soul are energies, which have no similarity to the heatings and coolings of bodies. Wherefore, if the affection of bodies is always accompanied with change, we must affirm that all incorporeal things are impassive. For the things that are separate from matter and from bodies are [as we saw] the same \textit{in actuality;} whereas those things which approach matter and bodies are themselves impassive, and those in which they are contemplated are affected. For whenever the animal perceives, the soul resembles a separate harmony, moving the strings in tune out of itself; while the body resembles the harmony in the strings, which is inseparate. The cause of the moving is the animal, and this from its being endowed with life. It (the animal) may be compared to the musician [who moves the strings], from being endowed with harmony. The bodies affected by a perceptive affection resemble the tuned strings. For there the harmony, which is separate, is not affected, but the string. And the musician moves [the strings] according to the harmony which is in him. Surely the string would not be moved musically, even if the musician wished, unless the harmony dictated it.

\paragraph{20} The names of incorporeal things are not imposed from community in one and the same genus, in the same manner as [those of] bodies are, but from their naked privation with respect to bodies. Hence they are not prevented, some from being existent, others from being non-existent. Thus some are before bodies, and some with bodies; some are separate from bodies, and some inseparate; some are existent by themselves, and some require others in order to be; some are the same with energies and self-moved lives, and some for their lives depend upon particular energies. For it is from a negation of what they are not, and not from a positing of what they are, that they are named.

\paragraph{21} The properties of matter, according to the ancients, are these: [It is] incorporeal, because it is other than bodies; lifeless, because it is neither intellect nor soul, nor aught living in itself; formless, other, infinite, impotent. Wherefore it is not even existent, but non-existent. It is not a non-existent as motion is, but a veritable non-existent. It is an image and phantom of mass; because it is that which is primitively in mass, it is the impotent, it is a striving for existence, a posited not in position, a somewhat always appearing its opposite in itself; small and great, less and greater, deficient and excessive; always becoming, and not remaining, nor yet able to flee; the deficiency of all the existent. Hence, in all that it professes, it lies. Even though it appear great, it is small; for it is like a toy, fleeing to the non-existent. For its flight is not in place, but in desertion from the existent. Hence, the images in it are in a worse image. As in a looking-glass, what is situated in one place is what appears in another. And it is full to appearance, while it contains nothing and seems everything.

\paragraph{22} The affections all relate to that to which destruction relates. For the admission of affection is the path to destruction, and destruction belongs to that to which affection belongs. But none of the incorporeal things perish. And some of them either are or are not, so that they are not at all affected. For that which is affected must not be of this character, but susceptible of becoming other, and of being destroyed by the qualities of those things which assail it and impart the affection. For that which is in a thing cannot be changed by anything that happens. Hence, for example, matter is not affected; for in itself it is without quality; nor are the forms which it takes, and which go in and come out. But the affection has relation to composition, and belongs to that which has its being in composition; for it is this that is contemplated as suffering amid opposite forces, and the qualities of those things which assail it and cause affection. Hence, also, those things whose life is from without and not from themselves are capable of being affected by living and not living. Those things whose being is in an impassive life, must necessarily remain in life; just as affection does not belong to lifelessness, as far it is lifelessness. As, then, change and affection are in the composite, or what is made up of matter and form, as we saw that body is, whereas the same is not true of matter: so also living and dying, and affections of this kind are conceived in the compound of soul and body. This, of course, does not happen to the soul, for it is not a thing composed of lifelessness and life, but of life and that alone. And it is so from being simple essence, and because the reason of the soul is self-moved.

\paragraph{23} The intellectual essence is homogeneous, so that the things which are, are in the universal soul and in the particular. But in the universal even particulars are universally, whereas in the particular even universals as well as particulars are particularly.

\paragraph{24} The death of the essence whose being is in life, and whose affections are lives, must itself lie in a kind of life, not in an absolute deprival of life; for the lifelessness in it is not an affection or path to non-living altogether.

\paragraph{25} In the case of incorporeal lives progressions take place, while the former ones remain firm and immovable, and do not lose anything of themselves into the existence of things below them, or change in anything. So that not even the things called into existence are so called with any loss or change; nor is this aught begotten, like generation, which partakes of decay and change. They are, therefore, ungenerated and indestructible, and, in this sense, begotten ungeneratedly and indestructibly.

\paragraph{26} In regard to that which transcends intellect, much is said in accordance with the acts of intellect; but it is contemplated by the absence rather than the presence of intellectual action. Just as many things are said in regard to the sleeping-state, through waking, whereas it is only through sleeping that cognition and comprehension [of sleep] are. For like is known by like, since all knowledge is an assimilation of the known.

\paragraph{27} Non-being we partly produce by being separated from being, partly preconceive by adhering to being. For, of course, if we are separated from being, we do not preconceive the non-being that is above being: but we give birth to a false feeling, that, namely, which takes place in the case of a person who goes beyond himself [rises into ecstasy]. For every individual, as being actually and through himself, must have the capability of being carried up to the non-being which is above being, and along to the non-being which is the decease of being.

\paragraph{28} To that which is in itself incorporeal, the existence of the body offers no obstacle preventing it from being where it chooses and as it pleases. For as that which has no mass is inapprehensible to body, and is nothing in relation to it, so also that which has mass cannot come in the way of the incorporeal, and stands to it as non-being. Neither locally does the incorporeal move where it chooses (for space is an attribute of mass); nor is it confined by the presence of bodies; whereas whatever is in any way in mass can be confined and makes transition locally. That which is altogether without mass is also without magnitude, incapable of being seized by those things which are in mass, having no participation in local motion. Accordingly it is found in a sort of relation, wheresoever it is related, being everywhere and nowhere. Hence it is by a sort of relation that it is contained, either beyond heaven, or in some part of the cosmos. But when it is contained in any part of the cosmos, it is not seen by the eyes, but its presence becomes manifest from its acts.

\paragraph{29} When we say that the incorporeal is contained in body, we do not mean that it must be shut up like wild animals in a cage, for nothing corporeal can shut up or embrace it; nor as a skin-bottle contains liquid or air; but it must [be supposed to] call into existence powers which incline from unity as related to it, outward, and by which then it descends and is interwoven with bodies. Its coercion into body, therefore, is through an ineffable extension. But neither does anything else bind it down, but it does so itself. In the same manner it does not free the body when broken down and decayed, but it frees itself by turning away from passivity [to the body].

\paragraph{30} Of essences which are whole and perfect no one turns towards its own offspring. But all the perfect essences are carried up to the things which produced them, from the cosmic body upwards. For it, being perfect, is carried up toward the soul, which is intellectual. And for this reason it moves in a circle. And the soul of it [is carried up] toward intellect, and intellect again toward the First. All things, therefore, make their transitions toward it, beginning at the extreme end, each according to its powers. But the ascent towards the First is, nevertheless, either immediate or mediate. Hence these things might be said not only to strive after the deity, but to partake of him according to their power. On the other hand, it is an attribute of divided existences, and those which are able to incline to many things, to turn toward their offspring. Hence, also, in these there must have been error, in these scoffing unbelief. These, then, matter defiles, because they are capable of turning to it, while having the power to be turned toward the divine. So perfection makes a separation of existence between the second and the first, preserving those which are turned towards the first things, whereas imperfection turns the first even toward the last, and makes them love those things which turned away (lapsed) before them.

\paragraph{31} God is every where because nowhere; and intellect is everywhere because nowhere; and soul is everywhere because nowhere. But God is everywhere and nowhere among the things that are after him; and he is there only as he is and desires. Again, intellect is in God, and everywhere and nowhere among the things that are after it. And soul is in intellect and in God everywhere, and nowhere in body; and body also is in soul and in God. And since all things that are and that are not are from God and in God, he is not the things that are and that are not, nor is he in them. For if he were only everywhere he would be all and in all; but since he is also nowhere, all things are produced from him and in him, because he is everywhere, and are other than he, because he is nowhere. Thus also intellect, being everywhere and nowhere, is the cause of souls and the things that are after them. And itself is not soul or the things after soul, nor is it in these, inasmuch as it is not only everywhere in the things that are after it, but also nowhere. And the soul is not body or in body, but the cause of body, because, while being everywhere in the body, it is nowhere. And the progress of the Universe is to that which is capable of being neither everywhere at once nor nowhere, but which partakes partially of both [modes].

\paragraph{32} As it is a property of soul to be upon the earth (not to walk upon the earth as bodies do) and to preside over body, which does walk on the earth, so also it is the property of soul to be in Hades when it presides over a shade, whose nature it is to be in space, but which possesses its essence in darkness. So that if Hades is a dark subterranean place, the soul, though not abstracted from being, comes into Hades, drawing the shade after it. For when it has gone out from the solid body, the spirit which it has collected to it from the spheres follows it. But as, from its sympathy with the body, it has its reason, as a partial one, projected, according to which it had its connection with such and such a body in living, from this sympathy an impression of the imagination is imparted to the spirit and thus it draws the shade to it. It is said to be in Hades because the spirit partakes of the invisible nature, and the murky one. And since the heavy, humid spirit passes down even to the subterranean places, the soul itself is said to depart under ground: not because the same essence traverses places, and comes into places, but because it adopts the relations of bodies whose nature it is to traverse places, and to have places assigned to them, such and such bodies receiving it according to their aptitudes, from their particular disposition toward it. For, according to the manner in which it is disposed, it finds a body determined in rank and properties. Hence, with a soul more purely disposed is united the body approaching the immaterial, viz.~the {\ae}therial one; while with one who has gone beyond reason into the projection of the imagination is united the solar one, and with one that has become effeminate and is impassioned after form, the lunar one is connected. After it it has lapsed into bodies, when, to accord with its shapelessness, there have risen appearances composed of humid vapors, there follows complete ignorance of being, and darkening, and childishness. And indeed also in its egress, when it still has its spirit defiled through the humid evaporation, it draws to it a shadow and is weighed down, inasmuch as such a spirit by nature hastens to depart to a recess of the earth, if no other reason draws it back. For just as the soul which wears the terrene shell must adhere to the earth, so also one that draws to it a humid spirit must wear a shade. And it draws a humid one to it when it studies continually to hold converse with nature, whose operations are in the humid, and mostly subterraneous. But when it studies to withdraw from nature, it becomes a dry splendor, shadowless, cloudless. For humidity in the atmosphere forms cloud, whereas dryness produces from vapor dry splendor.

\paragraph{33} These are the things which can be affirmed with truth regarding the perceptible and the material: that it is universally diffused, that it is changeable, that it has its essence in otherness, that it is composite, that it has [no] existence in and for itself, that it is intuited in place and in mass, and so forth. On the other hand, [the things that can be affirmed] of that which essentially is, are, that it exists in and for itself; that it is always situated within itself, and similarly that it always is in the same manner; that its essence is invested with identity; that it is unchangeable in its essence; that it is incomposite, indissoluble, and not in place, or diffused into mass; that it neither becomes nor decays, and so forth. Adhering to these [distinctions], we ought not, in speaking, to make any confusion between their different natures, or to listen to others when they in speaking do so.

\paragraph{34} One set of virtues belongs to the citizen, another to the man who ascends to contemplation, and who is called for this reason contemplative, and even a contemplator. And different still are those of the intellect, in as far as it is intellect purified from soul. Those of the citizen, consisting [as they do] in moderation of passion, are to follow and to conform to the conclusions based upon a calculation of what is proper or expedient in actions. Hence, because they have in view a social organization which shall not inflict injury upon its members, from the aggregation of the civil community they are called political. And prudence is conversant with that which is reasoned; valor with the passionate; temperance lies in the agreement and harmony of the desires and affections with rational calculation; while justice is the simultaneous limiting of each of these to its own sphere of action, in respect to ruling and being ruled. On the other hand, the virtues of the man who tends to contemplation lie in withdrawal from things here [below]; hence these are also called purifications, being viewed as [consisting] in abstinence from actions requiring the co\"{o}peration of the body, and from sympathies with it. For these belong to the soul which withdraws toward true being. But the political virtues adorn the mortal man, and the political ones are preparatives for the purifications. For the man who is adorned with these must withdraw from doing anything by predilection with the body. Hence in purifications, not to opine with the body, but to energize, alone constitutes right thinking, and it is perfected through thinking purely. Again, freedom from sympathies [with the body] constitutes temperance. Not to fear, when withdrawing from the body, as if it were into something empty and non-being, constitutes valor. And when reason and intellect lead and nothing opposes, this is justice. The disposition, therefore, which is based upon the political virtues may be stated as consisting in moderation of passion, having for its aim to enable a man to live as a man according to nature. The disposition based upon the contemplative virtues consists in apathy, the end whereof is assimilation to God. But since purification [has a twofold meaning], being either that which performs the purifying function, or a property of those who are purified, the contemplative virtues are viewed with reference to both the significations indicated of purification. For they purify the soul, and are with it when it is purified. For the end of purifying is to be purified. But since purifying and having been purified are the removal of all that is alien, the good must be [something] other than the purifying. For if previously to contamination the process of being purified were good, purification would be sufficient. And purification does suffice; but what remains after it is the good, not purification. But the nature of the soul is not a good, but capable of partaking of the good, and having the form of the good. But the good for it is to be united with that which produced it, and evil for it is to be joined with what is after it. The evil is twofold, [first,] the being united with these, and [secondly,] being so with excess of passions. Hence all the political virtues, which free it at least from one evil, have been called virtues and honorable ones. But the purificatory virtues are more honorable, and free the soul from the evil which belongs to it as soul. Wherefore, when it has purified itself, it must unite with that which produced it. And virtue [predicable] of it after its adversion (ascent) consists in cognition and knowing of that which is. Not that it does not have this [knowledge] in itself, but because, without that which is before it, it does not see the things of itself. There is, therefore, a third class of virtues besides the purificative and political, those, namely, which belong to the soul energizing intellectually. Wisdom and prudence lie in contemplation of the things which intellect has, whereas justice is self-related action in the progress toward intellect, and the energizing toward intellect. Temperance again is the turning inward toward intellect. Fortitude is absence of passion, in assimilation to that toward which it looks, and which is by nature passionless. And these follow each other in turn, as others do. There is a fourth species of virtues, namely, the pattern ones, which are in the intellect. These are superior to those of the soul, and are the patterns of those to which the similitudes of the soul belong. For intellect is that in which all things are as patterns. Science is prudence; wisdom is the intellect cognizing; self-relatedness, temperance; peculiar function, self-related action. Valor is sameness, and a remaining pure in self-dependence, through abundance of power. Four kinds of virtues, therefore, have been shown; [first,] those which are of the intellect, exemplars, and concurrents of its essence; [second,] those of the soul already looking inward toward intellect, and filled from it; [third,] those which belong to the soul of a man purifying itself, and purified from the body and irrational passions; [fourth,] those belonging to the soul of man which adorns the man, by setting limits to irrationality and inculcating moderation of the passions. He who has the greater, has, of necessity, the less; but by no means \textit{vice vers\^{a}}. Moreover, from the fact of having the less, he who has the greater will no longer energize according to the less by predilection, but only in consequence of the circumstance of birth. For, as has been said, they have a generic difference of scope. The scope of the political ones is to set a limit to the passions as far as regards the practical energies that have reference to nature; that of the purificative ones is to free entirely from the passions; that of those which relate to the intellect is to energize, without [those who practise them] ever coming to a recollection of the freeing from the passions. The scope of the others is in a manner analagous to those mentioned. Hence he who energizes according to the practical virtues is an earnest man; he who energizes according to the purificative ones, is a demonic man or even a good demon. He who energizes according to those alone which relate to intellect is God. He who energizes according to the pattern virtues is the father of the gods. We ought, therefore, to direct our attention chiefly to the purificative virtues, considering that the attainment of them is possible in this life. And it is through them that the ascent to the more honorable virtues is. Hence we must consider how far and to what extent purification can be carried. For it is a withdrawal from the body and the irrational movements of the passions. How it may be carried out and how far must be stated.

In the first place, then, the foundation, as it were, and basis of purification is self-knowl\-edge---knowledge that one’s soul is bound up with an alien substance of different essence.

In the second place, that which is seen from this basis is [how] to collect oneself from the body, and that which, as it were, is extended in places, and certainly stands in apathetic relation to it. For a person who energizes continually according to sensation, even if he does not do so with sympathy and enjoyment of pleasure, is, nevertheless, distracted by the body, being connected with it through sensation. And we share in the pleasures or pains of the objects of sense with a sympathetic inclination and approval. From which disposition it is incumbent upon a man to purify himself above all things. And this must take place if one partakes only of necessary pleasures, and of the sensations only as far as is necessary for health, or as a relaxation from labor, in order that he may not be fettered. Pains also must be removed; but if this is not possible, they must be borne meekly, and diminished by withdrawal of attention (sympathy) from them. Passion also, as far as possible, must be taken away, and must not be brooded over at all. If this cannot be done, the will, at least, must not be allowed to commingle with it, but must be free from all preference for anything else. But the involuntary is weak and small. And fear must be absent always, for a man must have no fear with regard to anything. The involuntary applies here also. Nevertheless passion and fear must be used in exhortation. Again, desire for everything evil must be exterminated. And he will not indulge in food and drink, in as far as he is self. In [the exercise of] the natural sexual passions the involuntary must have no part, except to the extent of the sudden imagination which takes place during sleep. In a word, let the intellectual soul of the man who is becoming purified be itself pure from all these things. And let it desire that that which moves in the direction of the irrationality of bodily appetites, be moved without sympathy or attention, so that the movements may be cancelled immediately by the presence of that which reasons. There will thus be no combat as the purification progresses; but henceforth reason, being present, will suffice. The inferior will reverence it, so that even the inferior itself will be indignant, if it is at all excited, because it did not keep silence when its master was present, and will reproach itself with weakness. These, then, are the moderations of passion which assume a tendency toward the absence of passion. And when the sympathetic has been thoroughly purged away, the apathetic succeeds it, inasmuch as even the affection derived its movement from the ratiocination which through inclination gives the key-note.

\paragraph{35} Everything, according to its own nature, is somewhere; if only it is somewhere, it is not contrary to nature. For body, therefore, which exists in matter and mass, to be somewhere is to be in place. Hence also for the body of the world, which is material and in mass, being everywhere, is being in extension and place of extension (distance), whereas for the intellectual world, and generally for that which is immaterial and in itself incorporeal, as being unconnected with mass and distance, there is no being in space. So that for the incorporeal, ubiquity is not spatial; and, furthermore, there is not one part of it here, and another part there. For in that case it would not be outside space, or unextended; but it is entire wheresoever it is. Nor is it present in one place and absent in another; for in this way it would be comprehended by space; but it is withdrawn from the hither. Nor is it far from this and near that. For the far and the near are spoken of as having reference to things which are by their nature in space, to measurable distances. So that the world is extendedly present to the intellectual, whereas the incorporeal is present to the world undividedly and unextendedly. And the undivided exists (becomes) entire in the extended, through every part, being the same as one in number. In that, therefore, which is by nature multiplied and magnified, the undivided and unmultiplied is magnified and multiplied; and thus it partakes of it according to its own nature, not according to that of the former. For the undivided and by nature unmultiplied, on the contrary, the divided and multiplied is undivided and unmultiplied, and thus it is present to it; that is to say, it is present without division or multiplication or position, according to its own nature, to that which is divided and multiplied and in space. But that which is divided and multiplied and in space is present to the \textit{other} of these, which is external, without division or multiplication or space. Hence in conducting our considerations, we must seize the peculiarity (property) of each, and not confound their natures; especially we must not imagine or fancy the things which are present to bodies, as connected with the incorporeal. For no one must ascribe the properties of the purely incorporeal to bodies. For with bodies every one has a familiarity; but of the others (incorporeal things) one arrives at a knowledge with difficulty, being undetermined with regard to them, and never coming in direct contact with them so long as he is determined by imagination. You might state it thus:---If the one is in space and outside of itself, inasmuch as it has passed over into mass, the intelligible is not in space, and is in itself inasmuch as it has not passed into mass. If the one is image, the other is archetype. The one possesses being as in relation to the intelligible, the other in itself. For every image is an image of intellect. So, remembering the properties of both, we must not wonder at the interchange which takes place in their conjunction, if indeed we can say conjunction at all; for we are not considering conjunction of bodies, but of things that lie altogether outside of each other in the properties of their existence. Hence, also, conjunction lies outside those properties that are wont to be attributed to things of like essence. There is, therefore, neither fusion, nor mixture, nor conjunction, nor apposition; but their mode is different, appearing on the occasion of the mutual communications that take place in any manner between things of like essence, but lying outside of all the things that fall under perception. In infinite parts, if the unextended, being present entire, meets the extended, it is neither present as divided, giving part to part, nor, though multiplied, does it present itself to multitude as manifold. But it is present to all the parts of that which is in mass, to each unit of the mass, and to the whole mass and the whole multitude without division or multiplication and as one in number. But the partaking of it dividedly and discretely is the attribute of things which have their power divided into parts, and to these it often happens that they falsely cloak their own deficiency under the nature of another, and are at a loss in regard to the essence, which is wont to pass from its own [essence] into that of another.

\paragraph{36} True being is neither great nor small; for great and small are attributes of mass properly. It lies outside the great and the small, and is beyond the greatest and beyond the least, being the same as one in number; although it is found to be partaken of by every greatest and every least. Wherefore you must not conceive it as a maximum, otherwise you will be puzzled as to how, being a maximum, it is present in the smallest masses without being diminished or contracted; neither as a minimum, otherwise you will again be at a loss to conceive how, being a minimum, it is present in the greatest masses without multiplication or increase or extension. But taking together that which goes beyond the greatest mass into a maximum, and the smallest mass into a minimum, you will perceive how it is viewed at once in individuals and in universals, by multitudes and masses, being the same and remaining within itself. For it co\"{e}xists with the magnitude of the world, according to its own properties, without division or magnitude, and, notwithstanding its own indivisibility, it comprehends the mass of the world and every part of the world. So, again, the world in its manifold divisibility is conversant with it, as divided into many parts, and as far as possible. Yet it is not able to include it either totally or to the full extent of its power; but in everything it encounters it as infinite, and incapable of being gone beyond; and this principally for the reason that it is free from all mass.

\paragraph{37} That which is greater in mass is less in power, as compared, not with similar genera, but with things different in species, or through otherness of essence. For as mass was seen to be a going outside of itself, and a division of power into small particles, so, that which excels in power is alien to all mass. For the power, returning into itself, is filled with itself, and strengthening itself maintains its own might. In this manner body, passing over into mass, departs, in diminution of power, from the power of incorporeal true being, to the extent to which true being is not exhausted in mass, remaining in the magnitude of power which is the same through absence of mass. Thus, as true being has neither magnitude nor mass as related to mass, so the corporeal in relation to true being is weak and powerless. For that which is greatest in magnitude of power is destitute of mass. So that the world being everywhere, and everywhere meeting true being---in the sense in which it is said to be everywhere---can not comprehend the magnitude of power. But it meets it as something not dividedly present to it, but present without magnitude or limitation. The presence, therefore, is not spatial, but assimilative, as far as it is possible for body to be assimilated to the incorporeal, and the incorporeal to mirror itself in body assimilated to it. Hence also the incorporeal is not present, in so far as the material cannot be assimilated to the purely immaterial. And the incorporeal is present to the corporeal in so far as it can be assimilated to it; not certainly by inception, for in that case both would be cancelled, the material receiving the immaterial through change into it, and the immaterial becoming material. Assimilations, therefore, and participations of powers and impotences take place reciprocally between things thus differing in essence. So there is great distance between the world and the power of being, and between being and the impotence of the material. But that which lies between, assimilating and assimilated, uniting the extremities, has been the cause of error in regard to the extremities, by adding, through assimilation, dissimilars to the dissimilar.

\paragraph{38} True being is said to be many, not from difference of place, or dimensions of mass, nor from accumulation, or from circumscription or comprehension of divided parts; but from otherness, which is immaterial, destitute of mass, and unmultiplied as regards discrete multiplicity. Hence it is one, not as one body, or as in one place, or as one mass, or as one many; inasmuch as, in so far as it is one, it is other; and its otherness is discrete and united. For its otherness is not acquired from without, nor is it adventitious, nor by participation in somewhat else, but it is many in itself. For with all its energies it energizes, remaining (unchanged), inasmuch as it constitutes its whole otherness through sameness, not reflecting itself in difference between one [part] and another, as in the case of bodies. In the case of these the opposite is true, and oneness consists in otherness, otherness in them being the leading [characteristic], and oneness being adventitiously superinduced from without. Whereas, in the case of being, oneness and sameness are the first, and otherness is produced from the oneness, being energetic. Wherefore the latter is multiplied in indivisibility, whereas the former is unified in multitude and mass. The latter also is situated within itself, being at one in itself; the former is never in itself, as having its constitution in extension. Accordingly, the One is all-efficient (universally-energetic), while multitude is in process of unification. Hence we must examine closely how the latter is one and other, and again how the former is multitude and one, and not interchange the properties of the one with those which belong to the other.

\paragraph{39} We must not think that on account of the multitude of bodies a multitude of souls were produced, but that before bodies there were many and one, without the one and universale preventing the many from being in it, nor the many’s dividing the one among them. They are distinct without being sundered, or having divided up the universal soul among them. And they are present to each other without being confounded, or making the universal soul an agglomeration: for they are neither separated by limits, nor again are they confounded; just as the sciences (knowledges), though many, are not confounded in one soul. Again, they do not inhere, as bodies do in soul, with a difference of essence. But they are a kind of energies of the soul; for the nature, of the soul is of infinite power, and throughout every particular of it, it is soul; and they are all one, and, again, the universal soul is other than they all. For as bodies, when divided \textit{ad infinitum}, do not finally merge into the incorporeal, having their difference in the mere mass (bulk) of the parts, so the soul, being a vital form, includes forms \textit{ad infinitum}. For the differences which it contains are specific ones, and the universal soul is with or without these. If there were anything like action in it¡ there would be otherness, while sameness remained. But if, in the case of bodies in which otherness prevails more than sameness, nothing incorporeal being superinduced broke the union, but all [parts] remained united according to their essence, but distinct as regards qualities and other determinations, what must be said and supposed in the case of specific, incorporeal life, in the case of which identity has prevailed over otherness, and nothing is hypostatized foreign to the form, and from which arises unity in bodies. And not even body when it is added to it breaks the union, although, as regards its energies, it hampers it in many respects. But its identity itself, through itself, does and discovers all things by means of its \textit{ad infinitum} specific energy, although each individual part is capable of all things when it is purified from bodies, just as each individual particle of seed has the power of the whole seed. And as seed contained in matter is contained in proportion to the capability of each individual in the seeds [parts] of matter, and everything that is drawn together within the power of the seed has the whole power of it in each of its parts; so, also, that which is thought under the form of a part of the immaterial soul has the power of the whole soul. But that which has inclined to matter, though receiving the form to which it has inclined, will also be capable of associating with an immaterial form even if it meets with matter in itself, when, withdrawing from the material, it reverts to itself. And since, when inclining to matter, it experiences a lack of all things, and an emptying of its own individual power, and when carried up into intellect, it experiences possession of the fullness of itself according to the power of the whole, those who first recognized this fullness of the soul, enigmatically called the former Poverty, the latter Satiety, and with reason.

\paragraph{40} The ancients, wishing to exhibit the nature of incorporeal being, as far as possible, in words, when they had called it one, immediately added [that it was also] all things, such as the things cognized by the senses are individually. But when we had reflected that this one is diverse, not seeing in the perceptible this whole as One, or that it is all, in the same sense that it is one, from the fact of the All’s being the One itself, they added ``the One in as far as it is one,'' in order that we might think being-all-things, when predicated of being, as an incomposite somewhat, and might be rid of [the idea of] agglomeration. And when they have said it was everywhere, they add that it is nowhere. And when they say it is in all things, and in each individual capable of receiving it sufficiently, they add that it is whole in whole. And generally they express it by means of the most contradictory terms, putting these together in order that we may eliminate from it those conceptions modelled upon bodies which obscure the characteristic properties of being.

\paragraph{41} When you grasp an eternal essence infinite in itself in the extent of its power, and begin to think a substance unwearied, unabating, nowhere deficient, raised aloft in the most utter life, full of itself, situated within itself, sated from itself, and not seeking even itself; [and] if you add to this the [notion of] place or of relation, by the diminution [arising] from lack of place or from relation, you have not at the same time diminished it. On the contrary, \textit{you} have swerved (deserted), taking as a cloak the obtrusive imagination of the reflection. For, such a thing you will neither exceed or go beyond, nor will you give it position or dependence, nor will you diminish it even to a small extont, since it can part with nothing in a process of gradual diminution. For it is more unceasing (inexhaustible) than all fountains, being the ever-flowing, thinking, and incessant. If you cannot encounter it directly, you will not, by comparison of it to all things, be making any inquiry about being. Or if you do make any such inquiry, you will miss your mark and look at something else. But if you make no search, taking your stand upon yourself and your own essence, you will be assimilated to the all, and will not be contained in any of the things that are [derived] from it. And if the All is not limited, neither are you; for, having put away limitation, you have become all. Notwithstanding you were all even before: nay, even something was added to you besides the all. And you became less by the addition, because the addition was not of being. For to it you can add nothing. When, therefore, space is produced from non-being, it is accompanied with poverty, and is lacking in all things. When, however, it puts away non-being, then it is itself all fulness of self. So that [...]\footnote{Lacuna in the text.} recovers itself, putting away the things that have degraded and belittled it; and particularly when one supposes himself to be the things which are small by their nature, and not what he is in truth. He revolted from himself at the same time when he revolted from being. But when anyone is present to his present self, then he is present to being which is everywhere. And when he let go himself he revolted also from it. Of such value is it to be present with that which is present in self, and absent from that which is outside of self. If being is present with us, non-being is absent; but while we are with other things, it is not present with us. It did not come to be present; but we depart when it is not present. And what is strange in this ? For you, by being present, are not absent from self; and [yet] you are not present with self, though present; present and absent being the same when you look at other things, and neglect to look at yourself. And, if thus, while present to self, you are not present, and for this reason are ignorant of yourself, and discover those things which are present to you. and yet far from you, rather than the self which is by nature present to you, why do you wonder if the not present is far from you, who have become far from it, by becoming far from yourself? For the more you belong to yourself, although present and undisjoined (for self is in proportion as it belongs to itself), the more you will belong to it, which is thus indeed inseparable from you in its essence as you are from yourself. Thus it is in your power to know exhaustively what is present to being and what is absent from being, which is present everywhere and again is nowhere. For to those who are able to retire intellectually into their own essence, and to cognize their own essence, and in this cognition and knowledge of cognition to recover themselves in the identity of the cognizing and the cognized, to those, being present with themselves, being also is present. But from those who go beyond their own being to other things, inasmuch as they are absent from themselves, being also is absent. And although we are so constituted as to be placed in the same essence, and to enrich ourselves from ourselves, and not to depart to that which we are not, and to be impoverished of ourselves, and through these again to be mated with poverty, even when self is present; yet, while we are not separated from being either by place or essence, or cut off from it by anything else, we separate ourselves from it by adversion to non-being, and accordingly pay this penalty, that, by turning away from being, we turn away from ourselves and are ignorant of ourselves. On the other hand, by adverting to love of ourselves, we recover ourselves and are united to God. Hence it has been well said that the soul is enclosed in the body as in a kind of prison r and that it is there bound with chains as runaway slaves are wont to be. Now it must strive to loose itself from these chains. Inasmuch as it has turned to the here, and deserted its divine self, it is, as Plato says, a fugitive and a wanderer from God. Every evil life is full of slavery, and hence is godless and unjust. There is in it a spirit full of impiety, and hence of unrighteousness. Wherefore it has rightly been said that by self-determination it finds the just, and that in awarding to each of one’s fellows his due lies an image and shadow of true righteousness.

\paragraph{42} That which has its being in another, and which has no essence in itself apart from another, if it turns to itself in order to cognize itself apart from that in which it has its essence, separating itself from that, is itself corrupted and destroyed, inasmuch as it withdraws itself from being. But that which is capable of knowing itself without that in which it is, when it recovers itself from itself, and is capable of doing this without destruction of itself, cannot possibly have its essence in that from which it is able to turn itself away without destruction of itself, and to cognize itself without that. Now if seeing and all perceptive power, neither is a perception of self, nor when separating itself from the body, lays hold of itself, or is preserved; while intellect, when separating itself from the body, then acts most intellectually, and turns to itself and is not destroyed; it is plain that the perceptive powers possess their energy through the body, whereas the intellect does not possess its energy or its being in the body, but in itself.

\paragraph{43} Incorporeal things are named and thought accurately by negation of body, as matter, according to the ancients, and the form of matter, when thought separate from matter, and natures and powers; so also space, and time, and limits. For all these things are named by negation of body. But besides these there are other things which are by an abuse of language called incorporeal, not from negation of body, but because it is altogether contrary to their nature to beget body. Wherefore that which has reference to what is first indicated, exists in relation to bodies; whereas those which have reference to the second, are completely distinct from bodies, and from the incorporeal things which have relation to bodies. For bodies are in space, and limits are in body. But intellect and intellectual reason exist neither in space nor in body. They neither immediately posit body, nor are posited by body, or by the things which are called incorporeal from negation of body. And though, for example, a void can be thought as incorporeal, it is not possible for intellect to be in a vo(,d. For the void would be receptive of body. But it is impossible to separate energy from intellect, or to give space to energy. As the \textit{Genus} appears double, one phase of it the disciples of Zeno do not accept at all; whereas, accepting the other, and observing that the first is not similar, they cancel it, instead of regarding it, as they ought to do, as a different genus. Because it is not the other, they ought not to regard it as not being at all.

\paragraph{44} Intellect and intelligible are one thing, perception and perceptible another. With intellect is correlated the intelligible, and with perception the perceptible. But neither perception seizes itself by itself, nor does the perceptible. And the intelligible being the correlate of intellect, the intelligible also falls in the sphere of intellect, and by no means under that of perception. But the intelligible falls in the sphere of intellect. If, therefore, intellect is intelligible and not perceptible, it must be an intelligible. And if it is intelligible by intellect and not by perception, it must be an intelligent. The same [intellect], therefore, which is the subject and object of intelligence is the whole of a whole, and does not stand in the relation of rubber and rubbed. It is not, therefore, thought with one part, while it thinks with another [i.~e.~one part is not the object, another the subject of intelligence]. For it is indivisible, and the whole is intelligible to the whole. And it is intellect throughout, having no knowledge of absence of intelligence in itself. Hence, not one part of it thinks, while the other does not think; for in so far as it does not think, it will be unintelligible. Nor does it withdraw from one thing and pass to another thing. For that from which it withdraws, it ceases to be able to think. But if it is not true that one thing after another is within its range, it thinks all things at once. Since, therefore, it thinks all things at once, and not one thing now and another then, it thinks all things now and forever at once [i.~e.~in one eternal \textit{now}]. If, therefore, the now belongs to it, the past and the future are removed from it, in this spaceless, present, timeless self-possession. So that the together (simultaneity), as regards both multitude and temporal distance, belong to it. Wherefore all things are as one, and in one---a One spaceless, timeless. This being the case, there is also no whence [or] whither for intellect, and hence also no movement; but energy as one, in one, free from increase and change and all evolution. But if multiplicity is as one, and energy is also timeless, a subordinate attribute of such an essence must be being-always-in-one. But this is eternity. Wherefore an attribute of intellect is eternity. But to that which does not think as one in one, but discursively and in movement, and in leaving one thing and seizing another, and in dividing and going-beyond, there belongs the attribute of time. For such movement presupposes a future and a past. Soul, for example, passes from one thing to another, taking up concept after concept, not because the first disappear, or that the second introduce themselves from elsewhere; but the one set have as it were departed while they remain in it, and others as it were are succeeding from elsewhere. But they do not come from without, but from its self-movement of itself into itself, and its passing of its eye over the things which it has part by part. For it is like a fountain whose waters do not flow away, but which spouts up what it contains in a circle into itself. Now the movement of this presupposes time; whereas the enduring of intellect in itself demands eternity, not separated from it as time is from soul. In the former the presuppositions are united. That which moves counterfeits eternity, by the mere immeasurableness of its movement producing an impression of eternity. And also that which endures, in relation to that which moves, counterfeits time, going beyond and multiplying as it were its now-there (everlasting present there) in imitation of time. For this reason, some have thought that time could be regarded as at rest, no less than as in motion, and eternity, as we have said, as infinite time. Thus the one imparts its own properties (conditions) to the other, the moving always copying eternity from the stable, as if eternity were identical with its own Always (unceasing duration); and the stable, in the identity of its energy, connecting time with its own enduring, from the energy. Further, in perceptible things, distinct time is one for one thing, another for another. For example, it is one for the Sun, another for the Moon, another for Lucifer,\footnote{The ``morning star,'' i.~e.~Venus.} and so forth. Hence one has one year, another another. And the year that includes these is consummated in the motion of the soul, inasmuch as all other things move in imitation of it. The movement of it being different from the movement of these, the time also of it is different from the time of these. The latter is extended both as regards locomotion and transition.

\end{document}
