% pdflatex
\documentclass[a4paper,12pt]{article}
\usepackage{ebgaramond}
\usepackage{microtype}

\title{On the Cave of the Nymphs\\in the Thirteenth Book of the \textit{Odyssey}}
\author{Porphyry \and Thomas Taylor (tr.)\footnote{T.~Taylor, \textit{Select Works of Porphyry,} 1823.}}
\date{}

\begin{document}
\maketitle


\paragraph{1} What does Homer obscurely signify by the cave in Ithaca, which he
describes in the following verses?

\begin{verse}
``High at the head a branching olive grows,\\
And crowns the pointed cliffs with shady boughs.\\
A cavern pleasant, though involv'd in night,\\
Beneath it lies, the Naiades' delight:\\
Where bowls and urns of workmanship divine\\
And massy beams in native marble shine;\\
On which the Nymphs amazing webs display,\\
Of purple hue, and exquisite array.\\
The busy bees within the urns secure\\
Honey delicious, and like nectar pure.\\
Perpetual waters through the grotto glide,\\
A lofty gate unfolds on either side;\\
That to the north is pervious to mankind;\\
The sacred south t' immortals is consign'd.''
\end{verse}

\noindent That the poet, indeed, does not narrate these particulars from
historical information, is evident from this, that those who have given us a
description of the island, have, as Cronius\footnote{This Cronius, the
Pythagorean, is also mentioned by Porphyry, in his \textit{Life of Plotinus}.}
says, made no mention of such a cave being found in it. This likewise, says he,
is manifest, that it would be absurd for Homer to expect, that in describing a
cave fabricated merely by poetical license and thus artificially opening a path
to Gods and men in the region of Ithaca, he should gain the belief of mankind.
And it is equally absurd to suppose, that nature herself should point out, in
this place, one path for the descent of all mankind, and again another path for
all the Gods. For, indeed, the whole world is full of Gods and men; but it is
impossible to be persuaded, that in the Ithacensian cave men descend, and Gods
ascend. Cronius therefore, having premised this much, says, that it is
evident, not only to the wise but also to the vulgar, that the poet, under the
veil of allegory, conceals some mysterious signification; thus compelling
others to explore what the gate of men is and also what is the gate of the
Gods: what he means by asserting that this cave of the Nymphs has two gates;
and why it is both pleasant and obscure, since darkness is by no means
delightful, but is rather productive of aversion and horror. Likewise, what is
the reason why it is not simply said to be the cave of the Nymphs, but it is
accurately added, of the Nymphs which are called Naiades? Why also, is the cave
represented as containing bowls and amphor{\ae}, when no mention is made of
their receiving any liquor, but bees are said to deposit their honey in these
vessels as in hives? Then, again, why are oblong beams adapted to weaving
placed here for the Nymphs; and these not formed from wood, or any other
pliable matter, but from stone, as well as the amphor{\ae} and bowls? Which
last circumstance is, indeed, less obscure; but that, on these stony beams, the
Nymphs should weave purple garments, is not only wonderful to the sight, but
also to the auditory sense. For who would believe that Goddesses weave
garments in a cave involved in darkness, and on stony beams; especially while
he hears the poet asserting, that the purple webs of the Goddesses were
visible. In addition to these things likewise, this is admirable, that the cave
should have a twofold entrance; one made for the descent of men, but the other
for the ascent of Gods. And again that the gate, which is pervious by men,
should be said to be turned against the north wind, but the portal of the Gods
to the south; and why the poet did not rather make use of the west and the east
for this purpose, since nearly all temples have their statues and entrances
turned towards the east; but those who enter them look towards the west, when
standing with their faces turned towards the statues they honour and worship
the Gods. Hence, since this narration is full of such obscurities it can
neither be a fiction casually devised for the purpose of procuring delight, nor
an exposition of a topical history; but something allegorical must be indicated
in it by the poet who likewise mystically places an olive near the cave. All
which particulars the ancients thought very laborious to investigate and
unfold; and we, with their assistance, shall now endeavour to develop the
secret meaning of the allegory. Those persons, therefore, appear to have
written very negligently about the situation of the place, who think that the
cave, and what is narrated concerning it, are nothing more than a notion of the
poet. But the best and most accurate writers of geography, and among these
Artemidorus the Ephesian, in the fifth book of his work, which consists of
eleven books, thus writes: ``The island of Ithaca, containing an extent of
eighty-five stadia,\footnote{\textit{I.~e.,} rather more than ten Italian miles
and a half, eight stadia making an Italian mile.} is distant from Panormus, a
port of Cephalenia, about twelve stadia. It has a port named Phorcys, in which
there is a shore, and on that shore a cave, in which the Ph{\ae}acians are
reported to have placed Ulysses.'' This cave, therefore, will not be entirely
an Homeric fiction. But whether the poet describes it as it really is, or
whether he has added something to it of his own invention, nevertheless the
same inquiries remain; whether the intention of the poet is investigated, or of
those who founded the cave. For, neither did the ancients establish temples
without fabulous symbols, nor does Homer rashly narrate the particulars
pertaining to things of this kind. But how much the more anyone endeavours to
show that this description of the cave is not an Homeric fiction, but prior to
Homer was consecrated to the Gods, by so much the more will this consecrated
cave be found to be full of ancient wisdom. And on this account it deserves to
be investigated, and it is requisite that its symbolical consecration should be
amply unfolded into light.


\paragraph{2} The ancients, indeed, very properly consecrated a cave to the
world, whether assumed collectively, according to the whole of itself, or
separately, according to its parts. Hence they considered earth as a symbol of
that matter of which the world consists; on which account some thought that
matter and earth are the same; through the cave indicating the world, which was
generated from matter. For caves are, for the most part, spontaneous
productions, and connascent with the earth, being comprehended by one uniform
mass of stone; the interior parts of which are concave, but the exterior parts
are extended over an indefinite portion of land. And the world being
spontaneously produced (\textit{i.~e.,} being produced by no external, but from
an internal cause), and being also self-adherent, is allied to matter; which,
according to a secret signification, is denominated a stone and a rock, on
account of its sluggish and repercussive nature with respect to form; the
ancients, at the same time, asserting that matter is infinite through its
privation of form. Since, however, it is continually flowing, and is of itself
destitute of the supervening investments of form, through which it participates
of \textit{morphe,}\footnote{\textit{Morphe,} as we are informed by Simplicius,
pertains to the colour, figure, and magnitude of superficies.} and becomes
visible, the flowing waters, darkness, or, as the poet says, obscurity of the
cavern. were considered by the ancients as apt symbols of what the world
contains, on account of the matter with which it is connected. Through matter,
therefore, the world is obscure and dark; but through the connecting power, and
orderly distribution of form, from which also it is called world, it is
beautiful and delightful. Hence it may very properly be denominated a cave; as
being lovely, indeed, to him who first enters into it, through its
participation of forms, but obscure to him who surveys its foundation and
examines it with an intellectual eye. So that its exterior and superficial
parts, indeed, are pleasant, but its interior and profound parts are obscure
(and its very bottom is darkness itself). Thus also the Persians, mystically
signifying the descent of the soul into the sublunary regions, and its
regression from it, initiate the mystic (or him who is admitted to the arcane
sacred rites) in a place which they denominate a cavern. For, as Eubulus says,
Zoroaster was the first who consecrated in the neighbouring mountains of
Persia, a spontaneously produced cave, florid, and having fountains, in honour
of Mithra, the maker and father of all things; a cave, according to Zoroaster,
bearing a resemblance of the world, which was fabricated by Mithra. But the
things contained in the cavern being arranged according to commensurate
intervals, were symbols of the mundane elements and climates.


\paragraph{3} After this Zoroaster likewise, it was usual with others to
perform the rites pertaining to the mysteries in caverns and dens, whether
spontaneously produced, or made by the hands. For as they established temples,
groves, and altars to the celestial Gods, but to the terrestrial Gods, and to
heroes, altars alone, and to the subterranean divinities pits and cells; so to
the world they dedicated caves and dens; as likewise to
Nymphs,\footnote{\-``Nymphs,'' says Hermias in his \textit{Scholia on the
Ph{\ae}drus of Plato}, ``are Goddesses who preside over regeneration, and are
ministrant to Bacchus, the offspring of Semele. Hence they dwell near water,
that is, they are conversant with generation. But this Bacchus supplies the
regeneration of the whole sensible world.''} on account of the water which
trickles, or is diffused in caverns, over which the Naiades, as we shall
shortly observe, preside. Not only, however, did the ancients make a cavern, as
we have said, to be a symbol of the world, or of a generated and sensible
nature: but they also assumed it as a symbol of all invisible powers; because
as caverns are obscure and dark, so the essence of these powers is occult.
Hence Saturn fabricated a cavern in the ocean itself and concealed in it his
children. Thus, too, Ceres educated Proserpine with her Nymphs in a cave; and
many other particulars of this kind may be found in the writings of
theologists. But that the ancients dedicated caverns to Nymphs and especially
to Naiades, who dwell, near fountains, and who are called Naiades from the
streams over which they preside, is manifest from the hymn to Apollo, in which
it is said: ``The Nymphs residing in caves shall deduce fountains of
intellectual waters to thee (according to the divine voice of the Muses), which
are the progeny of a terrene spirit. Hence waters, bursting through every
river, shall exhibit to mankind perpetual effusions of sweet
streams.\footnote{These lines are not to be found in any of the hymns now
extant, ascribed to Homer.}'' From hence, as it appears to me. the
Pythagoreans. and after them Plato, showed that the world is a cavern and a
den. For the powers which are the leaders of souls, thus speak in a verse of
Empedocles:

\begin{verse}
``Now at this secret cavern we're arrived.''
\end{verse}

\noindent And by Plato, in the seventh book of his \textit{Republic}, it is
said, ``Behold men as if dwelling in a subterraneous cavern, and in a denlike
habitation, whose entrance is widely expanded to the admission of the light
through the whole cave.'' But when the other person in the dialogue says: ``You
adduce an unusual and wonderful similitude,'' he replies, ``The whole of this
image, friend Glauco, must be adapted to what has been before said,
assimilating this receptacle, which is visible through the sight to the
habitation of a prison; but the light of the fire which is in it to the power
of the Sun.''


\paragraph{4} That theologists therefore considered caverns as symbols of the
world, and of mundane powers, is through this, manifest. And it has been
already observed by us, that they also considered a cave as a symbol of the
intelligible essence; being impelled to do so by different and not the same
conceptions. For they were of opinion that a cave is a symbol of the sensible
world because caverns are dark, stony, and humid; and they asserted that the
world is a thing of this kind, through the matter of which it consists, and
through its repercussive and flowing nature. But they thought it to be a symbol
of the intelligible world, because that world is invisible to sensible
perception, and possesses a firm and stable essence. Thus, also, partial powers
are unapparent, and especially those which are inherent in matter. For they
formed these symbols, from surveying the spontaneous production of caves, and
their nocturnal, dark, and stony nature; and not entirely, as some suspect,
from directing their attention to the figure of a cavern. For every cave is not
spherical, as is evident from this Homeric cave with a twofold entrance. But
since a cavern has a twofold similitude, the present cave must not be assumed
as an image of the intelligible but of the sensible essence. For in consequence
of containing perpetually flowing streams of water, it will not be a symbol of
an intelligible hypostasis, but of a material essence. On this account also it
is sacred to Nymphs, not the mountain or \textit{rural Nymphs,} or others of
the like kind, but to the Naiades, who are thus denominated from streams of
water. For we peculiarly call the Naiades, and the powers that preside over
waters, Nymphs; and this term also, is commonly applied to all souls descending
into generation. For the ancients thought that these souls are incumbent on
water which is inspired by divinity, as Numenius says, who adds, that on this
account, a prophet asserts, that the Spirit of God moved on the waters. The
Egyptians likewise, on this account, represent all d{\ae}mons and also the Sun,
and, in short, all the planets, not standing on anything solid, but on a
sailing vessel; for souls descending into generation fly to moisture. Hence
also, Heraclitus says, that moisture appears delightful and not deadly to
souls; but the lapse into generation is delightful to them. And in another
place [speaking of unembodied souls], he says, ``We live their death, and we
die their life.'' Hence the poet calls those that are in generation
\textit{humid,} because they have souls which are \textit{profoundly} steeped
in moisture. On this account, such souls delight in blood and humid seed; but
water is the nutriment of the souls of plants. Some likewise are of opinion,
that the bodies in the air, and in the heavens, are nourished by vapours from
fountains and rivers, and other exhalations. But the Stoics assert, that the
Sun is nourished by the exhalation from the sea; the Moon from the vapours of
fountains and river; and the stars from the exhalation of the earth. Hence,
according to them, the Sun is an intellectual composition formed from the sea;
the Moon from the river waters and the stars from terrene exhalations.


\paragraph{5} It is necessary, therefore, that souls, whether they are
corporeal or incorporeal, while they attract to themselves body, and especially
such as are about to be bound to blood and moist bodies, should verge to
humidity, and be corporalised, in consequence of being drenched in moisture.
Hence the souls of the dead are evocated by the effusion of bile and blood; and
souls that are lovers of body, by attracting a moist spirit, condense this
humid vehicle like a cloud. For moisture condensed in the air constitutes a
cloud. But the pneumatic vehicle being condensed in these souls, becomes
visible through an excess of moisture. And among the number of these we must
reckon those apparitions of images, which, from a spirit coloured by the
influence of imagination, present themselves to mankind. But pure souls are
averse from generation; so that, as Heraclitus says, \textit{``a dry soul is
the wisest.''} Hence, here also the spirit becomes moist and more aqueous
through the desire of generation, the soul thus attracting a humid vapour from
verging to generation. Souls, therefore, proceeding into generation are the
nymphs called naiades. Hence it is usual to call those that are married nymphs,
as being conjoined to generation, and to pour water into baths from fountains,
or rivers, or perpetual rills.


\paragraph{6} This world, then, is sacred and pleasant to souls who nave now
proceeded into nature, and to natal d{\ae}mons, though it is essentially dark
and \textit{obscure;} from which some have suspected that souls also are of an
\textit{obscure} nature, and essentially consist of air. Hence a cavern, which
is both pleasant and dark, will be appropriately consecrated to souls on the
earth, conformably to its similitude to the world, in which, as in the greatest
of all temples, souls reside. To the nymphs likewise, who preside over waters,
a cavern, in which there are perpetually flowing streams, is adapted. Let,
therefore, this present cavern be consecrated to souls, and among the more
partial powers, to nymphs that preside over streams and fountains, and who, on
this account, are called \textit{fontal} and \textit{Naiades.} What, therefore,
are the different symbols, some of which are adapted to souls, but others to
the aquatic powers, in order that we may apprehend that this cavern is
consecrated in common to both? Let the stony bowls, then, and the amphor{\ae}
be symbols of the aquatic nymphs. For these are, indeed, the symbols of
Bacchus, but their composition is fictile, \textit{i.~e.}~consists of baked
earth, and these are friendly to the vine, the gift of God; since the fruit of
the vine is brought to a proper maturity by the celestial fire of the Sun. But
the stony bowls and amphor{\ae} are in the most eminent degree adapted to the
nymphs who preside over the water that flows from rocks. And to souls that
descend into generation and are occupied in corporeal energies, what symbol can
be more appropriate than those instruments pertaining to weaving? Hence, also,
the poet ventures to say, ``that on these, the nymphs weave purple webs,
admirable to the view.'' For the formation of the flesh is on and about the
bones, which in the bodies of animals resemble stones. Hence these instruments
of weaving consist of stone, and not of any other matter. But the purple webs
will evidently be the flesh which is woven from the blood. For purple woolen
garments are tinged from blood. and wool is dyed from animal juice. The
generation of flesh, also, is through and from blood. Add, too, that the body
is a garment with which the soul is invested, a thing wonderful to the sight,
whether this refers to the composition of the soul, or contributes to the
colligation of the soul [to the whole of a visible essence]. Thus, also,
Proserpine, who is the inspective guardian of everything produced from seed, is
represented by Orpheus as weaving a web,\footnote{The theological meaning of
this Orphic fiction is beautifully unfolded by Proclus as follows: ``Orpheus
says that the vivific cause of partible natures [\textit{i.~e.}~Proserpine],
while she remained on high, weaving the order of celestials, was a nymph, as
being undefiled; and in consequence of this connected with Jupiter and abiding
in her appropriate manners; but that, proceeding from her proper habitation,
she left her webs unfinished, was ravished; having been ravished, was married;
and that being married, she generated in order that she might animate things
which have an adventitious life. For the unfinished state of her web indicates,
I think, that the universe is imperfect or unfinished, as far as to perpetual
animals [\textit{i.~e.}~the universe would be imperfect if nothing inferior to
the celestial Gods was produced]. Hence Plato says, that the one Demiurgus
calls on the many Demiurgi to weave together the mortal and immortal natures;
after a manner reminding us, that the addition of the mortal genera is the
perfection of the textorial life of the universe, and also exciting our
recollection of the divine Orphic fable, and affording us interpretative causes
of the unfinished webs of Proserpine.'' See vol.~\textsc{ii} p.~356, of my
translation of Proclus \textit{on the Tim{\ae}us}.

The \textit{unfinished webs} of Proserpine are also alluded to by Claudian in
his poem \textit{De Raptu Proserpin{\ae}}, in the following verse:

\begin{verse}
``Sensit adesse Deas, \textit{imperfectumque laborem Deserit.}''
\end{verse}

I only add, that, by ancient theologists, the shuttle was considered as a
signature of \textit{separating,} a cup of \textit{vivific,} a sceptre of
\textit{ruling,} and a key of \textit{guardian power.}} and the heavens are
called by the ancients a veil, in consequence of being, as it were, the
vestment of the celestial Gods.


\paragraph{7} Why, therefore, are the amphor{\ae} said not to be filled with
water, but with honeycombs? For in these, Homer says, the bees deposit their
honey. And honey is the nutriment of bees. Theologists also have made honey
subservient to many and different symbols because it consists of many powers;
since it is both cathartic and preservative. Hence, through honey, bodies are
preserved from putrefaction, and inveterate ulcers are purified. Farther still,
it is also sweet to the taste, and is collected by bees, who are ox-begotten
from flowers. When, therefore, those who are initiated in the Leontic sacred
rites, pour honey instead of water on their hands; they are ordered [by the
initiator] to have their hands pure from everything productive of molestation,
and from everything noxious and detestable. Other initiators [into the same
mysteries] employ fire, which is of a cathartic nature, as an appropriate
purification. And they likewise purify the tongue from all defilement of evil
with honey. But the Persians, when they offer honey to the guardian of fruits,
consider it as the symbol of a preserving and defending power. Hence some
persons have thought that the nectar and ambrosia,\footnote{The theological
meaning of nectar and ambrosia is beautifully unfolded by Hermias, in his
\textit{Scholia on the Ph{\ae}drus of Plato}, where he informs us, ``that
\textit{ambrosia} is analogous to dry nutriment, and that on this account it
signifies an establishment in causes: but that \textit{nectar} is analogous to
moist food, and that it signifies the providential attention of the Gods to
secondary natures; the former being denominated, according to \textit{a
privation of the mortal and corruptible;} but the latter, according to
\textit{a privation of the funeral and sepulchral.} And when the Gods are
represented as energising; providentially, they are said to drink nectar. Thus
Homer in the beginning of the 4th book of the \textit{Iliad}:

\begin{verse}
``Now with each other, on the golden floor,\\
Seated near Jove, the Gods converse; to whom\\
The venerable Hebe nectar bears\\
In golden goblets; and as these flow round\\
Th' immortals turn their careful eyes on Troy.''
\end{verse}

\noindent For then they providentially attend to the Trojans. The possession,
therefore, of immutable providence by the Gods is signified by their drinking
nectar; the exertion of this providence, by their beholding Troy, and their
communicating with each other in providential energies, by receiving the
goblets from each other.} which the poet pours into the nostrils of the dead,
for the purpose of preventing putrefaction, is honey; since honey is the food
of the Gods. On this account also, the same poet somewhere calls nectar golden;
for such is the colour of honey, [\textit{viz.}, it is a deep yellow]. But
whether or not honey is to be taken for nectar, we shall elsewhere more
accurately examine. In Orpheus, likewise, Saturn is ensnared by Jupiter through
honey. For Saturn, being filled with honey, is intoxicated, his senses are
darkened, as if from the effects of wine, and he sleeps; just as Porus, in the
banquet of Plato, is filled with nectar; for wine was not (says he) yet known.
The Goddess Night, too, in Orpheus, advises Jupiter to make use of honey as an
artifice. For she says to him:

\begin{verse}
``When stretch'd beneath the lofty oaks you view\\
Saturn, with honey by the bees produc'd\\
Sunk in ebriety,\footnote{Ebriety, when ascribed to divine natures by ancient
theologists, signifies a deific superessential energy, or an energy superior to
intellect. Hence, when Saturn is said by Orpheus to have been intoxicated with
honey or nectar, the meaning is, that he then energised providentially, in a
deific and super-intellectual manner.} fast bind the God.''
\end{verse}

\noindent This therefore, takes place, and Saturn being bound is emasculated in
the same manner as Heaven; the theologist obscurely signifying by this that
divine natures become through pleasure bound, and drawn down into the realms of
generation; and also that, when dissolved in pleasure they emit certain seminal
powers. Hence Saturn emasculates Heaven, when descending to earth through a
desire of coition.\footnote{Porphyry, though he excelled in philosophical, was
deficient in theological knowledge; of which what he now says of the
castrations of Saturn and Heaven is a remarkable instance. For ancient
theologists, by things preternatural, adumbrated the transcendent nature of the
Gods; by such as are irrational, a power more divine than all reason; and by
things apparently base, incorporeal beauty. Hence in the fabulous narrations to
which Porphyry now alludes, the genital parts must be considered as symbols of
prolific power; and the castration of these parts as signifying the progression
of this power into a subject order. So that the fable means that the prolific
powers of Saturn are called forth into progression by Jupiter, and those of
Heaven by Saturn; Jupiter being inferior to Saturn, and Saturn to Heaven. See
the \textit{Apology for the Fables of Homer} in Vol.~\textsc{i} of my
translation of Plato.} But the sweetness of honey signifies, with theologists,
the same thing as the pleasure arising from generation, by which Saturn, being
ensnared, was castrated. For Saturn, and his sphere, are the first of the orbs
that move contrary to the course of Coelum or the heavens. Certain powers,
however, descend both from Heaven [or the inerratic sphere] and the planets.
But Saturn receives the powers of Heaven and Jupiter the powers of Saturn.
Since, therefore, honey is assumed in purgations, and as an antidote to
putrefaction, and is indicative of the pleasure which draws souls downward to
generation; it is a symbol well adapted to aquatic Nymphs, on account of the
unputrescent nature of the waters over which they preside, their purifying
power, and their co-operation with generation. For water co-operates in the
work of generation. On this account the bees are said, by the poet, to deposit
their honey in bowls and amphor{\ae}; the bowls being a symbol of fountains,
and therefore a bowl is placed near to Mithra, instead of a fountain; but the
amphor{\ae} are symbols of the vessels with which we draw water from fountains.
And fountains and streams are adapted to aquatic Nymphs, and still more so to
the Nymphs that are souls, which the ancient peculiarly called bees, as the
efficient causes of sweetness. Hence Sophocles does not speak unappropriately
when he says of souls,

\begin{verse}
``In swarms while wandering, from the dead,\\
A humming sound is heard.''
\end{verse}


\paragraph{8} The priestesses of Ceres, also, as being initiated into the
mysteries of the terrene Goddess, were called by the ancients bees; and
Proserpine herself was denominated by them \textit{honied.} The Moon, likewise,
who presides over generation, was called by them a bee, and also a bull. And
Taurus is the exaltation of the Moon. But bees are ox-begotten. And this
application is also given to souls proceeding into generation. The God,
likewise, who is occultly connected with generation, is a stealer of oxen. To
which may be added, that honey is considered as a symbol of death, and on this
account it is usual to offer libations of honey to the terrestrial Gods; but
gall is considered as a symbol of life; whether it is obscurely signified by
this, that the life of the soul dies through pleasure, but through bitterness
the soul resumes its life, whence, also, bile is sacrificed to the Gods; or
whether it is, because death liberates from molestation, but the present life
is laborious and bitter. All souls, however, proceeding into generation, are
not simply called bees, but those who will live in it justly and who, after
having performed such things as are acceptable to the Gods, will again return
[to their kindred stars]. For this insect loves to return to the place from
whence it first came, and is eminently just and sober. Whence, also, the
libations which are made with honey are called sober. Bees, likewise, do not
sit on beans, which were considered by the ancients as a symbol of generation
proceeding in a right line, and without flexure; because this leguminous
vegetable is almost the only seed-bearing plant whose stalk is perforated
throughout without any intervening knots.\footnote{Hence, when Pythagoras
exhorted his disciples to abstain from beans, he intended to signify, that they
should beware of a continued and perpetual descent into the realms of
generation.} We must therefore admit, that honeycombs and bees are appropriate
and common symbols of the aquatic nymphs, and of souls that are married [as it
were] to [the humid and fluctuating nature of] generation.


\paragraph{9} Caves, therefore, in the most remote periods of antiquity were
consecrated to the Gods, before temples were erected to them. Hence, the
Curetes in Crete dedicated a cavern to Jupiter; in Arcadia, a cave was sacred
to the Moon, and to Lycean Pan; and in Naxus, to Bacchus. But wherever Mithra
was known, they propitiated the God in a cavern. With respect, however, to the
Ithacensian cave, Homer was not satisfied with saying that it had two gates,
but adds that one of the gates was turned towards the north, but the other
which was more divine, to the south. He also says that the northern gate was
pervious to descent, but does not indicate whether this was also the case with
the southern gate. For of this, he only says, ``It is inaccessible to men, but
it is the path of the immortals.''


\paragraph{10} It remains, therefore, to investigate what is indicated by this
narration; whether the poet describes a cavern which was in reality consecrated
by others, or whether it is an enigma of his own invention. Since, however, a
cavern is an image and symbol of the world, as Numenius and his familiar
Cronius assert, there are two extremities in the heavens, \textit{viz.}, the winter
tropic, than which nothing is more southern, and the summer tropic, than which
nothing is more northern. But the summer tropic is in Cancer, and the winter
tropic in Capricorn. And since Cancer is nearest to us, it is very properly
attributed to the Moon, which is the nearest of all the heavenly bodies to the
earth. But as the southern pole by its great distance is invisible to us, hence
Capricorn is attributed to Saturn, the highest and most remote of all the
planets. Again, the signs from Cancer to Capricorn are situated in the
following order: and the first of these is Leo, which is the house of the Sun;
afterwards Virgo, which is the house of Mercury; Libra, the house of Venus;
Scorpio, of Mars; Sagittarius, of Jupiter; and Capricorn, of Saturn. But from
Capricorn in an inverse order Aquarius is attributed to Saturn; Pisces to
Jupiter; Aries to Mars; Taurus to Venus; Gemini to Mercury; and in the last
place Cancer to the Moon.


\paragraph{11} Theologists therefore assert, that these two gates are Cancer
and Capricorn; but Plato calls them entrances. And of these, theologists say,
that Cancer is the gate through which souls descend; but Capricorn that through
which they ascend. Cancer is indeed northern, and adapted to descent; but
Capricorn is southern, and adapted to ascent.\footnote{Macrobius, in the
twelfth chapter of his \textit{Commentary on Scipio's Dream}, has derived some of the
ancient arcana which it contains from what is here said by Porphyry. A part of
what he has farther added, I shall translate on account of its excellence and
connexion with the above passage. ``Pythagoras thought that the empire of Pluto
began downwards from the Milky Way, because souls falling from thence appear to
have already receded from the Gods. Hence he asserts that the nutriment of milk
is first offered to infants, because their first motion commences from the
galaxy, when they begin to fall into terrene bodies. On this account, since
those who are about to descend are yet in \textit{Cancer,} and have not left
the Milky Way, they rank in the order of the Gods. But when, by falling, they
arrive at the \textit{Lion,} in this constellation they enter on the exordium
of their future condition. And because, in the \textit{Lion,} the rudiments of
birth and certain primary exercises of human nature, commence; but
\textit{Aquarius} is opposite to the and presently sets after the \textit{Lion}
rises; hence, when the Sun is in \textit{Aquarius,} funeral rites are performed
to departed souls, because he is then carried in a sign which is contrary or
adverse to human life. From the confine, therefore, in which the zodiac and
galaxy touch each other, the soul, descending from a round figure, which is the
only divine form, is produced into a cone by its denuxion. And as a line is
generated from a point and proceeds into length from an indivisible, so the
soul, from its own point, which is a monad, passes into the dyad, which is the
first extension. And this is the essence which Plato, in the \textit{Tim{\ae}us}, calls
impartible and at the same time partible, when he speaks of the nature of the
mundane soul. For as the soul of the world, so likewise that of man, will be
found to be in one respect without division, if the simplicity of a. divine
nature is considered; and in another respect partible, if we regard the
diffusion of the former through the world, and of the latter through the
members of the body.

``As soon, therefore, as the soul gravitates towards body in this first
production of herself, she begins to experience a material tumult, that is,
matter flowing into her essence. And this is what Plato remarks in the
Ph{\ae}do, that the soul is drawn into body staggering with recent
intoxication; signifying by this the new drink of matter s impetuous flood,
through which the soul, becoming denied and heavy, is drawn into a terrene
situation. But the starry \textit{cup} placed between Cancer and the Lion is a
symbol of this mystic truth, signifying that descending souls first experience
intoxication in that part of the heavens throught the influx of matter. Hence
oblivion, the companion of intoxication, there begins silently to creep into
the recesses of the soul. For if souls retained in their descent to bodies the
memory of divine concerns, of which they were conscious in the heavens, there
would be no dissension among men about divinity. But all, indeed, in
descending, drink of oblivion; though some more, and others less. On this
account, though truth is not apparent to all men on the earth, yet all exercise
their opinions about it; because a \textit{defect of memory is the origin of
opinion.} But those discover most who have drunk least of oblivion, because
they easily remember what they had known before in the heavens.

``The soul, therefore, falling with this first weight from the zodiac and milky
way into each of the subject spheres, is not only clothed with the accession of
a luminous body, but produces the particular motions which it is to exercise in
the respective orbs. Thus in Saturn it energises according to a ratiocinative
and intellective power; in the sphere of Jove, according to a practic power; in
the orb of the Sun, according to a sensitive and imaginative nature; but
according to the motion of desire in the planet of Venus; of pronouncing and
interpreting what it perceives in the orb of Mercury; and according to a
plantal or vegetable nature and a power of acting on body, when it enters into
the lunar globe. And this sphere, as it is the last among the divine orders, so
it is the first in our terrene situation. For this body, as it is the dregs of
divine natures, so it is the first animal substance. And this is the difference
between terrene and supernal bodies (under the latter of which I comprehend the
heavens, the stars, and the more elevated elements), that the latter are called
upwards to be the seat of the soul, and merit immortality from the very nature
of the region and an imitation of sublimity; but the soul is drawn down to
these terrene bodies, and is on this account said to die when it is enclosed in
this fallen region, and the seat of mortality. Nor ought it to cause any
disturbance that we have so often mentioned the death of the soul, which we
have pronounced to be immortal. For the soul is not extinguished by its own
proper death, but is only overwhelmed for a time. Nor does it lose the benefit
of perpetuity by its temporal demersion. Since, when it deserves to be purified
from the contagion of vice, through its entire refinement from body, it will be
restored to the light of perennial life, and will return to its pristine
integrity and perfection.''

The powers, however, of the planets, which are the causes of the energies of
the soul in the several planetary spheres, are more accurately described by
Proclus in his admirable \textit{Commentary on the Tim{\ae}us}, as follows:
``If you are willing, also, you may say that of the beneficent planets the Moon
is the cause to Mortals of nature, being herself the visible statue of fontal
nature. But the Sun is the Demiurgus of everything sensible, in consequence of
being the cause of sight and visibility. Mercury is the cause of the motions of
the phantasy; for of the imaginative essence itself, so far as sense and
phantasy are one, the Sun is the producing cause. But Venus is the cause of
epithymetic appetites [or of the appetites pertaining to desire]; and Mars, of
the irascible motions which are conformable to nature. Of all vital powers,
however, Jupiter is the common cause; but of all gnostic powers, Saturn. For
all the irrational forms are divided into these.''} The northern parts,
likewise, pertain to souls descending into generation. And the gates of the
cavern which are turned to the north are rightly said to be pervious to the
descent of men; but the southern gates are not the avenues of the Gods, but of
souls ascending to the Gods. On this account, the poet does not say that they
are the avenues of the Gods, but of immortals; this appellation being also
common to our souls, which are \textit{per se,} or essentially, immortal. It is
said that Parmenides mentions these two gates in his treatise \textit{On the
Nature of Things}, as likewise that they are not unknown to the Romans and
Egyptians. For the Romans celebrate their Saturnalia when the Sun is in
Capricorn, and during this festivity, slaves wear the shoes of those that are
free, and all things are distributed among them in common; the legislator
obscurely signifying by this ceremony that through this gate of the heavens,
those who are now born slaves will be liberated through the Saturnian festival,
and the house attributed to Saturn, \textit{i.~e.,} Capricorn, when they live
again and return to the fountain of life. Since, however, the path from
Capricorn is adapted to ascent, hence the Romans denominate that month in which
the Sun, turning from Capricorn to the east, directs his course to the north,
Januanus, or January, from \textit{janua,} a gate. But with the Egyptians, the
beginning of the year is not Aquarius, as with the Romans, but Cancer. For the
star Sothis, which the Greeks call the Dog, is near to Cancer. And the rising
of Sothis is the new Moon with them, this being the principle of generation to
the world. On this account, the gates of the Homeric cavern are not dedicated
to the east and west, nor to the equinoctial signs, Aries and Libra, but to the
north and south, and to those celestial signs which towards the south are most
southerly, and, towards the north are most northerly; because this cave was
sacred to souls and aquatic nymphs. But these places are adapted to souls
descending into generation, and afterwards separating themselves from it.
Hence, a place near to the equinoctial circle was assigned to Mithra as an
appropriate seat. And on this account he bears the sword of Aries, which is a
martial sign. He is likewise carried in the Bull, which is the sign of Venus.
For Mithra, as well as the Bull, is the Demiurgus and lord of
generation.\footnote{Hence Phanes, or Protogonus, who is the paradigm of the
universe, and who was absorbed by Jupiter, the Demiurgus, is represented by
Orpheus as having the head of a \textit{bull} among other heads with which he
is adorned. And in the Orphic hymn to him he is called \textit{bull-roarer.}}
But he is placed near the equinoctial circle, having the northern parts on his
right hand, and the southern on his left. They likewise arranged towards the
south the southern hemisphere because it is hot; but the northern hemisphere
towards the north, through the coldness of the north wind.


\paragraph{12} The ancients, likewise, very reasonably connected winds with
souls proceeding into generation, and again separating themselves from it,
because, as some think, souls attract a spirit, and have a pneumatic essence.
But the north wind is adapted to souls falling into generation; and, on this
account, the northern blasts refresh those who are dying, and when they can
scarcely draw their breath. On the contrary the southern gales dissolve life.
For the north wind, indeed, from its superior coldness, congeals (as it were
the animal life), and retains it in the frigidity of terrene generation. But
the south wind, being hot, dissolves this life, and sends it upward to the heat
of a divine nature. Since, however, our terrene habitation is more northern, it
is proper that souls which are born in it should be familiar with the north
wind; but those that exchange this life for a better, with the south wind. This
also is the cause why the north wind is, at its commencement, great; but the
south wind, at its termination. For the former is situated directly over the
inhabitants of the northern part of the globe, but the latter is at a great
distance from them; and the blast from places very remote, is more tardy than
from such as are near. But when it is coacervated, then it blows abundantly and
with vigour. Since, however, souls proceed into generation through the northern
gate, hence this wind is said to be amatory. For, as the poet says,

\begin{verse}
``Boreas, enamour'd of the sprightly train,\\
Conceal'd his godhead in a flowing mane.\\
With voice dissembled to his loves he neighed,\\
And coursed the dappled beauties o'er the mead;\\
Hence sprung twelve others of unrivalled kind,\\
Swift as their mother mares, and father wind.\footnote{\textit{Iliad},
lib.~\textsc{xx} v. 223, \&c.}''
\end{verse}

\noindent It is also said, that Boreas ravished Orithya,\footnote{This fable is
mentioned by Plato in the \textit{Ph{\ae}drus}, and is beautifully unfolded as
follows by Hermias, in his \textit{Scholia} on that dialogue: ``A twofold
solution may be given of this fable: one from history, more ethical; but the
other, transferring us [from parts] to wholes. And the former of these is as
follows: Orithya was the daughter of Erectheus, and the priestess of Boreas;
for each of the winds has a presiding deity, which the telestic art, or the art
pertaining to sacred mysteries, religiously cultivates. To this Orithya, then,
the God was so very propitious, that he sent the north wind for the safety of
the country; and besides this, he is said to have assisted the Athenians in
their naval battles. Orithya, therefore, becoming enthusiastic, being possessed
by her proper God Boreas, and no longer energising as a human being (for
animals cease to energise according to their own peculiarities, when possessed
by superior causes, died under the inspiring innuence, and thus was said to
have been ravished by Boreas. And this is the more ethical explanation of the
fable.

``But the second, which transfers the narration to wholes, and does not
entirely subvert the former, is the following, for divine fables often employ
transactions and histories, in subserviency to the discipline of wholes. It is
said, then, that Erectheus is the God that rules over the three elements, air,
water, and earth. Sometimes, however, he is considered as alone the ruler of
the earth, and sometimes as the presiding deity of Attica alone. Of this deity
Orithya is the daughter; and she is the prolific power of the earth, which is,
indeed, co-extended with the word \textit{Erectheus,} as the unfolding of the
name signifies. For it is \textit{the prolific power of the earth, flourishing
and restored, according to the seasons.} But Boreas is the providence of the
Gods, supernally illuminating' secondary natures. For the providence of the
Gods in the world is signified by Boreas, because this divinity blows from
lofty places. And the elevating power of the Gods is signified by the south
wind, because this wind blows from low to lofty places; and besides this,
\textit{things situated towards the south are more divine.} The providence of
the Gods, therefore, causes the prolific power of the earth, or of the Attic
land, to ascend, and become visible.

``Orithya also may be said to be a soul aspiring after things above. Such a
soul, therefore, is ravished by Boreas supernally blowing. But if Orithya was
hurled from a precipice, this also is appropriate, for such a soul dies a
philosophic, not receiving a physical death, and abandons a life pertaining to
her own deliberate choice at the same time that she lives a physical life. And
philosophy, according to Socrates in the Ph{\ae}do, is nothing else than a
meditation of death.''} from whom he begot Zetis and Calais. But as the south
is attributed to the Gods, hence, when the Sun is at its meridian, the curtains
in temples are drawn before the statues of the Gods; in consequence of
observing the Homeric precept: ``That it is not lawful for men to enter temples
when the Sun is inclined to the south, for this is the path of the immortals.''
Hence, when the God is at his meridian altitude, the ancients placed a symbol
of midday and of the south in the gates of the temples, and on this account, in
other gates also, it was not lawful to speak at all times, because gates were
considered as sacred. Hence, too, the Pythagoreans, and the wise men among the
Egyptians, forbade speaking while passing through doors or gates; for then they
venerated in silence that God who is the principle of wholes (and, therefore,
of all things).


\paragraph{13} Homer, likewise, knew that gates are sacred, as is evident from
his representing {\OE}neus, when supplicating, shaking the gate:

\begin{verse}
``The gates he shakes, and supplicates the son.\footnote{\textit{Iliad}, lib.~\textsc{xi} v.~579.}''
\end{verse}

\noindent He also knew the gates of the heavens which are committed to the
guardianship of the hours; which gates originate in cloudy places, and are
opened and shut by the clouds. For he says:

\begin{verse}
``Whether dense clouds they close, or wide unfold.\footnote{\textit{Iliad}, lib.~\textsc{viii} v.~395.}''
\end{verse}

\noindent And on this account these gates omit a bellowing sound, because thunders roar through the clouds:

\begin{verse}
``Heaven's gates spontaneous open to the powers;\\
Heaven's bellowing portals, guarded by the Hours.\footnote{\textit{Iliad}, lib.~\textsc{viii} v.~393.}''
\end{verse}

\noindent He likewise elsewhere speaks of the gates of the Sun, signifying by
these Cancer and Capricorn, for the Sun proceeds as far as to these signs, when
he descends from the north to the south, and from thence ascends again to the
northern parts. But Capricorn and Cancer are situated about the galaxy, being
allotted the extremities of this circle; Cancer indeed the northern, but
Capricorn the southern extremity of it. According to Pythagoras, also,
\textit{the people of dreams}\footnote{The souls of the suitors are said by
Homer in the 24th book of the \textit{Odyssey} (v. 11) to have passed, in their
descent to the region of spirits, beyond \textit{the people of dreams.}} are
the souls which are said to be collected in the galaxy, this circle being so
called from the milk with which souls are nourished when they fall into
generation. Hence, those who evocate departed souls, sacrifice to them by a
libation of milk mingled with honey; because, through the allurements of
sweetness they will proceed into generation: with the birth of man, milk being
naturally produced.  Farther still, the southern regions produce small bodies;
for it is usual with heat to attenuate them in the greatest degree. But all
bodies generated in the north are large, as is evident in the Celt{\ae}, the
Thracians and the Scythians; and these regions are humid, and abound with
pastures. For the word Boreas is derived from the word which signifies
nutriment. Hence, also, the wind which blows from a land abounding in
nutriment, is called by a word describing being of a nutritive nature. From
these causes, therefore, the northern parts are adapted to the mortal tribe,
and to souls that fail into the realms of generation. But the southern parts
are adapted to that which is immortal,\footnote{Hence, the southern have always
been more favourable to genius, than the northern parts of the earth.} just as
the eastern parts of the world are attributed to the Gods, but the western to
d{\ae}mons. For, in consequence of nature originating from diversity, the
ancients everywhere made that which has a twofold entrance to be a symbol of
the nature of things. For the progression is either through that which is
intelligible or through that which is sensible. And if through that which is
sensible, it is either through the sphere of the fixed stars, or through the
sphere of the planets. And again, it is either through an immortal, or through
a mortal progression. One centre likewise is above, but the other beneath the
earth; and the one is eastern, but the other western. Thus, too, some parts of
the world are situated on the left, but others on the right hand; and night is
opposed to day. On this account, also, harmony consists of and
\textit{proceeds} through contraries. Plato also says that there are two
openings\footnote{See my translation of the tenth book of his
\textit{Republic}.} one of which affords a passage to souls ascending to the
heavens, but the other to souls descending to the earth. And according to
theologists, the Sun and Moon are the gates of souls, which ascend through the
Sun, and descend through the Moon. With Homer likewise, there are two tubs,

\begin{verse}
``From which the lot of every one he fills\\
Blessings to these, to those distributes ills.\footnote{\textit{Iliad}, lib.~\textsc{xiv} v.~528.}''
\end{verse}

\noindent But Plato in the \textit{Gorgias} by tubs intends to signify souls,
some of which are malefic, but others beneficent; and some which are rational,
but others irrational.\footnote{The passage in the Gorgias of Plato, to which
Porphyry here alludes, is as follows:

``Soc.: But, indeed, as you also say, life is a grievous thing. For I should
not wonder if Euripides spoke the truth when he says: `Who knows whether to
live is not to die, and to die is not to live?' And we perhaps are in reality
dead. For I have heard from one of the wise that we are now dead, and that the
body is our sepulchre; but that the part of the soul in which the desires are
contained, is of such a nature that it can be persuaded and hurled upwards and
downwards. Hence a certain elegant man, perhaps a Sicilian, or an Italian,
denominated, mythologising, this part of the soul a tub, by a derivation from
the probable and persuasive; and, likewise he called those that are stupid or
deprived of intellect, uninitiated. He further said that the intemperate and
uncovered nature of that part of the soul in which the desires are contained,
was like a pierced tub, through its insatiable greediness.''

What is here said by Plato is beautifully unfolded by Olympiodorus in his
\textsc{ms.}~\textit{Commentary on the Gorgias}, as follows: ``Euripides (in
\textit{Phryxo}) says, that to live is to die, and to die to live. For the soul
coming hither as she imparts life to the body, so she partakes [through this]
of a certain privation of life, because the body becomes the source of evils.
And hence, it is necessary to subdue the body.

``But the meaning of the Pythagoric fable which is here introduced by Plato, is
this: We are said to be dead, because, as we have before observed, we partake
of a privation of life. The sepulchre which we carry about with us is, as Plato
himself explains it, the body. But Hades is the unapparent, because we are
situated in obscurity, the soul being in a state of servitude to the body. The
tubs are the desires; whether they are so called from our hastening to fill
them as if they were tubs, or from desire persuading us that it is beauitiful.
The initiated, therefore, \textit{i.~e.,} those that have a perfect knowledge,
pour into the entire tub, for these have their tub full; or in other words,
have perfect virtue. But the uninitiated, \textit{viz.}, those that possess
nothing perfect, have perforated tubs. For those that are in a state of
servitude to desire always wish to fill it, and are more inflamed, and on this
account they have perforated tubs, as being never full. But the sieve is the
rational soul mingled with the irrational. For the [rational] soul is called a
circle, because it seeks itself, and is itself sought, finds itself and is
itself found. But the irrational soul imitates a right line, since it does not
revert to itself like a circle. So far, therefore, as the sieve is circular, it
is an image of the rational soul; but, as it is placed under the right lines
formed from the holes, it is assumed for the irrational soul. Right lines,
therefore, are in the middle of the cavities. Hence, by the sieve, Plato
signifies the rational in subjection to the irrational soul. But the water is
the flux of Nature; for as Heraclitus says, \textit{moisture is the death of
the soul.}''

In this extract the intelligent reader will easily perceive that the occult
signification of the tubs is more scientifically unfolded by Olympiodorus than
by Porphyry.} Souls, however, are [analogous to] tubs, because they contain in
themselves energies and habits, as in a vessel. In Hesiod, too, we find one tub
closed, but the other opened by Pleasure, who scatters its contents everywhere,
Hope alone remaining behind. For in those things in which a depraved soul,
being dispersed about matter, deserts the proper order of its essence, in all
these it is accustomed to feed itself with (the pleasing prospects of)
auspicious hope.


\paragraph{14} Since, therefore, every twofold entrance is a symbol of nature,
this Homeric cavern has, very properly, not one portal only, but two gates,
which differ from each other conformably to things themselves; of which one
pertains to Gods and good [d{\ae}mons], but the other to mortals and depraved
natures. Hence Plato took occasion to speak of bowls, and assumes tubs instead
of amphor{\ae}, and two openings, as we have already observed, instead of two
gates. Pherecydes Syrus also mentions recesses and trenches, caverns, doors and
gates: and through these obscurely indicates the generations of souls, and
their separation from these material realms.) And thus much for an explanation
of the Homeric cave, which we think we have sufficiently unfolded without
adducing any further testimonies from ancient philosophers and theologists,
which would give a needless extent to our discourse.


\paragraph{15} One particular, however, remains to be explained, and that is
the symbol of the olive planted at the top of the cavern, since Homer appears
to indicate something very admirable by giving it such a position. For he does
not merely say that an olive grows in this place, but that it flourishes on the
summit of the cavern.

\begin{verse}
``High at the head a branching olive grows,\\
Beneath, a gloomy grotto's cool recess.''
\end{verse}

\noindent But the growth of the olive in such a situation, is not fortuitous,
as some one may suspect, but contains the enigma of the cavern. For since the
world was not produced rashly and casually, but is the work of divine wisdom
and an intellectual nature; hence an olive, the symbol of this wisdom
flourishes near the present cavern, which is an image of the world. For the
olive is the plant of Minerva, and Minerva is wisdom. But this Goddess being
produced from the head of Jupiter, the theologist has discovered an appropriate
place for the olive by consecrating it at the summit of the port; signifying by
this that the universe is not the effect of a casual event and the work of
irrational fortune, but that it is the offspring of an intellectual nature and
divine wisdom, which is separated indeed from it [by a difference of essence],
but yet is near to it, through being established on the summit of the whole
port; [\textit{i.~e.,} from the dignity and excellence of its nature governing
the whole with consummate wisdom]. Since, however, an olive is
ever-flourishing, it possesses a certain peculiarity in the highest degree
adapted to the revolutions of souls in the world, for to such souls this cave
(as we have said) is sacred. For in summer the whiter leaves of the olive tend
upwards, but in winter the whiter leaves are bent downward. On this account
also in prayers and supplications, men extend the branches of an olive,
ominating from this that they shall exchange the sorrowful darkness of danger
for the fair light of security and peace. The olive, therefore being naturally
ever-flourishing, bears fruit which is the auxiliary of labour [by being its
reward]; it is also sacred to Minerva; supplies the victors in athletic labours
with crowns and affords a friendly branch to the suppliant petitioner. Thus,
too, the world is governed by an intellectual nature, and is conducted by a
wisdom eternal and ever-flourishing; by which the rewards of victory are
conferred on the conquerors in the athletic race of life, as the reward of
severe toil and patient perseverance. And the Demiurgus who connects and
contains the world [in ineffable comprehensions] invigorates miserable and
suppliant souls.


\paragraph{16} In this cave, therefore, says Homer, all external possessions
must be deposited. Here, naked, and assuming a suppliant habit, afflicted in
body, casting aside everything superfluous, and being averse to the energies of
sense, it is requisite to sit at the foot of the olive, and consult with
Minerva by what means we may most effectually destroy that hostile rout of
passions which insidiously lurk in the secret recesses of the soul. Indeed, as
it appears to me, it was not without reason that Numenius and his followers
thought the person of Ulysses in the \textit{Odyssey} represented to us a man,
who passes in a regular manner over the dark and stormy sea of generation, and
thus at length arrives at that region where tempests and seas are unknown, and
finds a nation

\begin{verse}
``Who ne'er knew salt, or heard the billows roar.''
\end{verse}


\paragraph{17} Again, according to Plato, the deep, the sea, and a tempest, are
images of a material nature. And on this account, I think, the poet called the
port by the name of Phorcys. For he says, ``It is the port of the ancient
marine Phorcys.\footnote{Phorcys is one among the ennead of Gods who, according
to Plato in the \textit{Tim{\ae}us}, fabricate generation. Of this deity,
Proclus observes, ``that as the Jupiter in this ennead causes the unapparent
divisions and separations of forms made by Saturn to become apparent, and as
Rhea calls them forth into motion and generation, so Phorcys inserts them in
matter, produces sensible natures, and adorns the visible essence, in order
that there may not only be divisions of productive principles [or forms] in
natures and in souls, and in intellectual essences prior to these, \textit{but
likewise in sensibles. For this is the peculiarity of fabrication.}''}'' The
daughter, likewise, of this God is mentioned in the beginning of the
\textit{Odyssey}.  But from Thoosa the Cyclops was born, whom Ulysses deprived
of sight. And this deed of Ulysses became the occasion of reminding him of his
errors, till he was safely landed in his native country. On this account, too,
a seat under the olive is proper to Ulysses, as to one who implores divinity,
and would appease his natal d{\ae}mon with a suppliant branch. For it will not
be simply, and in a concise way, possible for anyone to be liberated from this
sensible life, who blinds this d{\ae}mon, and renders his energies
inefficacious; but he who dares to do this, will be pursued by the
anger\footnote{\-``The anger of the Gods,'' says Proclus, ``is not an
indication of any passion in them, but demonstrates our inaptitude to
participate of their illuminations.''} of the marine and material Gods, whom it
is first requisite to appease by sacrifices, labours, and patient endurance; at
one time, indeed, contending with the passions, and at another employing
enchantments and deceptions, and by these, transforming himself in an
all-various manner; in order that, being at length divested of the torn
garments [by which his true person was concealed], he may recover the ruined
empire of his soul. Nor will he even then be liberated from labours; but this
will be effected when he has entirely passed over the raging sea, and, though
still living, becomes so ignorant of marine and material works [through deep
attention to intelligible concerns], as to mistake an oar for a corn-van.


\paragraph{18} It must not, however, be thought that interpretations of this
kind are forced, and nothing more than the conjectures of ingenious men; but
when we consider the great wisdom of antiquity, and how much Homer excelled in
intellectual prudence, and in an accurate knowledge of every virtue, it must
not be denied that he has obscurely indicated the images of things of a more
divine nature in the fiction of a fable. For it would not have been possible to
devise the whole of this hypothesis, unless the figment had been transferred
[to an appropriate meaning] from certain established truths. But reserving the
discussion of this for another treatise, we shall here finish our explanation
of the present Cave of the Nymphs.


\end{document}
