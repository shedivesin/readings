\documentclass[12pt]{article}
\usepackage{fontspec}
\usepackage{microtype}
\setmainfont[Ligatures=TeX,Numbers=Lowercase]{EB Garamond}

\title{On D{\ae}mons}
\author{Proclus\footnote{Commentary on the First Alcibiades of Plato.} \and
Thomas Taylor (tr.)\footnote{Excerpted from T.~Taylor, additional notes to
``Dissertation \textsc{xxvi}: What the D{\ae}mon of Socrates Was'' and
``Dissertation \textsc{xxvii}: Again, Concerning the D{\ae}mon of Socrates,''
in \textit{The Dissertations of Maximus Tyrius,} vol.~2, 1804.}}
\date{}

\begin{document}
\maketitle

\noindent Let us now speak, in the first place, concerning d{\ae}mons in
general; in the next place, concerning those that are allotted us in common;
and in the third place concerning the d{\ae}mon of Socrates. For it is always
requisite that demonstrations should begin from things more universal, and
proceed from these as far as to individuals. For this mode of proceeding is
natural, and is more adapted to science. D{\ae}mons, therefore, deriving their
first subsistence from the vivific goddess,\footnote{Juno.} and flowing from
thence as from a certain fountain, are allotted an essence characterized by
soul. This essence in those of a superior order is more intellectual and more
perfect according to hyparxis;\footnote{The summit of essence.} in those of a
middle order, it is more rational; and in those which rank in the third degree,
and which subsist at the extremity of the d{\ae}moniacal order, it is various,
more irrational and more material. Possessing therefore an essence of this
kind, they are distributed in conjunction with the gods, as being allotted a
power ministrant to deity. Hence they are in one way subservient to the
liberated gods\footnote{Gods who immediately subsist above the mundane deities,
and are therefore called supercelestial.} who are the leaders of wholes prior
to the world; and in another to the mundane gods, who proximately preside over
the parts of the universe. For there is one division of d{\ae}mons, according
to the twelve supercelestial gods, and another according to all the idioms of
the mundane gods. For every mundane god is the leader of a certain
d{\ae}moniacal order, to which he proximately imparts his power;
\textit{viz.}~if he is a demiurgic god, he imparts a demiurgic power; if
immutable an undefiled power; if telesiurgic, a perfective power. And about
each of the divinities, there is an innumerable multitude of d{\ae}mons, and
which are dignified with the same appellations as their leading gods. Hence
they rejoice when they are called by the names of Jupiter, Apollo, and Hermes,
\&c.~as expressing the idiom, or peculiarity of their proper deities: and from
these, mortal natures also participate of divine influxions. And thus animals
and plants are fabricated, bearing the images of different gods; d{\ae}mons
proximately imparting to these the representations of their leaders. But the
gods in an exempt manner supernally preside over d{\ae}mons; and through this,
last natures sympathize with such as are first. For the representations of
first are seen in last natures; and the causes of things last are comprehended
in primary beings. The middle genera too of d{\ae}mons give completion to
wholes, the communion of which they bind and connect; participating indeed of
the gods, but participated by mortal natures. He therefore will not err who
asserts that the mundane artificer established the centers of the order of the
universe, in d{\ae}mons; since Diotima also assigns them this order, that of
binding together divine and mortal natures, of deducing supernal streams,
elevating all secondary natures to the gods, and giving completion to wholes
through the connection of a medium. We must not therefore assent to their
doctrine, who say that d{\ae}mons are the souls of men, that have changed the
present life. For it is not proper to consider a d{\ae}moniacal nature
\textit{according to habitude} as the same with a nature \textit{essentially}
d{\ae}moniacal, nor to assert that the perpetual medium of all mundane natures
consists from a life conversant with multiform mutations. For a d{\ae}moniacal
guard subsists always the same, connecting the mundane wholes; but soul does
not always thus retain its own order, as Socrates says in the Republic; since
at different times, it chooses different lives. Nor do we praise those, who
make certain of the gods to be d{\ae}mons, such as the erratic gods, according
to Amelius; but we are persuaded by Plato, who calls the gods the rulers of the
universe, but subjects to them the herds of d{\ae}mons; and we shall every
where preserve the doctrine of Diotima, who assigns the middle order, between
all divine and mortal natures, to a d{\ae}moniacal essence. Let this then be
the conception respecting the whole of the d{\ae}moniacal order in common.

In the next place, let us speak concerning the d{\ae}mons which are allotted
mankind. For of the d{\ae}mons which, as we have said, rank in the middle
order, the first and highest are divine d{\ae}mons, and who often appear as
gods, through their transcendent similitude to the divinites. For in short,
that which is first in every order, preserves the form of the nature prior to
itself. Thus the first intellect is a god, and the most ancient of souls is
intellectual: and hence of d{\ae}mons the highest genus, as being proximate to
the gods, is uniform and divine. The next to these in order, are those
d{\ae}mons who participate of an intellectual idiom, and preside over the
ascent and descent of souls, and who unfold into light and deliver to all
things the productions of the gods. The third are those who distribute the
productions of divine souls to secondary natures, and complete the bond of
those that receive defluxions from thence. The fourth are those that transmit
the efficacious powers of whole natures to things generated and corrupted, and
who inspire partial natures with life, order, reasons, and the all-various
perfect operations, which things mortal are able to effect. The fifth are
corporeal, and bind together the extremes in bodies. For how can perpetual
accord with corruptible bodies, and efficients with effects, except through
this medium? For it is this ultimate middle nature which has dominion over
corporeal goods, and provides for all natural prerogatives. The sixth in order,
are those that revolve about matter, connect the powers which descend from
celestial to sublunary matter, perpetually guard this matter, and defend the
shadowy representations of forms which it contains.

D{\ae}mons therefore, as Diotima also says, being many and all-various, the
highest of them conjoin souls proceeding from their father, to their leading
gods: for every god as we have said, is the leader in the first place of
d{\ae}mons, and in the next of partial souls. For the Demiurgus disseminated
these, as Tim{\ae}us says, into the sun and moon, and the other instruments of
time. These divine d{\ae}mons therefore, are those which are essentially
allotted to souls, and conjoin them to their proper leaders: and every soul
though it revolves together with its leading deity requires a d{\ae}mon of this
kind. But d{\ae}mons of the second rank preside over the ascensions and
descensions of souls; and from these the souls of the multitude derive their
elections. For the most perfect souls who are conversant with generation in an
undefiled manner, as they choose a life conformable to their presiding god, so
they live according to a divine d{\ae}mon, who conjoined them to their proper
deity, when they dwelt on high. Hence the Egyptian priest admired Plotinus, as
being governed by a divine d{\ae}mon. To souls, therefore who live as those
that will shortly return to the intelligible world whence they came, the
supernal is the same with the d{\ae}mon which attends them here; but to more
imperfect souls the essential is different from the d{\ae}mon that attends them
at their birth.

If these things then are rightly asserted, we must not assent to those who make
our rational soul a d{\ae}mon. For a d{\ae}mon is different from man, as
Diotima says, who places d{\ae}mons between gods and men, and as Socrates also
evinces, when he divides a d{\ae}moniacal oppositely to the human nature: for,
says he, not a human, but a d{\ae}moniacal obstacle detains me. But man is a
soul using the body as an instrument. A d{\ae}mon, therefore, is not the same
with the rational soul.

This also is evident from Plato in the Tim{\ae}us, where he says that intellect
has in us the relation of a d{\ae}mon. But this is only true as far as
pertains to analogy. For a d{\ae}mon according to essence, is different from a
d{\ae}mon according to analogy. For in many instances that which proximately
presides, subsisting in the order of a d{\ae}mon with respect to that which is
inferior, is called a d{\ae}mon. Thus Jupiter in Orpheus, calls his father
Saturn an illustrious d{\ae}mon, and Plato in the Tim{\ae}us, calls those gods
who proximately preside over, and orderly distribute the realms of generation,
d{\ae}mons: ``for,'' says he, ``to speak concerning other d{\ae}mons, and to
know their generation, exceeds the ability of human nature.'' But a d{\ae}mon
according to analogy is that which proximately presides over any thing, though
it should be a god, or though it should be some one of the natures posterior to
the gods. And the soul, that through similitude to the d{\ae}moniacal genus
produces energies more wonderful than those which belong to human nature, and
which suspends the whole of its life from d{\ae}mons, is a d{\ae}mon according
to habitude, \textit{i.~e.}~proximity or alliance. Thus, as it appears to me,
Socrates in the Republic calls those, d{\ae}mons, who have lived well, and who,
in consequence of this are transferred to a better condition of being, and to
more holy places. But an essential d{\ae}mon, is neither called a d{\ae}mon
through habitude to secondary natures, nor through an assimilation to something
different from itself; but is allotted this peculiarity from himself, and is
defined by a certain summit, or flower of essence (hyparxis) by appropriate
powers, and by different modes of energies. In short, the rational soul is
called in the Tim{\ae}us the d{\ae}mon of the animal. But we investigate the
d{\ae}mon of man, and not of the animal; that which governs the rational soul
itself, and not its instrument; and that which leads the soul to its judges,
after the dissolution of the animal, as Socrates says in the Ph{\ae}do. For
when the animal is no more, the d{\ae}mon which the soul was allotted while
connected with the body, conducts it to its judge. For if the soul possesses
that d{\ae}mon while living in the body, which is said to lead it to judgement
after death, this d{\ae}mon must be the d{\ae}mon of the man, and not of the
animal alone. To which we may add, that beginning from on high it governs the
whole of our composition.

Nor again, dismissing the rational soul, must it be said that a d{\ae}mon is
that which energizes in the soul: as for instance, that in those who live
according to reason, reason is the d{\ae}mon; in those that live according to
anger, the irascible part; and in those that live according to desire, the
desiderative part. Nor must it be said that the nature which proximately
presides over that which energizes in our life, is a d{\ae}mon: as for
instance, that reason is the d{\ae}mon of the irascible, and anger of those
that live according to desire. For in the first place to assert that d{\ae}mons
are parts of our soul, is to admire human life in an improper degree, and
oppose the division of Socrates in the Republic, who after gods and d{\ae}mons
places the heroic and human race, and blames the poets for introducing in their
poems heroes in no respect better than men, but subject to similar passions. By
this accusation therefore it is plain that Socrates was very far from thinking
that d{\ae}mons who are of a sublimer order than heroes are to be ranked among
the parts and powers of the soul. For from this doctrine it will follow that
things more excellent according to essence give completion to such as are
subordinate. And in the second place, from this hypothesis, mutations of lives
would also introduce multiform mutations of d{\ae}mons. For the avaricious
character is frequently changed into an ambitious life, and this again into a
life which is formed by right opinion, and this last into a scientific life.
The d{\ae}mon, therefore, will vary according to these changes: for the
energizing part will be different at different times. If therefore, either this
energizing part itself is a d{\ae}mon, or that part which has an arrangement
prior to it, d{\ae}mons will be changed together with the mutation of human
life; and the same person will have many d{\ae}mons in one life, which is of
all things the most impossible. For the soul never changes in one life the
government of its d{\ae}mon; but it is the same d{\ae}mon which presides over
us till we are brought before the judges of our conduct, as also Socrates
asserts in the Ph{\ae}do.

Again, those who consider a partial intellect, or that intellect which subsists
at the extremity of the intellectual order, as the same with the d{\ae}mon
which is assigned to man, appear to me to confound the intellectual idiom, with
the d{\ae}moniacal essence. For all d{\ae}mons subsist in the extent of souls,
and rank as the next in order to divine souls, and is neither allotted the same
essence, nor power, nor energy.

Further still, this also may be said, that souls enjoy intellect then only when
they convert themselves to it, receive its light, and conjoin their own with
intellectual energy; but they experience the presiding care of a d{\ae}moniacal
nature, through the whole of life, and in every thing which proceeds from fate
and providence. For it is the d{\ae}mon that governs the whole of our life, and
that fulfils the elections which we made prior to generation, together with the
gifts of fate, and of those gods that preside over fate. It is likewise the
d{\ae}mon that supplies and measures the illuminations from providence. And as
souls indeed, we are suspended from intellect, but as souls using the body, we
require the aid of a d{\ae}mon. Hence Plato, in the Ph{\ae}drus, calls
intellect the governor of the soul; but he every where calls a d{\ae}mon the
inspector and guardian of mankind. And no one who considers the affair rightly,
will find any other one and proximate providence of every thing pertaining to
us, besides that of a d{\ae}mon. For intellect, as we have said, is
participated by the rational soul, but not by the body; and nature is
participated by the body, but not be the dianoetic part. And further still, the
rational soul rules over anger and desire, but it has no dominion over
fortuitous events. But the d{\ae}mon alone moves, governs, and orderly disposes
all our affairs. For he gives perfection to reason, measures the passions,
inspires nature, connects the body, supplies things fortuitous, accomplishes
the decrees of fate, and imparts the gifts of providence. In short, he is the
king of every thing in and about us, and is the pilot of the whole of our
life. And thus much concerning our allotted d{\ae}mons.

In the next place, with respect to the d{\ae}mon of Socrates, these three
things are to be particularly considered. First, that he not only ranks as a
d{\ae}mon, but also as a god: for in the course of this dialogue he clearly
says, ``I have long been of opinion that \textit{the god} did not as yet permit
me to hold any conversation with you.''

He calls the same power, therefore, a d{\ae}mon and a god. And in the Apology,
he more clearly evinces that this d{\ae}mon is allotted a divine transcendency,
considered as ranking in a d{\ae}moniacal nature. And this is what we before
said, that the d{\ae}mons of divine souls, and who make choice of an
intellectual and anagogic life, are divine, transcending the whole of a
d{\ae}moniacal genus, and being the first participitants of the gods. For as is
a d{\ae}mon among gods, such also is a god among d{\ae}mons. But among the
divinities the essence is divine; but in d{\ae}mons, on the contrary the idiom
of their essence is d{\ae}moniacal, but the analogy which they bear to divinity
evinces their essence to be godlike. For on account of their transcendency with
respect to other d{\ae}mons, they frequently appear as gods. With great
propriety, therefore, does Socrates call his d{\ae}mon a god: for he belonged
to the first and highest d{\ae}mons. Hence Socrates was most perfect, being
governed by such a presiding power, and conducting himself by the will of such
a leader and guardian of his life. This then was one of the illustrious
prerogatives of the d{\ae}mon of Socrates. The second was this: that Socrates
perceived a certain voice proceeding from his d{\ae}mon. For this is asserted
by him in the The{\ae}tetus and in the Ph{\ae}drus. And this voice is the
signal from the d{\ae}mon, which he speaks of in the Theages; and again in the
Ph{\ae}drus, when he was about to pass over the river, he experienced the
accustomed signal from the d{\ae}mon. What then does Socrates indicate by these
assertions, and what was the voice, through which he says the d{\ae}mon
signified to him his will?

In the first place, we must say, that Socrates through his dianoetic power, and
his science of things, enjoyed the inspiration of his d{\ae}mon, who
continually recalled him to divine love. In the second place, in the affairs of
life, Socrates supernally directed his providential attention to more imperfect
souls; and according to the energy of his d{\ae}mon, he received the light
proceeding from thence, neither in his dianoetic part alone, nor in his
doxastic\footnote{The powers belonging to \textit{opinion}, or that part of the
soul which knows \textit{that} a thing is, but not \textit{why} it is.} powers,
but also in his spirit, the illumination of the d{\ae}mon, suddenly diffusing
itself through the whole of his life, and now moving sense itself. For it is
evident, that reason, imagination, and sense enjoy the same energy differently;
and that each of our inward parts is passive to, and is moved by the d{\ae}mon
in a peculiar manner. The voice, therefore, did not act upon Socrates
externally with passivity; but the d{\ae}moniacal inspiration proceeding
inwardly through his whole soul, and diffusing itself as far as to the organs
of sense, became at last a voice, which was rather recognized by consciousness,
than by sense: for such are illuminations of good d{\ae}mons, and the gods.

In the third place, let us consider the peculiarity of the d{\ae}mon of
Socrates: for it never exhorted, but perpetually recalled him. This also must
again be referred to the Socratic life: for it is not a property common to our
allotted d{\ae}mons, but was the characteristic of the guardian of Socrates. We
must say, therefore, that the beneficent and philanthropic disposition of
Socrates, and his great promptitude with respect to the communication of good,
did not require the exhortation of the d{\ae}mon. For he was impelled from
himself, and was ready at all times to impart to all men the most excellent
life. But since many of those that came to him were unadapted to the pursuit of
virtue and the science of wholes, his governing good d{\ae}mon restrained him
from a providential care of such as these. Just as a good charioteer alone
restrains the impetus of a horse naturally well adapted for the race, but does
not stimulate him, in consequence of his being excited to motion from himself,
and not requiring the spur, but the bridle. And hence Socrates, from his great
readiness to benefit those with whom he conversed, rather required a recalling
than an exciting d{\ae}mon. For the unaptitude of auditors which is for the
most part concealed from human sagacity requires a d{\ae}moniacal
discrimination; and the knowledge of favorable opportunities, can by this alone
be accurately announced to us. Socrates therefore being naturally impelled to
good, alone required to be recalled in his unseasonable impulses.

But further still, it may be said that of d{\ae}mons, some are allotted a
purifying and undefiled power; others a generative; others a perfective; and
others a demiurgic power: and in short they are divided according to the
characteristic peculiarities of the gods, and the powers under which they are
arranged. Each, likewise, according to his essence incites the object of his
providential care to a blessed life; some of them moving us to an attention to
inferior concerns, and others restraining us from action, and an energy verging
to externals. It appears therefore, that the d{\ae}mon of Socrates being
allotted this peculiarity, \textit{viz.}~cathartic, and the source of an
undefiled life, and being arranged under this power of Apollo, and uniformly
presiding over the whole of purification, separated also Socrates from too much
commerce with the vulgar, and a life extending itself into multitude. But it
led him into the depths of his soul, and an energy undefiled by subordinate
natures: and hence it never exhorted, but perpetually recalled him. For what
else is to recall than to withdraw from the multitude to inward energy? And of
what is this the peculiarity except of purification? Indeed it appears to me
that as Orpheus places the Apolloniacal monad over king Bacchus, which recalls
him from a progression into Titanic multitude, and a desertion of his royal
throne, in like manner the d{\ae}mon of Socrates conducted him to an
intellectual place of survey, and restrained his association with the
multitude. For the d{\ae}mon is analogous to Apollo, being his attendant, but
the intellect of Socrates to Bacchus: for our intellect is the progeny of the
power of this divinity.

\end{document}
